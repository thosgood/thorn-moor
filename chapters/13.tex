%!TEX root = ../thorn-moor.tex

I started this day to collect an architect from Hilltop in Plymouth and visited
a generator unit that is powered in a Devon and Cornwall Police Hilltop Radio
station. My companion was named Tom Jane.    We knew each other well as we
had carried out many of such inspections so conversation was always great but a
happening occurred which affected both of us after we looked at the generator
and started our quite long descent to return. I was driving quite slowly so as
not to cause damage as the path was very deeply rutted with small loose
boulders. Looking in the distance I could see a spiralling cloud of dust
moving very fast. Tom's voice said in alarm, ``Better pull in here where there
is a wide area''. We stopped and waited. The drive was in another world as a
van passed. The energy from it was unforgettable. We remarked to each other
that we would not like to be on top of the hill when he gets there.

Next morning Radio Devon News mentioned that a person had committed suicide on
Kit Hill and was asking if anyone had any information to contact Lostwithial
Police. I did give full particulars of what we had seen. The response was,
``Goodness! You are spot on. How did you report that information in such a
short time?'' I told them that the energy coming from that van was disturbing.
I could not help but see there was big trouble of some sort.

Whilst talking to the Police, let's talk of the policing of Thorn Moor, Cheriton
Bishop. The area policeman was a typical comic book build, tubby, red faced
and kindly man, named PC Tonkins who lived in the Police house at Tedburn St
Mary but to me a small boy was always in the right place of minor offences
committed by small boys.

One day my friend did not have a bike so not wishing him to walk I gave him a
lift. We arrived with David sitting on my rear carrier. The shock of seeing
PC Tonkins cycling towards me caused me to lose control and run up a very steep
bank and fall backwards with the bike on top of both of us. PC Tonkins stood
in the road and exclaimed that that demonstrated the reason for not riding 2
people on a bike. He waited to see if we were OK before continuing his
journey. Luckily it was outside of David's house and I continued to cycle
home. Sometimes we would see PC Tonkins ride past our gate on his way to his
favourite cider cellar and later in the day riding in a car with his bike on
the back having a very necessary lift home.

Some years later PC Tonkins retired and was replaced by a PC Cannon immediately
renamed by the locals as ``bang bang''. Big enquiries were made about his
character. I wanted to know whether or not he let minor deeds go unnoticed. I
also wanted to find out whether he enjoyed paperwork as if he did this would
not be good as he would probably make out a ticket for blowing our nose.

One evening I was aged about 16 and was walking quietly beside a hedge leading
to a gate to the road. As I got to the gate I checked again that the gun I was
carrying was empty and left open. Prior to opening the gate and stepping on to
the road I was confronted by PC Cannon or should I say Bang Bang. He had
remained hidden until I stepped on to the road. He greeted me with the
following ``Let's see, your name is Michael?'' ``Yes sir'', I replied. ``Let's
see'' he said ``Is that a 12 bore shot gun?'' Again I replied, ``Yes sir. It
needs a shot gun licence like this one'' I said pulling the very item from my
top pocket and handing it over. PC Cannon said. ``Ah yes. I thought you of
all people would have one.'' I thought, I expect he was hoping that I did not.
I watched him read through the form and I thought that he was a paper man so
nothing out of place will go unnoticed so watch out farmers in this area.

A rumour circulated that this PC had been relieved after the war as a prisoner
of the Japanese Death Railway. I thought that if that was the case, I would
forgive him his general unforgiving conduct.

Some years later my friend George Gillard and I were asked by PC Alan Parsley of
Whiddon Down if we would become Special Constables to support him with traffic
or, as happened at the time, check points following convicts' escape from
Dartmoor Prison.  This we both did and spent many winter nights having
training at Moretonhampstead Town Hall. One evening when a member arrived late
he told us that President Kennedy had been assassinated. We all stood in
disbelief.

I was called to help a regular Constable with a road check at Beter Cross on
Dartmoor commencing at 7 pm until midnight. We sat in my car as the night grew
cold with heavy fog due so we could hear a car from miles away and would get
out of the car standing on each side of the road and wave our torches at the
car to stop. Some would not wish to stop as our torch lights would reveal
their passenger to be someone other than his wife. Not our care so we thanked
them for stopping and waved them on their way. In the late 1950s I was led to
understand convicts would escape to pay back a debt to the tobacco baron and
the longer they could stay out and cause disruption the more the debt was
reduced.

When I left Caterpillar and joined Savilles it was purely to stop living out of
a suitcase and working excessively long hours so that I could spend quality
time at home with my Vera. The Company had the International Harvester Crawler
dealership. In the first 2 years I was there I could see trade falling with
the lack of International machine sales until the Company became a Lister parts
dealer for the South of England. This opened up new avenues. I got the
contract to carry all Hilltop Radio Station general enquiries for Devon \&
Cornwall Police.

We were then made MAN and Volkeswagon commercial dealers for the nation
resulting in a major workshop expansion and myself building a team to
specialise in Lister generators, MAN trucks and existing track dozers and
loaders. Fortunately we had a good base of apprentices so ended up with a very
strong and knowledgeable team of men to the extent that the workshop and parts
team won the International dealership championship for 5 consecutive years,
once attending Castle Combe race track and driving Formula Ford race cars.

One day I was asked to assist in the filming of Lion of the Desert as there were
no Fiat or MAN track vehicles and for work in the desert, real track vehicles
were required. I was shown a print of an original tank as they required an
immediate answer on the spot in the office from me. Yes was given and I drove
back from Cornwall designing the complete unit in my head. I stepped out of
the car at Exeter Depot and requested a used International 125 from the sales
fleet, drove home in the evening and commenced to carry out drawings covering
the intended conversion. The next day the crew continued with normal
dealership duties whilst 2 service men cleaned at speed and washed the floor
which had dried whilst the donor dozer was stripped of dozer blade, all frame
work, operator cab, fuel tank and bonnet. By evening the floor was dried out
and the track power unit squared into position. The 3 men assistants were sent
home for tea anticipating that they would have a complete mindset. In the
meantime I carried out chalk drawings on the floor using nothing more than a
tape measure, square and plain metal straight edges. You must understand as it
came out of my head there was no heavy or sophisticated calculation but I knew
of the benefit of the likeness to the track configuration must follow the
original drawings. When completed the film crew came down to Exeter with our
hearts thumping to test in a very rough field and that afternoon well proved
itself so the heart thumping changed to back slapping. You can imagine the
sales team thinking Christmas had come again when the film company ordered the
conversion of 5 extra units. I understand they went on their way to carry out
the filming. You will find Lion of the Desert in detail on Google and if you
scroll down to ‘military equipment used' you will see a full photo of my Fiat 2
MAN tank conversions with various comments from military experts still trying
to identify the type of tank used. It was nothing like they guessed it to be
but came out of my head and was made by the boys of Exeter from an
International 125 so years passed and the holding financial Company started
closing and made all the MAN engineers redundant but retaining some of the
generator maintenance engineers. I was then made redundant. Fortunately having
observed the action taken over the past months I had a position waiting for me.
When I walked into the Job Centre it became apparent that I was the wise man
with still a young attitude. I thought what a pity it was that the Job Centre
employed people unable to identify and communicate by using a different
attitude with people that had worked all their lives and were made redundant
for the first time and those with no knowledge of what to do regarding
redundancy and the inefficient waster.

End of Saville employment and on to Sleeman Hawken.

Caterpillar Geneva Courses: Having spent a time with Bowmaker I was sent to
Geneva on a series of excellent and in depth Caterpillar engineering courses on
power shift transmissions, hydraulics 977, torque converters and turbochargers.
Then some time after having carried out sea trials to the D333 CAT engines for
Morgan Giles, on the Monaco 489 built at Teignmouth I was sent to Geneva on a
Marine Power analyst course of 2 weeks. I will cover this enchanting period.
The CAT school was located out of the town up in the hills a complex looking
into the mountain of Mont Blank. The tutors were Swiss and the class consisted
of dealership engineers from all over Europe --- a Spaniard, 2 Belgians, a
Danish gentleman, a Phillipino and myself.

The first day was spent analysing engine output or lack of it, the remainder
explaining that a hull designed to a particular shape and long ton weight could
not obtain a speed designed, for instance, a hull with a power of 500 hp unit
with a maximum design speed of 7 knots will never go over that speed if the
engine is replaced with a 750 hp unit. Also propeller correct size is vital
for this. We were presented with a special set of slide rules and taught
(with difficulty) what vital specification to ask for before coming to work out
propeller size on the slide rules. Why go to such depths you might ask. Well
if you are the CAT engine test engineer on the sea trial and the engine reaches
full rpm and power, everyone gets a pat on the back and completes the sea
trials with a big smile. If not and the rpm was below specification, open
warfare would erupt in the wheel house between the designer, boat builder and
the Cat man --- yours truly.

This Course was to furnish us with details of the correct question we should
ask, ie what weight did the boat finish up with at the build finish etc and
using the slide rules supplied calculate to prop size.  Regarding this we
could stand our ground and respond with the useful information required.

The weekend came and no let up. We were paired up, I with the Spaniard and
given a problem that needed a propeller calculation using the slide rules –
heaven forbid that I was always lagging behind the very switched on Belgian
and needed a night's rest to think things over. We got to the hotel and my
partner, the Spaniard said ``Bye Mike I am here to enjoy myself --- see you
Monday morning''.

Saturday morning with the warm sun shining on me I walked with homework the 200
meters from the hotel to the Jet d'Eau on the lake (you are going to feel sorry
for me) as I sat beside this lovely lake environment looking at the d'Eau I was
reminded of the jetting on the lake surface to the top of the nozzle Jet to now
140 meters with a volume of the nozzle of 7 tons of water. Working alone on
the problem I had a wonderful time as a local old man offered me a ride in his
boat around the lake. It ended up a very delightful 2 days. Back to school
questions were asked to the teams of 2 and it arrived at us before any
questions could be presented by the tutor. My Spanish friend said loudly, ``As
Mike is an English officer and a gentleman I left him with the problem.'' Well,
I took the full force of the questions and corrections had to be made. I felt
I came out wiser but somewhat sand blasted.

Out of this became a better understanding and the desire to think about
imagining faults and possible remedies thus I had spare time to dream.

Bye bye Savilles --- Good day Sleeman Hawkins: I commenced this job working
from home working with the Company as a Parts Engineer and Marine sales. I
could carry out my customer PR visits with great confidence knowing that any
Lister Petter parts ordered would be with the customer within a day or so.

The area I travelled was Cornwall to Poole RNLI Headquarters at Malmsbury and
Wales M4 and the Gower Peninsula. Vera and I would load the van with 3 Lister
Marine engines and show kit and depart on Sundays or Bank Holidays in the
morning, arrive at Newlyn fish market and build up a complete marine engine
show stand.

We were ready for the Newlyn fish festival that commenced on Bank Holiday Monday
a very popular event that drew in approximately 22,000 visitors. That day
engines sold would be followed up with a visit to the installation site. How
useful my past training in Geneva came into play. We would follow by having a
stand in the NEC at Birmingham and one in Lyme Regis. Eventually I retired
with Vera following 2 years later.

I was well pleased to be able to load 3 Lister marine engines and show kit into
my Astra van. This was achieved because I owned a three piece breakdown engine
lift and on weekends was able to check out Health and Safety with fingers on
the buzzers, ie I manufactured an extension on the lifting arm so angled that
it lay parallel with the engine thus manoeuvring the clearance between the
engine rocker cover to the height of the van roof. With my minimising engine
lift I was self-contained and able to load.

After retirement I got a piece of allotment and was given an old rusty disused
Merrytiller rotavator minus the engine. I purchased and used the overpowered
Honda 5 hp and it operated the unit at just over tick over and returned an
excellent job.

When Vera and I enjoyed our Ruby wedding anniversary we invited friends and
relatives to a Sunday lunch with a request for no presents but if they wished
to put a donation for the Devon Air Ambulance in a bucket, that would be
appreciated. The following morning we delivered the bucket  to the Devon Air
Ambulance HQ to be opened. It was an exceptional surprise to find generous
amounts of money had been donated. We were then asked if we could help as
there was a shortage of speakers. Vera said, ``Mike will be a speaker''. So
from then on I became a speaker and Vera and I always worked as a Devon Air
Ambulance team of speakers, Vera operating the Power Point system whilst I
spoke. This we continued to do until the Covid 19 lockdown. In the previous
16 years we had given talks at the length and breadth of Devon manning stalls
at shows and in the latter period Vera worked in the garden whilst I spent time
in the garage/workshop making leather swords, fobs and belts as I had taught
myself leather work at the beginning of my story when I mentioned a very young
John Anstee who would spend much time with me as an engineer who controls a
welding and engineering business of his own. We spent many hours discussing
engineering projects from the latest to the old types of problems and carried
out modifications to a great friend's son's new kit car.

I also remember being in daily contact with Gordon Long. One day I was
contacted by Poltimore Church warden in desperation for a replacement old
fashioned church key. I told him that I would make a replacement. Having
never made a key before I thought this should be interesting so I examined the
old 8 feet long item and selected the metal required. I set forth with quite
an amount of hand hacksawing followed by an equal amount of small filing and it
came to pass that a key presented itself. I ended up making 2 or 3 more for
Poltimore and one for Huxham Church. I returned them with the completion of
the regular 2.5 inches and ¼'' key rings as I remembered at Sunday School in
Cheriton Bishop Mrs Hill explaining that all church keys should have a 2.5 inch
key ring for should the bridegroom or best man forget the wedding ring the
vicar would call the Church Warden to hand over the church key and the special
ring was placed on the bride's finger and duly blessed so three good things
came out of the manufacturing of the keys. The Church Committees were not
overladen with the cost of a replacement key though these keys were approved of
by the old customers. Two Church Wardens had been re-educated to cope with
modern day weddings should the unthinkable happen.

Vera and I have had great fortune in having such lovely people around us to keep
our early retirement alive, such as Julie, first cousin to Vera and Godmother
to Julie's daughter, and her husband Richard, who both chaperoned us to five
unbelievable holidays on Lake Garda and then to Venice and to Florence in
Tuscany and kept a close watch on us at all times. Also John and Ann Anstee
now identifying that the old joints need an extra grease nipple fitted together
with an extra squirt of WD40, collect us and take us for Sunday lunch and the
everlasting friendship of Judy and Gordon Long ever ready to present a meal and
our very many friends and acquaintances.

To cast my mind back in time I was introduced to John Anstee by his Mum, Elsie
when I was 15 years of age. He was a babe in arms and as you can imagine we
lost touch over the years until the 1990s when John carried out a specialist
stainless steel welding upgrade on my Super Seal 26 boat rudder. Now standing
before me is this very experienced welder with all metal type experience. From
that time on we sailed on the south coast and carried out mooring repairs at
the Ham, Dittisham thus extending our happy boating life until Covid 19 hit the
world.

We were very lucky because we Caterpillar men, known as the CAT men had not met
for 49 years so I set forth with pen and phone and arranged a reunion on 15
February 2020 which developed into a great event at the Beefeater, Middlemoor
with Devon Life sitting with us and interviewing us all and put us in Google.
As a band of brothers we had the following boys, Joe Morrish aged 94 who was
Depot Service Controller, Jack Carter, Depot Parts Manager and Colin Wroth,
Engine Rebuild Supervisor. Also there was Bob Richards aged 70ish, Depot
Overseas Service Engineer, John Cole aged 80, Depot and overseas Service
Engineer, Johnny Blackmore aged 72 and myself as Field Service Engineer. We
both worked the length of theM5, part of the M4, Telegraph Hill and I did a
section of the Plymouth Highway whilst Johnny Blackmore accompanied by the late
Tony Butt continued to manage the Plympton site and sometimes I would break
from duty to engage on marine work as discussed earlier in my story.

My Uncle Ronald said, ``The trouble with folks moving down to Devon is that they
think we Devonians are daft because we have no sense!''
