%!TEX root = ../thorn-moor.tex

Working back in time one morning Mr Saunders asked me to take a look at a large
granite roller with a broken main pin with the following remark ``this is a
repair that was executed in the dark ages.  I have never carried out such a
repair myself but I know how so I will give you step by step moves.''

First step, light the blow torch and melt the retaining lead until all clear,
gently pull out the broken shaft together with the many retaining wedges until
the stone hole is clean.  Second step: cut a piece of 3/4''round bar and put
one end in the forge and smith a mushroom square, put shaft in stone hole and
pack the sides with 4 mushroom metal 5'' pieces.  Third step:  manufacture by
forge many wedges thin enough to curl around the larger metal to cause and
complete lock up.  It is very important that you listen to the sound of the
hammer to stop high pressure causing severely strong damage.  Fourth step:  Dig
a hole in the ground to take half the length of the stone roller.  Final step:
Heat the metal pin together with retaining wedges hot enough to be tinned with
the lead that is heated to liquid in a pouring ladle (if the lead spits when
pouring, heat up the pin more as it is not hot enough to pour the lead.  Keep
on until it looks like liquid.  Leave to cool for many hours and return to the
customer.

The secret sign of the Water Team:  one day after lunch Mr Saunders asked me to
get in the car as we were visiting the water gang working in a very isolated
farmstead.  We were to look at designing a small cattle grid.  We arrived and
proceeded to walk across 2 fields to locate the men and the mole ploughing
exercise.  Suddenly Mr Saunders barked (quote) ``What is the matter with our
lorry driver, Arthur Lock.''  I looked in the direction of his pointed finger
and observed Arthur standing as high as possible lifting his cap up and down on
the top of his head and looking in the direction Arthur was facing, a chain
reaction set in.  Other members of the team commenced to repeat the signal.  I
of course said I had no idea, well knowing it was the team's warning sign of
authority approaching.  We continued the visit taking measurements of the grid.
Nothing was said about the incident but I am sure the boss had worked out the
sign because when we returned and opened the doors of the Triumph Herald to sit
in John Saunders removed his cap and repeated the signal whilst starting the
car, a slight smile on his face and nothing more was said.

The next morning having made the extra early start found me being collected for
Jersey airport by an engineer of south Pier Shipyard who explained the set up
by the harbour.  All owners of Fairy power boats were invited to what is
described as a birthday party for the new Fairy Swordsman named Point One.  The
plan was that all owners and their ladies congregate with the new owner at St
Hellier harbour to have drinks and as a group make passage to St Marlow in
France for the weekend celebration and return.  On arrival I was greeted with
horror of horrors.  The quay was lined with groups of power boat owners and
their ladies.  The new boat owner who had given his full feelings to the
shipyard manager and looked somewhat flushed greeted me with, ``they can't
expect you to mend it in 10 minutes.  I unfortunately found the problem was a
seized clutch and explained that the major failure would result in the gearbox
removal and returning to England for repair.  The men and I were greeted with
the words what can be described as extreme disappointment.  The boats departed
from St Hellier on passage to St Marlo and me left with the boat to remove the
gearbox.  This bit is outside the box --- whilst removing the gearbox and working
with no shirt I became sunburnt and had a bottle of Calomine lotion to help the
soreness.

The gear box was loaded on a fairly hefty sailing schooner on delivery to
Hamble.  Fog had shut down Jersey Airport so needing to be at Southampton I
went to the docks and signed on as supernumery with a coaster named Loone
Fisher controlled by a Captain Edwards.  On arrival at Southampton I started
walking with my heavy holdall of tools making my way to the main gates when I
was picked up by the Customs Officer and taken off  the for a strip search.  I
remember watching with pleasure as each officer tasted the bottle contents,
shaking their heads and passing it on to the next.  Eventually I was cleared
and the gear box rebuilt and I returned to reinstallation.

The Construction of the M5:  When the various contractors were there to build
the M5 motorway  they purchased millions of pounds worth of Caterpillar extra
heavy earthmoving equipment.  The deals required that a Caterpillar engineer be
on site to carry out warranty repairs.  As Bowmaker  CAT was our employers this
task entailed Field Service engineers, three of us on site at all times, namely
Tony Butt, Johnny Blackmore and myself.  We worked in weekly shifts, 2 working
the normal South West Depot.

The first station was located at Banwell for the contractor A E Farr.  I toured
the site as it pushed in to the West to carry out repairs as required.  We had
radio control in those days so would radio in for parts required if necessary.
Iwould leave Banwell early morning and drive and collect from our Melksham
depot in Wiltshire.  There was a spread of heavy equipment CAT D9 and D8
pushers and dozers 631 scrapers 16 grader CAT977 and 955 CAT D6 and of course
many other types of equipment.

Banwell was hard rock towards Bristol and caused severe sand blasting in
machine radiators travelling down to the Webington Country Club to bog terrain
and men developed stomach problems from insect bites down to Edithmead.  The
base was filled with fly ash from the steel mills carried by rail in from
Cardiff.    This condition continued.  As the machine came out of warranty I
was relieved for other duties such as attend to Vosper Thornecroft, Portsmouth
and carry out engine inspections.

Another very interesting thing was to attend Fairy Marina at Hamble to carry out
sea and speed trials to a new 45' Fairy Swordsman using the then new 6 cylinder
overhead cam force valve for piston engines.  This Company built the Fairy
Delta which went through to 1000 miles per hour with the test pilot named Peter
Twist.  Peter had been retained when the aircraft works closed and remained as
test skipper for the power boat division and he skippered the boat on the out
on the Solent so that I could proceed with the engine power test.  The very
same boat kicked back at me.  The following Friday at 5 o'clock I had just put
my keys in the ignition of the van to go home for a long overdue weekend with
Vera when the field service controller came out of the office waving and making
windmill impressions and said, ``Big problems! The Fairy boat was delivered
that morning and was demonstrating its ability to an invited crowd in St
Helllier harbour when one turning clutch had jammed in gear so I need you on
site tomorrow at 9 o'clock and by the way, Exeter airport is not running so its
back to Southampton for 7 o'clock --- a bit of an early start boy'' poking the
back of my head through the van window.

About the area of Thorn Moor.  How lucky was I to have been born and reared
through to adulthood at Thorn Moor superbly located on the perimeter of
Dartmoor, high enough to see Exeter blackout glow and look in to Dartmoors
Causdon Beacon and snow line.  Followed by the charm of a deep valley and
surrounding my home where there were wonderful reeded marshlands and ponds.
This wonderful expanse of terrain enabled me to experience first hand a part of
the war action experienced in Normandy and the east coast seemed to live among
the wildlife we can only dream of today. I will briefly explain the Exeter
bombing which seemed to me to be a set pattern as follows:  an air raid siren
would sound around the countryside.  Mother would take me from my bed and she
and Dad and I went into the top bedroom and Mother would have extinguished
anything that could be lit, not even a birthday candle survived.  We set our
eyes to the eastern horizon and waited until the dark bombers arrived showering
down fully lit colourful flares on Exeter giving a marker to the bombers that
followed.  Even from 12 or 14 miles away the bombs and bombing aircraft engines
were intense to my ears and in a short time a group would be flying directly
over our house.  With Spitfires chasing them over their flight path course over
Dartmoor.  Later in life I discovered the reason the flight path was following
the same course --- Germany having occupied Normandy the bombers were stationed
near Caen and following a flight path from Cherbourg across the Channel to
Start Point and then turned and followed the coast line to Exmouth and then
turned inland and followed the River Exe to Exeter to drop bombs --- mission
completed.  They would set a compass course taking them over the area to
Dartmoor and then link up with the River Dart to Dartmouth then reset for the
channel crossing to Cherbourg and home.

We did have some trauma after such a raid.  We never gave any thought to the
machine gun bullets but one morning Dad went into the vegetable garden and
found that a bullet from a plane had deeply grooved the handle of the garden
fork.  The next time he was outside the Post Office at Cheriton Bishop, around
the corner from the school, he met Mrs Rice, a lovely serene lady who normally
wore a black dress with a spotless white collar and apron and finishing the
normal greeting referred to the aerial battle 2 nights before and Dad relaying
the bullet groove story to her, Mrs Rice responded by throwing her hands in the
air and exclaimed ``Oh my!  I have never known such bravery like it.  Come in
and have a cup of tea'' --- a hero for the day.

With regard to the wildlife that existed in my area, I will do an imaginary walk
recalling life as existed in those days.  Coming out of the back door and
walking slowly around to the front garden I walked over to a privet hedge of
medium height that was shielding a steep slope to the vegetable garden
(remember the bullet in the fork handle) and an orchard of cider apples with a
family tree George Ponsford cider, Tom Pud and a Queenie red eating apple I
would part the bushes of the privet and without fail would find a bullfinch
sitting on her nest.  What a joyous sight!

Continuing my walk down the path to the road gate I was aware of the steady hum
from my Grandfather's top apiary of bees consisting of one WBC and eight or
nine homemade straw skips.  As it was nearly the end of July and the near cut
off of the honey flow my mind lingered on the activity that will take place
here in early august when honey will be extracted in the old fashion way.

I turned and walked down the hill passing Little Thorn Farm over the small
bridge and to the next field on the left.  I open the gate --- please note that I
opened it and did not climb over it --- as it was someone else's property.  I
step in a few paces and look at the mixture of colours caused by the rough clay
ground barren in areas with multitudes of stones the size of large potatoes
covering the surface, a haven for ground birds to lay their clutch of eggs
keeping them well concealed.  So we had an abundance of curlew, larks, lapwing
throughout the field giving a healthy replacement to nature (not possible
today).

Today we have too many self-appointed naturalist officials in high authority
laying laws of preservation.  The result is that songbirds and ground birds are
diminishing at an alarming rate.

I will tell you as it was in the early days.  A sparrow hawk was looked on as no
better than a rat.  It would catch small hen birds and kill the hen birds so
the chicks would starve and die causing the end of that species for that year.
So the sparrow hawk was shot on sight thus reducing the numbers of these
wasteful predators.



The badger, one of the most destructive predators to wild life will eat all
ground birds' eggs and chicks and kill with a cruelty and savage any hedgehogs.
No wonder the nation is requested to report on the hedgehog population in the
garden, if any.  I am afraid the sightings will diminish in time.  In 2018 we
were recording by a night camera the movements of a family of 4 hedgehogs. Then
one day my neighbour told me with great joy that a badger had moved into a fox
home nearby. I told him that would be the end of the 5 hedgehogs and sure
enough no more hedgehogs have been seen by the overnight camera.

In past times badgers were considered a predator and encouraged to move on as
were the magpies which do so much destruction to nesting birds.  But in this
instance I walk out of this field of life feeling fully confident that he
chicks reared will add to the remaining wild life in this area.

I return and retrace my footsteps continuing past my front road gate and
continue up the road.  I am delighted that many yellow hammers are flying out
of the hedge about 2' high or nest height in their continuing circle of chick
feeding.  I would repeat an abundance in those days.  Regrettably not one
yellow hammer is seen flying today.

I climb over the gate into my field on my return journey home knowing full well
that in August some pheasants will be nesting in the well-established grass.
Returning to the field at the rear of the house nature has fashioned it to
slope down to an acre of wet land of approximately one acre consisting of a
healthy stream, resulting in the growth of water rushes.  The area was amass of
reeds and rich green grass that looked deserted but should you walk slowly
across you would suddenly spring to life as a snipe would take flight with a
squeal and combine a characteristic bowel movement from under your feet giving
you a severe start before the heart returned to the normal rhythm.  A pair of
woodcock and a jay would hurriedly leave the safety of a good hideaway and fly
into the next field of brushwood.  This was proof that the area was full and in
abundance of small mammals that made up a plentiful food chain.  Two owls had
taken up residence in the very old hollow tree and raised 3 broods.

Eight magpies and sparrow hawks were no threat to small songbirds as they were
well controlled.  Even wood pigeons thrived to be shot in early winter for
pigeon breast pie.  Remember food was rationed.

Sewn up by a Goat Farmer:  I was about 10 years old when one Sunday I was
walking around Thorn Moor with my gun looking for a lone pigeon.  I jumped off
a bank while paying no particular attention to the landing area.  This resulted
in me landing on a broken bottle cutting my ankle into the bone.  Luckily I was
near the house so was able to hobble in doors.  One look from my father who
asked me how I did it happen and who shall we find on a Sunday afternoon to
stitch it as our Dr Jackson lived in Crediton.  Suddenly inspiration came to my
father who decided to ring Dr Bonnelly, a new man living at Hittisleigh Mill
who was an ex-army doctor starting up as a goat farmer.  I will ring him. He
arrived in a land Rover and said to my mother ``Clear the big dinner table and
put a pillow at one end and let me have 2 pudding basins''.  He then poured
methalated spirit in and steralised his equipment by lighting a flame.  He then
said that he had some horse hair but no anaesthetic.  He asked my mother to put
me on the table so that he could stitch me up.  He told my Dad to hold me down.
I can tell you I squealed throughout the procedure.  Dr Bonnelly was thanked
while I fought the nerve end pains that night.  The doctor came back and
removed 3 stitches.  From that day I had a special man and boy relationship
with him as I saw him as a robust Army doctor.  It was not a long time from
that we heard he had purchased Venbridge House at Cheriton Bishop and started
up as the Cheriton Bishop doctor.  I bet it paid better than goat farming.  I
would think I must have been one of the first patients before the formation of
his surgery to stimulate the forming of a very excellent practice.
