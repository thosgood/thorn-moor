%!TEX root = ../thorn-moor.tex

Having covered the location and the character of living at Thorn Moor I must
explain some of the work encountered to keep life turning over. I will call it
meat management. It starts with my father bringing home a small piglet and
placing it in the pig house as it buries itself in the new straw. From then it
is well fed with pig meal supplement, potato skins and all manner of items
grown in the garden. After a period of time we have standing before us not a
little piglet but a very fine animal weighing in at 15 to 16 score. This
represents a matured animal with well matured body fat. From that day things
began to happen. An item that looked like a very sturdy stretcher with
standing legs about 2 feet high was recovered from one of the sheds, cobwebs
and all. It was scrubbed with boiling water with the everyday scrubbing brush.
The back house complete with the water trough was also scrubbed down. A small
mounting of small logs were positioned beside the copper boiler and the
container filled with water ready for boiling. Four hazel nut sticks about
half an inch in diameter were cut about 1 foot long, skinned of bark and placed
in the stone water trough for use on the day of action. I was soon to find out
their use. An ash stick was shaped like a yoke used by milkmen with 2 very
sharp pointed ends and were positioned in the back shed where all was to
happen. To a small boy of approximately 8 years the impending action could be
felt. The day came and word must have been given to Mr Partridge of West Pitton
Farm who came with a canvas bag containing a full kit of butcher's knives
hanging from his bicycle handlebars. Mr Partridge looked over the equipment at
our outhouse and nodded approval. Ten gallons of water was now boiling. I was
told to go indoors and help mother and not to come out until collected. The
carcass of our pig was lying on the wooden stretcher having been washed down
with the cold water hose. The carcass was lifted further into the backhouse
and plied with boiling water at intervals whilst the men were vigorously
scraping off the very course hog hair which continued until the whole body was
white. A slit was cut in each rear leg and the newly made milkmaid yoke was
attached by pushing the points through the leg slits. A pulley was attached
and the whole body winched, back legs uppermost, to the ceiling crooks. From
that position Mr Partridge demonstrated his butchering skills. He removed the
stomach and entrails, put them in a galvanised bath and placed the whole
contents in the stone trough of cold running water. It was at this point that
I gained knowledge of the use of the 4 prepared hazel sticks as mother prepared
to clean the intestine to become a remarkably clean item with all content
washed away. By evening the body had been completely jointed except for one
quarter which was collected by Roy Partridge who had walked from West Pitton, a
distance of 6 miles, sat down and talked of the daily problems, had a taste of
the chitterlings and agreed that they were very tasty. Despite me having
watched the cleaning of them I only then discovered their function and as a
small boy could not entertain taking a bite. The time was 10 pm and Roy
wrapped up his quarter of pig in a hessian sack, tied some binder twine at each
corner and positioned it over his back, wished us good night and walked out
into the moonless night to walk the 6 miles home loaded down with pork! They
made them tough in those days. Mother put the remaining joints of meat in a
large wooden vat and mixed up a solution of salt brine and poured in enough
brine to cover the joints to preserve them and added a round wooden lid to keep
the meat free of dust and dirt. I feel the leaves were breaking into colour and
the sap flow to the leaves was rising up so when October arrived the sap of the
trees and smaller bushes were in retreat. Father would action his hedge and
fence maintenance plan in such a way as to steep 120 yards of hedge. This
entailed cutting out untidy growth of brushwood, cutting down the enlarged and
out of condition wood stumps but retaining the useful saplings about 3 inches
in diameter and naturally positioned when laid over would if possible form a
continuous handrail appearance and as hoped when the small branches grew the
length of the sapling turned vertical would then form a continuous hedge
growing vertical.  The art of cutting and bending a sapling was to cut a V
block in line with the desired direction. The branch is required to lay over
and hold down with a wooden crook fastened by an unrequired tree branch. The
hedge earth bank itself will have been found to have been damaged by the
retained sheep, cattle or early rabbits. This would be corrected by rebuilding
the shape by cutting and repositioning the earth that had slowly slipped out
into the field.  The best implement for this job is a Devon shovel
manufactured by Morris of Dunsford or Finches Foundry of Sticklepath near
Okehampton. All excess timber was put into piles according to need. The
heavier trees and limbs up to 4'' in diameter were trimmed back and cut to
about 9 or 10 feet lengths. Cuttings and brushwood were made into faggots and
the heavy tree stumps left separate. So at the end of the hedge laying
operation all the faggots were returned to our house and were built up into a
small square wood rick that was thatched with green pond rushes to keep them
dry. The heavy 9-10 feet poles were firstly formed as a tripod and all
remaining poles filled in to form a wigwam. These poles would dry out over the
months as the wind would blow through the expanded legs at the bottom and the
rain would run off the new vertical surface, In January father would start
selecting wood that had been stacked against the east and south side of the
wigwam as it was considered that the cold east winds would dry the wood free of
sap. Enough was cut into logs for the hearth fire as a faggot was partially
removed from the rick and positioned ready for use as every day kindling. The
large tree stumps were axed and split with metal wedges to become the correct
size to use as a hearth back stick for all day burning so as we see, nothing
was wasted. A period of interest raises its head for the month of May --- hay
harvest time. The fields of grass were grown to waist height so spring breezes
would send waves rolling and cascading across the surface of the tall waving
grasses.  A tractor would arrive with a grass mower and cut the field leaving
wonderful patterns of cut grass. As always we hoped the weather remained dry
with sunshine to keep the air warm to convert the green and very moist grass
into dry quality hay. This was escalated by using a hay turning machine to
lift the grass into the light fluffy rolls allowing the warm air to breeze
through speeding up the drying process. Checks would be made until the hay was
considered correctly dried.  Word went out and the team of neighbouring
farmers and farmhands would gather bringing the desired equipment to assist
with the labour saving of the harvesting process. First on the scene is
normally a hay pole delivered on a trailer with 2 wood poles approximately 6''
in diameter connected together with a steel sleeve. It is raised to stand
vertical and held in that position using 4 excessively long heavy ropes
attached to 4 heavy steel securing pegs. Hanging from the top of the pole is a
yardarm with a pulley at each end. A rope runs through this unit.  At one end
is a hinged 4 pronged hay lifting fork whilst the power rope is threaded from
one wheel through the yardarm down the hay pole tree to a pulley block at the
bottom ready for the lifting power to be connected, normally to a horse. The
tractor or hay sweep is coupled up so the team would consist of 2 men to build
the hayrick, 2 men to load the hay pole grabs and one to lead the horse. There
would also be one man or boy to drive the hay sweep. Any extra help was not
turned away and most important part way through the day mother would arrive
with the most important full harvest basket and kettles of tea. I remember one
season I would have been about 14 years old and the day had just started when
one man building the hay rick said to me, ``Come here and help me build the
rick. If you do as I say you will be able to build a solid waterproof hay
rick''. I had no idea what he was talking about until the hay was constantly
pitched to us from the ground. Picking up and repositioned in the desired spot
with the constant command from my teacher ``keep the centre full''.
Particularly he kept his eye on each corner and expressed strongly this command
that the corners should slope out. He told me that a well-made rick should
look like the well risen surface of a well baked bread loaf.  The hay sweep is
a large long pronged rake bolted to the front axle of a tractor. It is then
driven out to the field and carefully run up the hay rows one at a time until
full. This was considered when the top of the load was level with tractor
radiator top. Harvest duty completed and Uncle made a visit on the week or so
after and thatched the rick to face the wind and weather. Life progressed and
extra work was taken on board as extra fields had been added together with
extra milking cows to boost the smallholding income  Father at this time had
changed from working at Yeoford and was involved with the Forestry Commission
attending to forests in Devon --- Fernworthy and Bodmin Moor. One day in October
life at Thorn Moor made a radical change when Dad was diagnosed with terminal
cancer. We sold most of the milking cows retaining one for our home use from
that then until Dad passed away and Thorn Moor was sold. It was out of bed
earlier to hand milk the cow and clean up the yard, then journey to Whiddon
Down for my daily workload of agricultural engineering. Evenings were taken up
with carrying out work required on the seasons requirements. My life and mind
made a big change when I met the love of my life, Vera. Fortunately the
meeting was 7 months before Dad's passing. So we both could sit by our fire
and talk at length to each other. I responded to an advertisement requesting a
service engineer. I was offered and accepted the job and it was agreed for me
to join them when the time was convenient. In due time the farm was sold, I
got married and after the return from my honeymoon I started working in the
engineering rebuild section of Bowmaker Plant Ltd the Caterpillar dealer, a
great advancement in the engineering field for me as the onsite lecture room
together with Company roving lecturers was able to keep all service engineers
in the loop with all aspects of this product.
