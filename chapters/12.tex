%!TEX root = ../thorn-moor.tex

Farmer Palmer:  Dear old Farmer Palmer, was seen through the eyes of a young boy
of my age, short,  portly and so innocent that the reality of life must have
mercifully passed him by.  He passed on his various encounters to second and
third parties which were received with goodwill and mirth.

Having seen the special tractor that Worthy Anstee had built for himself, Farmer
Palmer requested that Worthy build a similar unit for him which was duly done
and the day came for a test and customer approval which was received.  The
tractor was duly delivered and with Worthy in the driving seat Farmer Palmer
was pleased sitting beside him.  The tractor was driven at top speed to show
off its capabilities.  The speed frightened Farmer Palmer who not wishing to
offend exclaimed ``slow down. Tis fast enough for cutting corn.''

In the early days of the bombing of Exeter Farmer Palmer stopped his car beside
our main gate.  After the normal greeting to my father he said, ``Quite a heavy
night of bombs''.  ``I said to my wife hail up.  You yerd the Germans are
coming''.  So we lifted our main sheets over heads and settled down
underneath.

The local garage had an urgent call from Farmer Palmer saying ``Will you come
quick as our old car          is alive.   Every time me son-in-law, Jack, turns
the starting handle to drive the car it keeps walking towards him and has
pinned him up against the wall so would you come and rescue him''.

When I retrace my memory back to my age of 10 years I become aware of how much
we as individuals manufactured for ourselves.  It started by me sitting on a
log in the orchard watching Dad make me a whistle from a willow branch using
his penknife.  A few days later found me sitting on the same log making a
whistle with my penknife – every country boy would own a penknife which was
probably given as a Christmas present or found in a box of trinkets.

As time went on I wanted a bow and arrow so I made myself a good one in my
opinion.  Then a catapult winter arrived with snow. Mother's big dinner tray
was ditched as I made a good toboggan copied from a winter Christmas card.

I did say the dog fights in our area were the first indication of war to us but
as time developed the civilian nation became drawn in to add to my interest as
a small boy because my Dad joined the Cheriton Bishop Home Guard platoon.  I
remember Dad returning home with an arm band with black letters L-D-V printed
on it.  When I asked the meaning of the letters Dad told me with laughter in
his voice that it meant look, duck and vanish.  As time went on equipment
arrived such as a uniform, a rifle and this continued expanding by each visit
by a Sergeant Lock.  He brought Sten guns and magazines with bullets that
needed to be installed in the empty magazine cases.

When fully operational my father would go off on Sunday morning exercises and by
stories relayed at our dining table, with mirth, we could follow the Dad's Army
television tales.  An example is that a Sunday morning would finish at 1 pm
when father returned to have a late lunch and normally sleep in the chair in
preparation for Monday.  A car would draw up about 6 pm and a very flustered
and red faced Sergeant Lock would ask if we had seen the man who was put on
guard duty at the waterworks.  Father would reply that he had not.  The guard
had been on duty all day because he had misheard at the briefing.

One farmer in particular would let the side down.  For example, under manoeuvers
a small number of the team were silently creeping up beside a hedge with the
intention of a surprise attack.  One very enthusiastic farmer would break from
the cover of the bushes and rush into the field and

(Mike's memories Part 13 Book 4 (cont.a)

exclaim loudly, ``Look at this wonderful field of growing early potatoes'' thus
wrecking any chance of a surprise attack. I recall going to an evening dance in
a marquee at the village of Crockernwell when I was 12 or 14 years old.  I was
accompanied by my parents who sat and chatted all evening with a Mr and Mrs
Herbert Gillard and as they had their 15 year old son, George, and their very
young daughter, Sheila, with them it was natural that George, and I should pal
up for the evening as we did and built up a friendship that has lasted a
lifetime even though in the past years we rarely visited each other except to
attend each other's 50th wedding anniversary parties playing our original roles
of Best Men.  Reverting to our relationship, when we were young bucks cementing
our friendship by swearing allegiance to the following we agreed not to steal
each other's girlfriends.  A pledge kept but sometimes regretted it was even
made depending on the quality of the other's latest girlfriend.  We also
pledged to be each other's Best Men if we ever became married.   Both pledges
were actioned 50 years later by both of us. The first deal that transpired
between us involved me purchasing an orange all spare parts four speed Sturmey
Archer bicycle for £4.  This machine was ridden everywhere as it was far better
than my discarded single speed black Halfords bicycle. We then started
analysing our success with girls.  In our opinion the lack of memory, bearing
in mind we were only 16 or 17 and green as grass George decided it was the
wearing of spectacles that held the secret and promptly ordered a set of
contact lenses.  He duly came home with them having waited for some weeks for
delivery.  My impression of them was nothing short of horror.  They were the
size of the eyeball made of glass that had to be filled with a small amount of
fluid which was supplied to keep the minute gap between the eyeball and glass
dirt free. These lenses remained a secret between George and I.  George's story
to all was that his eyes had improved! However on one particular hot night at a
dance at Chagford events caused my friend and myself panic and disruption.  It
was about 10.45 pm when the dance floor was packed and body heat of all was at
a maximum that I became aware of a group of people looking worried advancing
towards me.  This was the start of a minor panic as voices said in unison
``George looks very ill as his yes had glazed over and dancing close to George
I immediately realised what was happening.  The heat of the dance hall had
caused the film of fluid to break from between the eye and the contact lenses
and condensation had covered the inner glass surface.  Both eyes were
completely grey and looking like the televsion production of the Old Wise Man
who commenced with                 grass hopper before giving advice to the
Kung Fu hero.  I was aware that our mutual secret could be exposed so I said
loudly ``Leave George to me.  I will sort him out in the cloakroom''.  This was
greeted with relief from all as dancing was the joy for all and they returned
to the dance floor. The contact lenses were cleaned and we returned to continue
dancing. Sunday afternoon was always the time to analyse the problems
experienced on the dance nights, particularly with regard to relationships with
girls, or lack of them – a unanimous decision was then made to ditch the use of
the contact lenses, put them away in a drawer and revert to spectacle use in
George's case and act normally.

The birth of the Exeter Pistol Club:  One spring evening in 1965 a very pleasant
man walked into my garden and said: ``Evening Mike.  I would like you to help
with 2 or 3 others to form a pistol club.  What do you think?''  I told him
that I was not sure about revolvers and fast draw seems to be on film sets but
when he produced a collection of semi-automatic pistols for precision
marksmanship  my attitude changed completely.  So Wednesday evening found me
joining a small group at Jim's

(Mike's memories Part 13 cont(b)

house to discuss the future of the group that formed the Exeter Pistol Club. Ken
Chard, the owner of the gun shop located at Exe Bridge, Exeter, Jim Austin
whose home we used for Committee Meetings, Ernie Hart who had knowledge of
competitive shooting who was the controlling body of the NSRA and yours truly
who had nothing to offer but listen to all that was said agreed to shoot at
Wyvern Barracks on Topsham Road twice a week and join the NSRA postal shooting
competition.

I was given a target so I manufactured wooden target holders.  We each put money
in the kitty and purchased 2 Russian Volslocks as Club pistols.  I sold my
father's 12 bore in exchange for a Smith and Wesson 41.  All administrative and
other items were in place ready for the first evening with the Pistol Shooting
Club at Wyvern Barracks.  So with targets and new holders 25 yards down the
range, tables across the range and a Range Officer wearing an arm band, 4 of us
lined up.

Having listened to the instructions of Ken Chard and Erny Hart and having not
taken arm muscle exercise I commenced to fire off my clip of 10 bullets.  It
would have been a praise we called awful.  Not one shot hit the paper target
but the bits of wood flying off the target holders gave a good indication of
our lack of ability.  At the end of one hour of shooting the holders were just
hanging together and looked like they had been dipped in a tank of sharks.

We all slowly improved and with arm exercises, together with the more we
practised, Jim decided that deepened experience was needed so booked us into
Bisley Pistol Championships.  On arrival Jim attended the Office Administrator
and in collecting our target numbers and shooting times he discovered Devon had
not entered a team in the County Championships.  When he returned to our camp
hut, our lodgings for the night, he (Jim) said, ``My men.  I have entered 4 of
us in the Devon Pistol team.''  I said, ``What! You must be off your head.  We
are only learning to be marksmen.  We will never level with the worse.''  The
return comment was that we would be OK and that it is the experience that
counts.

With a good nights sleep under our belts we set forth into the marked areas at
Bisley carrying out our own shooting in various ranges.  Our individual
results: my Club average was not disappointing I was told, particularly under
competition pressure.  Jim went to the official marking office for the results
of the inter County Championships.  We, as expected, did not do well.  As Jim
returned not sure of our position he told us that it was very difficult to find
the results.

Years passed by and we all improved and were shooting in Division 2.  I entered
a postal elimination shoot at the Eley Olympic 1980 competition and ended the
postal round in Bisley 60 finalists.  In August Vera and I went to Bisley and
to my surprise I finished third.  The cut glass bowl stands in pride of place
today.

As I was beginning to help charity organisations I decided to move on from
pistol shooting remembering the great fun and good time over the years, I
purchased and gave the Club a cup to be presented each year to the most
improved shot in the last 12 months but I never heard anything back so they
must run without a Chairman or Secretary.  I believe they are shooting
somewhere in Pinhoe and I hope they have same enjoyment and sense of
achievement that we had all those years ago.
