%!TEX root = ../thorn-moor.tex

\here
As time and seasons drifted by I would travel to the area of Forder river with
my home-made fishing rod copied from a fishing journal. I had made copies of
the fish hooks found in a picture book. Sometimes with my trusty Webley air
rifle and sometimes with lifetime birthday or Christmas presents. How strange
no mobile phone or watch but always home in time for lunch, tea or bed times.
My trips to the river fishing taught me the dark side of human nature. Whilst
fishing over the summer school holiday I met another local lad, may be two
years older than I. We became friends and spent fishing morning or afternoon
together and with great delight for me I discovered that I should be catching
the school bus and attending the same school, Chagford Secondary Modern.
Eventually the school start day arrived and I cycled to Rydons Cross to catch
the bus complete with new found friend. At 8 o'clock the bus ground to a halt.
What happened next was not anything I had ever experienced before or expected.
The school bus door flew open and my new found friend started to shout to the
rest of the kids in the bus ``look, bats ears, I told you he would get on here.
They look bigger today than last. Welcome Batsears.'' This onslaught continued
until we reached the school and for the rest of the week. I could not come to
terms with the actions of the lad in those days particularly when at the end of
the week as I alighted from the bus my tormentor's voice quietly and
expectantly said ``See you down the river tomorrow''? I think I said, ``I do
not think so as I am a bit busy''. This unhappy event faded with time. Lessons
and events at Chagford School were ideal in my opinion for the nature of the
region had an agricultural background. They ran a scheme that after a pupil had
completed a period of time he or she could elect to join an agricultural based
class which was named Ag1 and Ag 2. This represented the last two years of
school. Regular visits were made to our classes by a Mr Bill Hingston from the
Ministry of Agriculture giving us first class instruction on thatching and in
particular stone walling which was put to good use as we commenced to construct
an open air theatre for the drama groups to use. With the school located in
the harsh environment of Dartmoor you can imagine the sparton conditions
rendered on the parade ground with scanty kit of white shorts and short sleeved
white shirts while a thick white frost covered all the grass and hedges
followed by a stern warning to us to run on only dry patches of tarmac.
Summertime swimming lessons were equally shocking. This event started with the
march from Chagford School to the swimming pool at Rushford Mill to be greeted
with a plain cement pool that was directly fed by the very fresh water from the
River Teign. We rushed from the changing rooms to the pool and jumping in
always resulted in the body feeling like it had been crushed by ice. We did
enjoy it but it took time to respond to the movement. The school also kept six
WBC beehives and a particular event always brings a smile to my memory. The
opening construction consisted of the six beehives positioned behind the school
block on a corner that looked through the headmaster's study window. It was
deemed one sunny afternoon that we, Class Ag1 would line up outside the
headmaster's window and watch our class teacher, Bill Warren (k/a Willie) and
an assistant carry out what is known in the beekeeping world as a spring clean.
Something my Grandfather, John Preston had carried out at home this weekend
before which entails smoking, lifting the cover and scraping excess bees wax
and replacing items as needed using a steady hand so as not to disturbed the
bee colony unduly. Unfortunately the teacher and helper would probably not
have done this job before and would not have had the skill John Preston
possessed. This had resulted in agitating the colony so more bees than normal
were flying out of the hive. The children lining the headmaster's wall began to
show concern and movement at which stage the Headmaster's voice boomed out with
anger ``Stop moving about and watch the demonstration'' and to back up his
frustration walked through the door into the apiary. He had two things against
him. He was a heavy smoker and very tall. Bees do not like stale tobacco
breath and a great target standing at the height of 6' 2'' our Headmaster was
walking around the apiary fast and agitated with much arm movement. At this
stage of play I thought things were going to turn nasty and sure enough a high
pitched buzzing immediately commenced for a number of guard bees, indicating
confrontation, started to escape and attack the headmaster who was leaving the
area and running bent doubled up with his head about 2' off the ground both
hands vigorously beating his bald head in a vain attempt to repel the four bees
which were engaged on a very successful stinging attack. I could never believe
how a body could run so fast bent double with head approximately 2' off the
ground through crops of potatoes, onions, carrots and leeks without falling
over. He disappeared from view. To keep face and dignity a rumour circulated
the next day that he spent the rest of the afternoon recovering from that event
in bed! After that incident the bees received a new status from all those who
had been caned throughout the term and there was applause and high praise but
from the girls a whisper of deep sympathy for the Headmaster. I must bring
attention to Bill (Willie) Warren our class teacher who had been tasked with
cleaning the bees. In my opinion Willie brought a breath of fresh air into
everyday school activities. Having been demobbed from the war, an ex-RAF
bomber pilot flying sterling bomber planes on night time sorties over Germany.
He brought into the class a more down to earth and open attitude which we all
need in everyday life. We had many occasions when he would stop a science or
English class to give us a short talk about his experience as a bomber pilot.
Flying for hours in a freezing plane then flying over a city wondering who was
looking up, perhaps the enemy, normal people, family units and thinking will I
survive tonight and will I be walking about at home tomorrow etc . In my
opinion these short stops never hindered a sluggish class but rejuvenated the
minds. It comes to my mind that that event took place just before our breakup
up for the summer school holidays. Class 1Ag was involved in an activities
afternoon cleaning the open air theatre borders to be dug over and the grass on
the pathways cut with a push drum mower. Having commenced my duty to dig over
the boarders it became quite warm so I removed my coat and worked in shirt
sleeves. All things were going well until Willie Warren arrived on the scene
complete with a hand push roller drum lawn mower requesting that we stop work
and move away from the lawn edges as he was going to cut the lawn path. Willie
commenced one end with purposefully long strides whilst I remained standing
admiring the even green grass cuttings cascading into the grass box. Suddenly
my admiration turned to horror as the green turned to brown and woolly. I ran
and retrieved my coat and horror of horrors it looked like a hedgehog that had
been shorn with sheep shears with lumps of thread sticking out all over. This
coat had been purchased by Mother two weeks before from Thomas Moore, High
Street, Exeter so homecoming on that day was not going to be very easy. Willie
Warren denied responsibility with the words ``typical Mike Saffin throws his
clothes down all over the place and never hangs anything up''. I responded,
``If this is the best you can see pushing a lawnmower how did your bombs hit
anything when you were flying over Germany?'' The homecoming to Thorn Moor was
a more serious event --- me wearing the coat as if there was nothing wrong with
it. Kids cheering and shouting from the school bus windows, me walking into the
kitchen keeping my back to the wall until Mother discovered the damage and
expressed horror at the sight of the damaged coat. I, for my contribution
expressed immense surprise at the extent at the time as it appeared minor to
me. Mother being resourceful said, ''What's done is done''. The coat can be
used winter time to keep out the cold with a light raincoat over to cover it
up. The season slid away to the best part of the school year --- the August
six weeks holiday period and for me the beginning of my last year at Chagford
school and for me a chance event that changed my outlook and direction in life.
It started one afternoon when I heard a combination of mechanical noises coming
from an adjoining field to ours, a mixture of a corn binder and a type of
tractor which I had never heard before which was certainly not the sluggish
noise of a standard Fordson. Climbing through the hedge when carrying out
further investigation I was greeted with a combine corn binder under tow with a
happy round faced elderly man wearing a grey pullover sitting and working the
machine controls. Towing the outfit was the most interesting tractor I had
ever set eyes on. In fact a totally homemade unit constructed by the 30
something years handsome unshaven man wearing a black pork pie trilby type hat.
As the unit got nearer to me he stopped the tractor and offered me a place to
sit up beside him, not just any old iron seat as was the norm but a big double
American leather covered seat. Iremained sitting in heaven until the field was
cut. I was so delighted that he offered me a lift back to his farmstead, Hole
Farm, on my list of farms easy for me to walk home using a direct cross country
route. On arrival at Hole Farm we stopped outside a very large black
corrugated workshop with a cob wall. On the Dartmoor weather side of the
workshop a small door was open and we walked in. The sight that met my eyes was
unbelievable such as a woodsaw, grinding welding equipment, spanners, band saw,
post drills, oil drums, grease guns, part built farm trailers in progress,
mechanical gear boxes, engines and various items that I could identify after a
number of visits. Many items were American and had been purchased from the
Ministry of Defence and converted to run of a Lister engine type or from a
large engine using some pulleys and belts that enabled the engine to power more
than one item. I must explain that all described was magic to me as the only
things mechanical to me were wood saws, thatchers hooks and a metal tin with a
hinged lid the remaining items were reed willow sticks and wood cut from the
copse. A relationship was set up. Mum and Dad would see me off on my bike to
visit Hole Farm and spend time in the workshop watching or helping with small
tasks such as cleaning out ex-army boxes and tidying items unknown. One very
pleasant job was to work or assist Worthy to reclaim overgrown fields of birch
saplings. This operation entailed using a Caterpillar R2 TVO burning crawler
with a long wire rope attached to the draw bar. I would reverse the Crawler in
to the birch and Worthy would hook up to the small tree and pull it out of the
ground clear. We reclaimed many acres in this manner.
