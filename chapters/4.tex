%!TEX root = ../thorn-moor.tex

One day an army tank was delivered and Worthy put it to good use: climbing up
the apple trees until they toppled over with the weight enabling them to be
removed with the roots attached.  The area of land would be cleaned leaving
deep pits that had held the removed root system. A big single furrow plough
called a Paraine buster was used to plough the whole field thus levelling in
the earthy surface for crop plantation.  The passing seasons saw many trailers
made with wonderfully strong frames with jacking systems attached and sold to
farmers far and wide. The month of March 1950 took a change when a motorcyclist
rode into the yard, a well travelled man, with a line of small flags
advertising the country visited decked with very well made pannier        boxes
and motorcycle spare tyres around his shoulders.  This was John (Jan), Worthy's
older brother who had left Celon on Boxing Day and ridden his Triumph 500cc
motorcycle to England to be demobbed at Plymouth naval base from his duties as
a Chief Petty Officer – another story told in a motorcycle magazine in the
1980s.  The ambitious man then commenced to build a lorry square framed on a
trailer frame.  When asked the plan I was told ``I am building a caravan home
for my family and going to position it in the little orchard behind this
workshop and in due time it was completed and positioned in the location
desired.  His wife Elsie and very young son David took up residence. Life
continued with me keeping a watchful eye on the wonders of John Preston Butts'
thatching and bee skip making skills watching with wonder at the engineering
activities.  Life took a different pace when one day a giant low loader arrived
and proceeded to unload a used yellow Caterpillar D7 bulldozer.  Worthy this is
now nine months that followed and saw activity in Hole Farm workshop as the new
regular business of trailer making was replaced with bulldozing hedges for
farmers. I remained at home and as a growing lad of 12 or 13 was tough enough
to help with duties at home, such as walking up to Big Pennypark and with much
pride used the large heavy hay knife to endeavour to cut out the hay square for
the cows to feed, hoe turnips, cut hedges and clean out the cow sheds when
needed.  Sadly John Preston Butt passed away in his sleep in April at the age
of 90.  From then on I never used John Preston's tin shed as I felt the shed
and its contents were for a past world of natural skills when everything could
be made from nothing and to introduce any other items was not correct.  So it
remained a shrine and I used my workshop set up in the still room next to the
main house.  I shall never forget one Saturday afternoon when I was helping
farmer Burridge at Rydon Farm leading his trusty carthorse pulling a weed hoe
between rows of new growing potatoes when news arrived that Worthy Anstee had
had a catastrophic accident with the Caterpiller D7 bulldozer.  This fatality
changed the lifestyle of many people for some years to come, just as it had
changed my attitude on the day I had rode in to Hole Farm on the home built
tractor.  Engineering was going to be my technical passion with its Caterpillar
and leading manufacture company for outstanding engineering and reliability
over the countryside. Devon life continued in the usual pattern with the normal
fall of snow over January and February.  I have particularly separated my
experience of the worst winter of 1947 when I was 9 years old.  I awoke one
morning to snow falling heavily and standing 4 inches deep.  My Grandfather
John Preston was still with us and when he came down for his breakfast at 9.30
am said ``I've seen worse than this when the River Exe froze over and they
roasted an oxe on it'' and he promptly moved and sat on the old settle wrapped
around the side of the open fire and continued with breakfast already prepared
by my mother. The snow continued to fall every day with the same response from
John Preston until Friday morning by which time the level of the snow outside
had reached the bottom quarter of the kitchen window and was level with the far
banks across the orchard making all tree trunks appear short.  The snow was
still falling with extreme cold making the flakes smaller.  This continued
until nightfall and with darkness the temperature changed and we realised that
the small snowflakes were indeed misty rain.  We had no idea what was going to
be the result of this condition but we did next morning in the full light of
day the countryside  was beautiful and looked like a tinsel town with all the
trees and bushes encased in ice.  The falling misty rain continued all day.
Consequently the encased ice bushes became heavy and nature gave way to the
weight and at intervals big crashes could be heard in the woods and hedges as
branches broke off under the overbearing weight of the ice encasement.  The
crash was closely followed by a continuing sound of tinkling bells as the ice
particles broke into pieces as they cascaded through the remaining ice encased
trees.  When you picked up the beautiful branch to examine it the heat of the
sun soon melted the ice encasement and you would be holding a very wet and just
a normal looking branch.  The temperature raised and spring entered our lives.
Some evenings would find me stalking the hedgerows with father with his trusty
double barrel 12 bore shot gun under my arm looking for a rabbit to change the
menu from beef and pork and later in the season helping father to receive the
open hogsheads  of newly pounded cider juice always discussing the quantity of
the apple crop this year.  Every few weeks would find us racking off the cider.
This entailed draining off the clear clean liquid into a clean barrel and
leaving the cloudy to ferment out further.  We would finish the racking period
with about four hogsheads of clear cider and 1.5 with cloudy cider to use at a
later date. Time drew near for Dad and I to have a discussion about work and
the direction I would take.  Dad made a phone call to Jack Saunders of Whiddon
Down and a date was set for me to attend one evening.  We caught the Okehampton
bus and duly arrived at Whiddon Down.  My first impression was The Post Inn on
the left with a long green shed running quite a distance on the right which was
a blacksmiths shop coupled to a wheelwright shop combined.  Little did I
realise how much time I would spend working the finer arts of ironwork and
canvas drive belt manufacturing in the shed.  We walked down through the yard
with green corrugated workshop standing on each side and met Mr and Mrs John
Saunders who I found to be a very quiet unassuming gentleman.  It was agreed
that I should commence work one week after leaving school (a week to adapt to
the life change).  It is said that help arrives from the most unexpected
places.  My uncle, Les Stevens, who lived at Red Ridges, Cheriton Bishop had
offered to give me a lift on the back of his motorcycle from Hooperton Cross to
Whiddon Down and return in the evenings so I cycled to Hooperton Cross, jumped
on the pillion seat of my Uncle Les' motorcycle and was delivered to outside
the long green blacksmith's shop.  I was met by Mr John Saunders who teamed me
up with another fine gentleman named George Endacott, the main water works man
for the depot using a short wheel base Land Rover as a service vehicle.  My
first job of the day was to attend and replace a set of knotter bills in a corn
binder standing in a field of the parish of Spreyton.  I was so delighted to
contribute to repair as my small schoolboy hands could find entry into the
system and line up the drive gear roll pip with the knotter bill shaft.  These
hands changed over the early years of engineering as finger joints look like
ball bearings and the right hand thumb looks quite flat due t constantly coming
second to the hammer on a miss-hit to the chisel head .  Much and various work
was carried out throughout the region with some standing out in my mind as very
hairy in some instances.  We were called to Wood House in South Tawton, a fine
house with quite a past history.  The fault was to correct a malfunction in a
very deep fresh water well. It was reputed to have a depth of 90' from surface
cover to the bottom of the well.  It was so deep it used a secondary foot valve
system to lift the water to the surface.  To achieve access to the problem
George Endacott bound and lowered 2 full length ladders with rope to serve as
one down the well and slid the sturdy wooden batons through the top rung to
hold whilst a third full length ladder held vertically by 2 men while George
again bound tight to the 2 ladders already down the well. On completion of
binding, the 3 ladders were then lowered further and the sturdy wood bottom was
slid in position to secure 3 full length ladders.  A rope was tied around
George's waist and the loose end tied to a nearby post.  Then unbelievably
George entered the open well.  I watched through the well opening until my hero
looked minute in the gloom.  Great care was taken by me not to allow any stones
to fall as George was only wearing a cloth cap and a stone falling from that
height would have been catastrophic.  All tools were lowered by rope in a metal
bucket.  The faulty bucket system was removed and the item repaired on the back
of the Land Rover.  Some time after I thought the physical endurance of George
Endacott was remarkable as he descended and ascended the 3 full lengths of
ladders and held on to its location at the bottom whilst repairs were executed.
Had I kept a diary its day should have read ``A funny thing happened at work
today''.  The job in question was to attend a newly constructed or sunken well
that had been lined with two cement rings and install a new suction pipe the
source of water which was very good.  So the well depth would be approximately
28 to 30 feet.  The surface of the water was approximately 6 feet from the
surface of the well and having newly dug was still cloudy but clearing.  George
unloaded the trusty  double ladders and set them down into the murky water.
Tools and components were loaded into the roped galvanise bucket ready for me
to haul in when all was ready.  George continued to step down the ladder until
he was in a position 6 inches above the surface of the water.  He then gave the
signal for the bucket of tools.  I turned to pick up some and then heard an
almighty splashing and coughing.  Looking into the water I was greeted with
George's cloth cap floating on the surface together with the top end of the
ladder.  Unknown to George the ladder had stopped 6 inches off the bottom and
rested on the small fragile ledge 6 feet from the bottom of the well and had
given away under the weight of George and the ladder and continued to the well
bottom,   When I looked in the well George's cloth cap was floating around
accompanied by George's head resurfacing with eyeballs that were so fixed that
they looked like they had just emerged from a Soho strip club.  George
recovered and we immediately drove off the site to George's home at Throwleigh
for a complete change of clothes and to dry off.  It was a good job the Land
Rover was constructed by sheet aluminium as the volume of water draining
through the bottom and the side door was continuous on the journey to
Throwleigh. The incident passed and I was aware the subject was never discussed
again and was left to die with the time.  We were instructed to attend the age
old pub of Uncle Tom Cobley at Spreyton and vent the well when a few days of
repair work could be executed.
