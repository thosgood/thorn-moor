%!TEX root = ../thorn-moor.tex

Some 18 months later I was promoted to the position of Master Mechanic (I think
a position that is held in the USA Mechanical Engineering world), and within a
few weeks of supervision at the engine rebuild team we had a classroom in which
we rebuilt the turbochargers and fuel pump injection units. Every engine built
up to 977 and D6 size was power tested at Exeter Depot test bed. Larger engines
such as the D8 and D9 were shipped to Cannock Depot for testing. You can
imagine the nervous anticipation when unwrapping the engine on its return,
looking for oil leaks or any other fault normally marked in very bright red pen
or felt tip --- nothing was spared. (It was called Inter Depot Competition, or
`one upmanship'). At the bottom far location was a very special bay housing for
heavy duty track press, necessary for the rebuilding of tracks and pin-and-bush
turning. Three bays were used for whole machine rebuilds. The welding team
occupied a bay for the use of rebuilding various bucket track frames at the
rear of the workshop, which also held a machine shop and track roller
reconditioning unit. Upstairs was a works canteen that management shared. All
meals were cooked on site by the resident cook named Olive, who was aged 45ish.
The rest of this floor contained the Depot Manager's office, and the General
Office containing the Service Manager and Sales Team. For some years the Depot
operated a rebuild system called `Bondedbuy', where a low loader lorry with
Caterpillar D8 or D9 would arrive from an open-cast coal mine that could be
described as end of life. The unit would be power washed in the yard, pushed in
to the workshop, and completely dismantled, with each unit overhead craned into
their respective bays for rebuilding. The rebuild was of a high standard as
they were marketed with a six-month warranty --- the same value as marketed by
the manufacturer when sold new. Two-week visits to the Caterpillar factory
Cat School certainly enhanced the practical knowledge gained in the last two
years.

One Friday afternoon I was called to the Field Service Manager's office. He said
that, as I had been in the Field Service world of agricultural engineering,
would I consider taking the position of Field Service Engineer, as experience
was invaluable with the proposed construction of the M5 from Bristol down, so
would I give the matter careful consideration with my wife. Careful
consideration being the key element to a Field Service Engineer position with
Caterpillar --- that was unusual. Holiday dates would remain as requested other
than service problems. It was hoped that I would respond any time of the day or
night, Bank holidays and weekends, as the emergencies of the of job required.
We had to take into consideration that at this time Caterpillar had probably
the most experienced mechanics, together with components, labour, and also
living near the motorway. The Salesman sells the first machine. The Service
sells the rest! Monday morning found me starting the changeover from workshop
to Field Service man, an event that did not start off in a manner I would wish:
all tools and equipment having been collected together and transferred to the
field service van. I was dropped off to collect my Ford Transit van from the
paint shop which was situated near Polsloe Bridge in Exeter, unfortunately. I
say unfortunately because I misjudged a right-hand square brick corner and
touched the side door. I did not stop thinking of the old saying ``Do not stop
to look or you will never start and go again''. When I arrived back at the
Depot I lined my van up with the other two. The scene of the people changed
into a circus. The Service Controller continually kept looking into the filing
cabinet. Some people smirked and slid away from the scene. When I walked around
the van and examined the damage it looked like it had been in a Morris dance
with a chainsaw. It was then I heard a cutting voice say ``You will have to
manage for a week or so'' and manage I did, feeling very confident with the
repair required. All components and gaskets are manufactured to a high quality
and give a higher confidence in repairs.

The first taste of the nature of the work to be experienced in this area was
Cullumpton By-Pass. We serviced and repaired a spread of D8 and 977 machines
used in the burrough pit. Some CAT 631 scrapers were added to spread the
landfill. Running repairs were saved until Saturday mid-day, when the site had
afternoons off. We would be joined by two or three men from our workshop, and
repairs would commence from 12 pm, working through the night until ready for
the commencement to work Sunday morning. If extra flood lights were required,
we would burn the gas off the welding set, keeping the nozzle free of carbon
build up. One day we were informed that a delivery of new Russian earth-moving
equipment was due and would outperform. In due time a delivery of bulldozers
quite obviously came out of the factory which had a small quantity of military
design, then had a bulldozer system, several off-road lorries and other
equipment that had started life as military. These units started and revved up
to 6000 rpm and smoked off unburnt fuel and oil until the operator disappeared
from view and ran round the site at speed with questionable stability as the
track idle was not designed for civilian work. The shattering noise of site
machinery gradually diminished and returned to the steady pulse of the original
equipment as it became apparent that the new Russian machine was lying around
waiting for a part replacement. Time marched on together with the construction
process of the by-pass.

One evening I was approached and asked if I would wish to join a group of men to
form the Exeter Pistol Club. I happily agreed advising them that my time would
be very limited due to work commitments. So we formed the Exeter Pistol Group
and met twice a week on Mondays and Thursdays at 7 o'clock at the small bore
range at Wyvern Barracks. I sold my trusty 12 bore double barrelled shot gun
and purchased a Smith Wesson 46 .22 hand gun. Our first visit was very
disappointing and we shot the wooden frame for holding the targets to pieces
until we began to gain ability. We were always under the control of an official
Range Officer and controlled by the rules of the NSRA. All pistols were kept in
their respective boxes until on the range. We started the Club in 1965 and I
continued my membership shooting in postal competitions up to the 2nd and 4th
Division. In 1980 I entered and shot the postal elimination stages of the Eley
Olympic 1980 competition until qualified as one of 60 finalists to shoot at
Bisley. Vera and I travelled to Bisley and, when sitting at the pistol range
waiting to start, I felt well content to have qualified into the finals after a
season of postal elimination rounds. So the shooting that day was like a shoot
at the club evening. In the afternoon whilst looking for the results I could
not find them in the lower half of the list but was more than delighted to find
my name listed in third position. This may not be much to some but I still look
upon the Eley goblet with pride.

Over the period of time I followed up the marine engine problems as a natural
progression of my engine bay rebuilding days. This was enhanced by an engine
power output and turbocharger week course at the Caterpillar factory located in
Glasgow. My first impression as a country boy was riding on the top deck of a
bus and, whilst riding through an area called Galowgate, seeing the doors of
each pub open wide and staff washing the stone floor free from stale spilt beer
out into the gutter --- the acrid smell of stale beer running through the
length of the bus is a constant reminder of another world. A visit I made to an
oil-pumping rig in Dorset presented real history to me. The Caterpillar engine
that was the power unit to this oil-pumping rig was a very old unit --- a
Caterpillar 2000. This was approximately the size of (being the forerunner of)
the D8 power unit with external push rods. The fault was a severely leaking
water pump, constructed with one main through shaft of 5/8'' in diameter using
a bronze calibrator adjusted to tighten a hemp-impregnated graphite rope gland
onto the shaft: as the unit had been running a lifetime with, I suspect,
regular service adjustment, the main shaft diameter of 5/8'' had worn down to
1/2'', eliminating any further adjustment to be taken. So a rebuild of parts
was available and they were from stock in the USA. Parts were delivered within
a few days and the pump rebuilt. A bigger surprise greeted me on my return to
the oil rig. Records show that the engine had been in full use after D Day. It
had been used to power the system known as PLUTO that pumped the oil and fuel
at the bottom of the sea bed across the channel to supply all our allied forces
and had been removed after hostilities to continue in peacetime work. What a
wonderful history to find by chance.

My position as a Field Service Engineer encouraged the company to send me to
Geneva, the then Cat School of Europe, located at the foot of Mont Blanc. The
first week was power shift and torque converter transmission; the second week
was loader hydraulic system; but the most interesting course for me was the two
weeks on boat hull speeds propeller-to-power calculation. To touch on the
above, there is no doubt it is an exact science in itself but a miscalculation
on each would bring heartbreak! I must confirm that the Courses at the Geneva
Cat School were booked for many service dealerships' service engineers. It is
said that a little knowledge in the wrong hands can cause problems. This saying
came to mind when I was called to Southampton to carry out sea trials on a 55'
MV vessel using twin Caterpillar D434 engine. The boat itself was on a new
build completion up the Hamble in a boatyard run by Robin Knox-Johnson. After
the general inspection and test the boat was taken into the Solent for trials
and was near to completion when it was found that full power was not achieved.
We were down 50 rpm. All tests to the turbo charger indicated that engine was
up to performance. I said that the propellers pitch or diameter should be
checked for correct size. This comment of mine stimulated the question, ``could
you explain further?''. My (unqualified) response was that it could be that the
diameter of the propeller needed reducing by 1/2''. So I returned to Exeter in
the full knowledge that the propeller would be reinvestigated and I would be
recalled on completion. Eventually I returned to the Hamble for what should
have been a final sea trial. The result of the full power --- yes you've got
it --- had lost a further 50 rpm and was down by a total of 100 rpm. Another
lesson in life --- a little knowledge in the wrong hands can cause problems!
