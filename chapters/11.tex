%!TEX root = ../thorn-moor.tex

I have always been active in my garage and as life has taken a turn have been
given a 1937 metal lathe with multiple speed pulley drives and a belt. I have
dismantled this system and have converted a 3 phase motor to interphase with a
multi speed computer system thus eliminating the multiple pulley system.

I turned away from engines and gears and made staves wooden and brass fittings
for banners all used for parades in Plymouth. I have taught myself leather
work and made sword belts and frogs for parades. I have often wondered when
sewing up a thatcher's hook pouch using the saddle stitch if I had learned this
trade and followed the occupation and become a leather worker would I be
content in my peaceful workshop or would I yearn for something more active as I
did at Thorn Moor in my early days?

Exeter Workshop --- Work from out of the blue! The phone rang one morning,
``Charlie Mann here. Wonder if you could journey down to me this morning. I
have a job I think you could manage to overcome for me.'' Charlie's company
over the years had grasped every opportunity as presented and found themselves
in demand for supplying the film industry with actual mock ups of military
vehicles. It transpires that Colonel Gaddafi of Libya was financing a film
with the title `Lion of the Desert' involving the factual history of when Italy
invaded Lybia in the 1920s to attempt to increase the expansion of the Roman
Empire and Gaddafi's own popularity in the world as at that stage it was at an
all time low due to the disastrous crime of the Lockerby airline bombing. The
problem in front of us is that the film required some military tanks Fiat 2 man
type. A mock up had been made from a Land Rover but failed as the wheels had
sunk in the desert sand. This time it is hoped I could recreate the Fiat tank
as near to the real thing as possible by using a bulldozer. This I did by
using an International 125 track loader from our sale stock and removed all
existing items such as the Operator's cab and load frames. I redesigned the
track layout by extending the length, raising the idler to match up with the
profile of the only copy plans available from Bovington Tank Museum. The tests
were so satisfactory that we finally manufactured six. These were shipped to
Libya for filming. They are found if you go into Google, type in Lion of the
Desert and scroll down the pages until you find `military vehicle'. Designing
the bulldozer to replicate a Fiat tank was a very hairy affair as Charlie
Mann's group of very serious faced men and Charlie sitting at a table in a
portacabin they passed to me a photocopy pages of the profile of the Fiat tank.
One question from them ``Can you make one like this?'' I was sitting stunned
looking at nothing, my mind running wild trying to create the minds eye view of
the modifications required --- 15 minutes thinking time. My response was, ``We
will do this for you''. All stood up and shook hands, in cars and home to
Exeter. We selected an International 125 from the used sales fleet. There were
no complicated drawings made to scale. I just made free hand drawings as they
came into my head. We conducted a completely different discipline of this
operation. All regular and day to day workshop repairs were carried out to 5
o'clock. Then all stopped, had a break and something to eat thus ensuring we
returned to the project with a completely changed mindset.

All bulldozer equipment set was removed on the yard workshop floor, washed with
hot water and detergent with the machine positioned and all recreations marked
and drawn out by chalk on the floor. My boys responded to the challenge in top
shelf manner and by the end of the week of evening work there was a military
power unit. The film company representative came and gave it a full test
resulting in a fully satisfactory pass and immediately ordered a further five
modified units.

My Uncle Ronald once said, ``The trouble with folks moving down to Devon is that they
think we Devonians are daft because we have no sense!''. How true as when I was a young man I set out with one aim in view to
finish the project only to find that a journey down the other road would have
achieved a better result --- a fault that holds fast in future years.
I decided to build a double seated canoe using plans from a practical
woodworking magazine. The length of the designed vessel was 17' 6'' to be
built in my garage in a very cold winter season so fault 1. I cut the length
from 17' 6'' to 17' to accommodate the project in the garage enabling the door
to close and lock.
Mistake 2. To upgrade the marine ply from 3mm to 4mm. The hull spares were
increased in size. On completion the vessel looked solid but when launched it
immediately gave the impression that things were heavier than required and the
removal of the 6'' waterline length meant the hull sat lower in the water.
Nevertheless my friend and I entered into the Exeter Canoe Club race. Despite
being left at the post to start with after half an hour of paddling we began to
catch the competitors and overtook at regular intervals being very concerned
that it was like pulling a log through the water rather than a responsive
vessel. We reached Double Locks, turned and kept up our relentless paddling
finally reaching the finish at Exeter Basin. We were congratulated on
achieving 2nd position in the touring canoe section.
Congratulations were quickly followed by reality as we wondered if there were
only 2 touring canoes! A further follow up from there is to never look on one
side only. Turn the coin over and look at the tails as well as the heads ---
that applies to every avenue of problems presented.





I started one day to collect an architect from Hilltop in Plymouth and visited
a generator unit that is powered in a Devon and Cornwall Police Hilltop Radio
station. My companion was named Tom Jane.   We knew each other well as we
had carried out many of such inspections so conversation was always great but a
happening occurred which affected both of us after we looked at the generator
and started our quite long descent to return. I was driving quite slowly so as
not to cause damage as the path was very deeply rutted with small loose
boulders. Looking in the distance I could see a spiralling cloud of dust
moving very fast. Tom's voice said in alarm, ``Better pull in here where there
is a wide area''. We stopped and waited. The drive was in another world as a
van passed. The energy from it was unforgettable. We remarked to each other
that we would not like to be on top of the hill when he gets there.

Next morning Radio Devon News mentioned that a person had committed suicide on
Kit Hill and was asking if anyone had any information to contact Lostwithial
Police. I did give full particulars of what we had seen. The response was,
``Goodness! You are spot on. How did you report that information in such a
short time?'' I told them that the energy coming from that van was disturbing.
I could not help but see there was big trouble of some sort.

Whilst talking to the Police, let's talk of the policing of Thorn Moor, Cheriton
Bishop. The area policeman was a typical comic book build, tubby, red faced
and kindly man, named PC Tonkins who lived in the Police house at Tedburn St
Mary but to me a small boy was always in the right place of minor offences
committed by small boys.

One day my friend did not have a bike so not wishing him to walk I gave him a
lift. We arrived with David sitting on my rear carrier. The shock of seeing
PC Tonkins cycling towards me caused me to lose control and run up a very steep
bank and fall backwards with the bike on top of both of us. PC Tonkins stood
in the road and exclaimed that that demonstrated the reason for not riding 2
people on a bike. He waited to see if we were OK before continuing his
journey. Luckily it was outside of David's house and I continued to cycle
home. Sometimes we would see PC Tonkins ride past our gate on his way to his
favourite cider cellar and later in the day riding in a car with his bike on
the back having a very necessary lift home.

Some years later PC Tonkins retired and was replaced by a PC Cannon immediately
renamed by the locals as ``bang bang''. Big enquiries were made about his
character. I wanted to know whether or not he let minor deeds go unnoticed. I
also wanted to find out whether he enjoyed paperwork as if he did this would
not be good as he would probably make out a ticket for blowing our nose.

One evening I was aged about 16 and was walking quietly beside a hedge leading
to a gate to the road. As I got to the gate I checked again that the gun I was
carrying was empty and left open. Prior to opening the gate and stepping on to
the road I was confronted by PC Cannon or should I say Bang Bang. He had
remained hidden until I stepped on to the road. He greeted me with the
following ``Let's see, your name is Michael?'' ``Yes sir'', I replied. ``Let's
see'' he said ``Is that a 12 bore shot gun?'' Again I replied, ``Yes sir. It
needs a shot gun licence like this one'' I said pulling the very item from my
top pocket and handing it over. PC Cannon said. ``Ah yes. I thought you of
all people would have one.'' I thought, I expect he was hoping that I did not.
I watched him read through the form and I thought that he was a paper man so
nothing out of place will go unnoticed so watch out farmers in this area.

A rumour circulated that this PC had been relieved after the war as a prisoner
of the Japanese Death Railway. I thought that if that was the case, I would
forgive him his general unforgiving conduct.

Some years later my friend George Gillard and I were asked by PC Alan Parsley of
Whiddon Down if we would become Special Constables to support him with traffic
or, as happened at the time, check points following convicts' escape from
Dartmoor Prison. This we both did and spent many winter nights having
training at Moretonhampstead Town Hall. One evening when a member arrived late
he told us that President Kennedy had been assassinated. We all stood in
disbelief.

I was called to help a regular Constable with a road check at Beter Cross on
Dartmoor commencing at 7 pm until midnight. We sat in my car as the night grew
cold with heavy fog due so we could hear a car from miles away and would get
out of the car standing on each side of the road and wave our torches at the
car to stop. Some would not wish to stop as our torch lights would reveal
their passenger to be someone other than his wife. Not our care so we thanked
them for stopping and waved them on their way. In the late 1950s I was led to
understand convicts would escape to pay back a debt to the tobacco baron and
the longer they could stay out and cause disruption the more the debt was
reduced.