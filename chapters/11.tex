%!TEX root = ../thorn-moor.tex

William James Saffin, my Great Grandfather, lived at Poleford between
Crockernwell and Cheriton Bishop. He was a stone cracker as in his early days
the A30 road was not covered in tarmacadam (it was probably not invented then)
but was a hard earth surface with the covering of duck egg sized stones that
would gradually break down to small granulated rubber under the iron rim of
various stage coaches, farm wagons and carts. In fact Great Grandfather
remembered a special occasion when a man on horseback rode through with a first
motor car which drove through with a red flag bearing the words of a warning of
mortal danger and beware hanging on its side in the form of a notice.

This was the first motor car to drive through the A30 road. Dad said there was
one story Great Granddad told in solemn mood which was the appalling attitude
of the aristocracy towards the working man experienced by him.  The story goes
that Great Grandfather had finished work for the day and started to walk home.
After a time of walking he observed a gentleman smartly dressed riding a horse.
As he drew nearer my Great Grandfather gave a happy smile. His greeting was
received with a very cold, stony expression and he was taken completely by
surprise when this individual gave him a full and hefty lash with his horsewhip
across his back with a force draining all his strength causing him to fall on
his hands and knees. Fortunately he had been an active working man and was in
fairly good rugged bodily condition. His recovery was quite quick and as he
started to rise to his feet he selected a good size ragged stone. The horseman
had lingered to watch my Great Grandfather's plight while he was lying among
the dust and stones and turned away to continue his journey. My Great
Grandfather threw his stone as hard as he could hitting the back of the
horseman's head. He fell off his horse with one foot tangled in the stirrup.
Heregained balance, remounted and kicked the horse to continue his journey.

There was never a word spoken about the savage incident between my Great
Grandfather and the gentleman horse rider. I have often thought of the
unsavoury incident and wondered what blackness must have been in the heart of
the rider --- the only 2 men in a lonely country road, one giving a smile to
then receive a whip lash with no other people present to witness the violence.

John Preston Butt who was known to all as Jack Butt had talents which were
greatly admired by me as a small boy.  Naturally my awareness expanded as the
years went by and I immediately acknowledged his skill as a thatcher and his
ability in making roof spars, straw ropes and bee skips. Then one morning I
was aware that a letter addressed to Jack Butt had arrived by postman Mr
Hawkins (Smiler). Mother read the contents to Grandfather whilst he sat on the
Devon settle listening with great concentration --- we must remember that he
had limited writing skills and even less reading abilities. I recall
Grandfather saying ``I was taught how to read and write by a man who would say
if I stopped (because I did, not knowing the word) ‘read on boy' thinking
thicky man's daid (dead). Irrespective of this limited tuition he was able to
write on a slate with a long carpenter's pencil the customers' names and
numbers of spars made and bundled for them and total up the pounds shillings
and pence. Following on from the letter reading I found Jack Butt dressed in
his second best waistcoat and carrying his bowler hat and 3 or 4 newly cut
hazel sticks waiting for a taxi.

Naturally after his departure I made enquiries to be told that he had gone to do
some water divining for someone who wished to dig a new well. It was some
years later that he showed me the technique telling me that only I would know
when I was ready.

I have never committed myself to divine water for anyone as in my opinion it
could cost money. For my own peace of mind I have divined and found a healthy
flow of water running under Cowick Street shopping centre car park in Exeter –
may be a big drainage. I was also asked if I would confirm the finding of two
professional diviners who had reported the finding of a large spring flow of
water.

across the road on a driveway between Newton Abbot and Penn Inn. I attended and
to my horror and disappointment could not get any reaction from my hazel sticks
but walking 50 yards away a great reaction against a stone wall. I reported my
finding and disappointment to have lost the little gift I had! So I got
further help.

Approximately a month later I received a phone call from the land owner saying
that I should not be disappointed as they had carried out drilling probes in
the drive which confirmed that I was correct that there was no water under the
driveway but they had found water 50 yards away, against the wall which was
once used as a communal water tap from a well so I felt very pleased.

A letter arrived giving a communication for a date at Grandfather's workshop for
a taxi car arriving and the visitor going into the shed with the door tightly
closed for privacy time. After some years it was noted that a few small
parcels addressed to Jack Butt and tied up with string with the knots sealed
with sealing wax would be delivered and opened to reveal an ounce of Digger
shag tobacco or may be two.

Grandfather would never give away anything about charming warts. I was then
told that it must be passed from male to female so Mother received the
information from Grandfather but did not feel happy to use it so passed it to
me. I charmed many warts over the years but did not sell the gift although I
have received great pleasure when thanked after the warts disappeared. Folks
might laugh and doubt but if they get a wart on their nose they might try, if
it won't go, to see me.

The blizzard of 1963: The blizzard started just after 12 o'clock on the morning
of Boxing Day. I had milked the cows and all was in order as I thought it was
just another flurry of snow. Mother and father were stocked up with food and
normal warmth. Dad was recovering from an operation at this time.

I then drove to Brampford Speke to visit Vera for lunch as I was visiting the
love of my life. Lunch went into tea time and at 9.30 pm with no regard to the
weather I looked out of the window and dressed for the journey home. You can
imagine the horror and deep feeling of irresponsibility when I saw the extreme
blizzard conditions knowing that my mother was at home with my unwell father in
a very isolated house and me 15 miles away. At that time I was driving a Ford
1 Courtina with a three speed column gear change and small wheels which were
most unsuitable for traction!

However, I started to drive with a passion knowing that I must get home to my
parents. So I drove zigg zagging through snow drifts while keeping the speed
up to power through the drifts. People who have driven through snow storms have
experienced the feeling of uncertainty when driving up hill, along the flat or
down hill. I experienced this ungodly sensation. When I reached the top of
Burridge Hill the engine gained speed and I identified that I had gained the
ascent and was descending far too fast to negotiate the T junction 100 yards
ahead so I did my best to reduce speed well knowing that I was going to contact
the hedge, which I did. Fortunately only the front wheel ran up the bank
leaving the car sitting with the front wheel sitting in the hedge. So knowing
the rear wheels would not grip I pulled my raincoat out of the boot of the car
and placed it under the rear wheels, jumped back into the car, shut the engine
down, put the gear into reverse and pressed the start button and the car wound
its way back off --- a very handy trick that modern cars are unable to do.

I drove off leaving my raincoat in the road as the weather was too harsh to
linger. I drove continually zigg zagging through small snowdrifts knowing that
after Tedburn St Mary conditions would become much worse. Fortunately my
judgement could not have been further from the truth as in reality as the A30
became more exposed to the strong winds tearing out of Dartmoor the snow was
being blown over the hedges leaving me a reasonably clear run until I turned
off on to the Yeoford road. The reality of the blizzard turned my progress
belly up. The wind had blown the snow from the open area of Red Ridges over
the hedge to cause a massive 11' high snow drift across the road into the
entrance to Glebelands Estate. Just down the nap from the Village Hall
everything was a white out and this obstacle could not be seen so I went into
it at a fairly steady speed and I think the car climbed into the drift at about
8' high, first a rapid deceleration then to a complete stop also in complete
darkness as the headlights were buried in the mass of snow. The door could not
be opened as we were wedged in the drift so I wound down the window and climbed
out on to the snow.

The next big shock was when I sunk deeper down in the snow and with some degree
of panic I detached myself from the snowdrift leaving my car to rest with
windows open and with only the roof and windows visible but to be covered later
with the falling snow. I switched on my trusty Pifco lamp and set out to walk
home. After I had walked, sometimes through knee deep snow, I became aware of
a clanking noise around ears so feeling my head I was then aware that the
snowflakes landing on my warm head had melted and whilst running down my hair
had frozen in the cold harsh wind. How unwise not to be wearing a hat. On
arriving home I was relieved to hear both my parents' calls who responded with
reassurances that I was home.

Next morning found the snow was level with the windowsills and it was still
falling heavily. It was obvious to me that Thorn Moor, Little Moor Farm and
other isolated farms were cut off. Whilst feeding the cattle I reflected on my
very lucky and dodgy drive back the previous night. I thought what a stupid
selfish bullet headed individual I was well used to knowing the weather in
store with a responsibility for a sick father and a very anxious mother and
with animal feeding and milking to be done. However I settled down to the
running of Thorn Moor under blizzard conditions. Fortunately it was a holiday
break so work was out of the question. I cannot remember having a day when it
didn't snow but it eventually stopped.

I then walked to Checkers shop at Cheriton Bishop for provisions and rather than
return on the strength sapping powdery snow I decided to walk to Crockernwell
via the A30 and Hooperton Cross which was much longer and involved an extra
walk of about 2 miles in blizzard conditions through knee high snow. The worse
was trying to breathe in the corridor of very fine particles of broken down
cold snow that had filtered through.

I then set up my very unsuitable car for snow use, adjusted the rear tyres to
half normal pressure and front tyres a little higher than standard. I wrapped
a rope around the rear drive wheels to connect blocks in the boot. I gave the
farmer living at Little Thorn Farm £1. He then towed me through the heavy snow
to the stone landing. My Ford was left unlocked because of the extra low
temperature --- a lock would never open with the boot facing Dartmoor's extreme
weather.      . To keep the engine clear of snow blown by the wind each
morning I would walk from Thorn Moor then drive to work so never was a days'
work lost as we had plenty of water pump air vessels to cast weld together with
a couple of Allis Chalmers engine cast blocks to also weld.

After some days I was aware that the road had hard packs of snow so the car was
driven home. After some days it was noticed that I could see over the hedges
and walls so the thickness of the packed ice was approximately one inch deep
and I was very aware that the thaw would cause problems when the surface broke
up. So for some days over the thaw the car was left at Hooperton Cross and I
walked to and from it each day until the normal surface returned.
