%!TEX root = ../thorn-moor.tex

\begin{itemize}
  \item Weather and action required by farmers in the month of March: When the
   blackthorn is covered with white flowers it is called a blackthorn winter
   (period of very cold east winds requiring overcoats).  East wind is called a
   lazy wind, too lazy to go around but blows straight through. When the
   blackthorn blossom is white, till the barley day or night. March winds and
   April showers bring forth May flowers.

  \item Advice to youth in respect of clothing: never cast a clout until May is
   out.

  \item Regarding Dartmoor weather:
  \begin{itemize}
    \item When I was a young man living at home before marriage — location Thorn
     Moor Farm — field Big Pennypark — looking across the Devon countryside at
     15+ miles directly into Causdon Beacon, if the moor looked very clear and
     looks close whilst sitting at the bottom of Big Pennypark, rain will
     develop in the next few hours or days and the heavy laden rain
     filled-humidity will cause magnification of the moor. If the moor looks
     far away and purple we will receive fine dry weather in a period of time.
    \item Home wind blowing east: dry day; wind from the west: warmer and wet.
  \end{itemize}

  \item When Kirton (Crediton) was a busy market town, Exeter was nothing but a
   fuzzy down.

  \item Instructions from a farmer to his workmen: ``Boy! Rin, boy ride, boy rin
   and tell the doctor Mrs been and poked the cow's horn in her eye!''

  \item Two pigs in a house by themselves do better than one together.

  \item Farmer looking over his crop of potatoes in the rain: this shower of
   rain will spoil the little potatoes.

  \item Trouble with clocks going forward for summertime: it gets late early!
\end{itemize}