%!TEX root = ../thorn-moor.tex

Walking up the hill to the crossroads we have Hask Lane, a corner of which was
purchased by a young Scotsman known as Jock who set up home by building his own
small house in the corner of the field. It commanded wonderful views of Causdon
Beacon in the distance. Jock drove a three-wheeled BSA car and was known to
stand outside his house in the evening sun and play the bagpipes wearing full
regalia of his clan. We never knew his first name but was always called Jock
Turtington, and when he played the bagpipes in the evening he always got Jack
Francis living across the valley to respond. It was always so disturbing when
Jock started falling out with his cats with a mysterious smile. Further along
Hask Lane we have a very fine constructed red brick house called Highfield,
which speaks for itself with clear views of Haldon Belvedere, Drewsteignton
church and village, and Causdon Beacon. This property was developed by a
Colonel Ward. My first introduction to Colonel Ward was of a 6' plus man of
military stature getting out of his car bellowing ``hay, there you are Worthy
Anstee, and who is the young man?'' (looking me in the eye) and saying ``I
could do with you to look after me next Tuesday as my wife cannot be with me''.
I agreed and the date was duly noted and my expected duty explained: to stay
with him all day and, should he stumble and/or blackout, to turn him on his
back and lift his head until he recovered. Apparently this condition had been
caused by a head injury which was the result of a flying house brick in a war
zone. The day was accompanied by lovely weather, and the Colonel and I walked
around his fields and he explained to me his plans to build a red brick house
and the intended name. As I was walking I thought ``this sounds like a magical
dream that will never happen''. Well, it was built, leaving the area with a
very majestic house, Highfield, on a prime position today. We finished the day
drinking tea in the Colonel's very comfortable wooden shed furnished with two
ex-army camp beds for an afternoon snooze, with tables chairs and a gas ring
for boiling the kettle. Only an old water tower remains in this area today. I
remember cycling home that day and as a 9-year old feeling completely drained,
as I worried all day that the big man would blackout and hurt himself or be in
need of water, and whether I could lift his head high enough for his recovery.

At the cross roads we have North Down, and property built solely by my great
friend John Charlie Anstee. This refurbishment took place under extreme
physical and mental courage which will be disclosed later in the story. Across
the road was a very neglected field in poor condition with well established
weed pods drying out for the heavy cropping of next year's undesirable crop.
Also a poorly constructed single-story house that looked like it had been
constructed mid-war years with very little satisfactory materials and an equal
lack of building knowledge. The property became occupied by a short stocky
Yorkshire man, known as Yorky, who was acknowledged for his short but witty
responses. One morning, Colonel Ward blustered in the yard at hay harvest time
and exclaimed loudly ``there you are Yorky; I've come to borrow your hay bale
fork''. Yorky replied after a short pause laughing ``thou knowest you are too
late because you borrowed it last year''.

The next property situated on the left further down the road was a very fine
railway carriage converted to a caravan living unit. It had been converted by
Worthy Anstee who lived and farmed, as well as working at engineering projects,
at Hole Farm, located down the road. The conversion consisted of a redundant
railway carriage with sleeping quarters to the west side divided with
conventional railway toilet, cooking and storage quarters with sitting area,
and large railway windows covering the open countryside. Externally the heavy
cast rail wheels had been removed, complete with heavy undercarriage, and
replaced with rubber-tyred lorry wheels for lighter and easier manoeuvrability.
This unit made a wonderful homestead for Jack Francis and his nephew Ken
Francis. Jack was a well-known linesman throughout the region of Devon County
Council working with a small team whose sole duty was to clean and keep clear
all drainage and water gullies, to keep the road free from flooding. Let me
explain: in our early days the surface of the by-roads were kept clean and free
of water overrun by two water channels running each side of the road kept the
running water in full control directing it to culverts and clean drains,
enabling any excess water to flow away. This system worked very well. There was
also a team of men working in pairs who would cover the by-roads of Devon
correcting minor floods and would show themselves all working in unison until
some bright spark sitting in a high throne in County Hall decided to stop all
activity to save money with no alternative. So here we are 40-plus years on
with most water gutters gone, and flooded roads after a very rainy shower. No
means of correcting any flooded area as most drain offs have been filled in
with four wheel drive vehicles. Enough said at this stage. Ken Francis, the
nephew, had a matchless 350cc motorcycle. He used it for work and pleasure
purposes. One day news reached us that Ken had been involved in a road accident
resulting in a broken leg. This unfortunate event was forgotten for some
months, until his Uncle Jack told us that Ken was not returning to this area to
live with him because he had fallen in love with one of the nurses of the
hospital ward and was setting up home with a view to marriage. What joy to
think that this lad had a life-changing event that changed the mundane life
into joy.

Some 39 yards down the road on the left we have the entrance to Hole Farm. Quite
a farm of note with a large acreage of land that was farmed by the Ponsford
family, who were believed to have been very wealthy landowners, owning a large
acreage of Drewsteignton giving rise to the Drewe family who built Castle
Drogo. In the distant past I remember being taken in to Hole Farm by my mother
and father and sitting in the large living room with an extra-large hearth fire
that enabled one person to sit at each side of the fireplace. These visits
would normally happen once a year, as my Grandmother Ada was Uncle Jim's
sister, whose Dad liked to discuss farming and local country news together,
parts of which were then relayed back to his mum on the next visit to her. The
farm was sold some years later due to age catching up with Uncle Jim and the
sons wishing to follow different adventures in life. The farm was purchased by
a young Worthy Anstee who started life at Half Acre farm. I shall return to
Hole Farm and Worthy Anstee later, as my experience here formed my passion in
my adult life.

Further down the road on the right a bungalow was built by Tom Edwards, who
lived there after his retirement from the Crockernwell pub. One feature that
captured my mind was, thinking outside the box, electricity. A windmill or
turbine was constructed on a high pole (the trade name was Freelite) and each
time I cycled past, particularly in a gale of wind, with the windmill blades
rotating at screaming pitch, my thoughts were of being cut to pieces should the
outfit disintegrate! So as you can imagine I was well pleased to be clear of
the area when a gale was in progress.

Travelling towards Thorn Cross we arrive at our fields at the first hilltop:
Higher Pennypark. The main large field lay in on the left to the bottom of the
hill to the threshing plat. This main field is called Pennypark and further
down the Hittisleigh Road laying in the left is Little Pennypark. I have always
thought ``what lovely names for country fields''. Thorn Farm house in itself
was a large and majestic thatched dwelling and could carry a past history. In
my early life it was owned by the Burridge family. It changed ownership just
after the finish of the second world war, and an ex-Army Captain Lawrence and
his wife and two daughters took possession. Unfortunately Captain Lawrence
developed severe stress from military experiences and it was not uncommon for
him to patrol the area after dark with a double barrelled shot gun, firing at
any movement he thought could be demons closing in to get him. As we lived some
half a mile away we tended to keep the doors bolted and keep our movements
close to the homestead after dark and pub closure. This practice continued
unknown to him, as we know from history that these poor men, having returned
home to a balanced society, were under post military action and turned to drink
as a substitute to block out the horrors experienced.

Continuing another one and a half miles north, there stood a smallholding named
Higher Thorne, occupied by a Mr Frank Burridge, his wife Rene and two children,
Gerald and Marjorie. Frank had a cattle lorry that kept him busy delivering
local cattle to markets in Holsworthy, Okehampton, Exeter, Chagford, and
Tavistock, whilst filling in spare time working on the necessary duties of hay
and corn harvest and so on his smallholding. I remember spending a day lying on
the sitting room carpet with Gerald and Marjorie listening to the wedding of
our queen to the Duke of Edinburgh. Continuing down the road northwards we
arrive at Rydons Cross, the farmhouse being quite attractive from the road
view, owned by and farmed by a Mr Norrish. The house was divided with a Mrs
Coles and her young son, Alec, who was one year older than I. Mr Coles served
in the Army. Rydons farmhouse was quite attractive in general, but to me, a
small boy, quite a horrible establishment, as quite by chance a rumour had
circulated in a circle containing my mother that farmer Norrish was an expert
at cutting young boys hair, and action was put in progress one afternoon. So
mother set off from Tilery to Thornmoor one afternoon with a small boy
(me) strapped to the rear carrier of her bicycle to Rydons Farm. We were
greeted by a very kindly farmer who gave mother a seat by the window, and then
lifted me up to sit on a flat piece of wood straddling two arms of a large
fireside wooden chair. A towel was placed about my shoulders. Within the next
few minutes it was quickly apparent that nothing was further than the distance
between expert and no idea! The next painful period of time was spent feeding
me with squares of chocolate for appeasement. The end result was a complete bar
of chocolate eaten by yours truly and my hair looking like it had received an
electric shock!  Upon returning home mother said, ``Farmer Norrish seemed a
nice man and a good place to go''.  So you could imagine the four-weekly horror
of facing farmer Norrish armed with rusty, blunt scissors that pulled at my
hair and head.

Following the crossroad to turn to the left to Hittisleigh, and some two miles
from the crossroads, laying in on the left and a quarter of a mile off the
road, is a very old and natural farm named West Pitton, occupied by Mr and Mrs
Partridge and their son, Roy. Mr Partridge worked at Lapford milk factory and
would cycle to that place of work every day, winter and summer, rain, snow or
shine. Mother would visit Mrs Partridge on a fortnightly corridor using a very
sturdy `sit up and beg' cycle with me sitting on the rear carrier chair. Roy
the son farmed the land in a very natural manner --- in fact never moving
forward into the mechanical area to use tractors. He possessed two lovely
Devon shire horses that assisted with the ploughing and the working of the land
for the crops, with both horses fully groomed and set up in full leather
harnesses and looking like something from a picture book. Sometimes war action
would visit this isolated countryside and on one occasion a dog fight had taken
place between our Spitfire and enemy aircraft, resulting in a Spitfire taking a
hit and rendering it unable to fly, so it crashed into the hillside. I can
remember a soldier on duty keeping guard of the wreckage and leaving the field
completely bland and unable to grow crops due to the contamination of aviation
fuel --- not even weeds would grow, leaving the soil exposed for many years.

Having given a built up picture of this area we return to my to my birth place
and the features attached to a worked-out roof, tile pits, and out of use kiln.
The best season to start would be deep in the month of January: the winter rain
had filled the three spent clay pits, turning them into a haven for wild life.
The first indication that nature was shaking itself off the winter's slumber
commenced with a few croaks that raised from the three ponds developing into a
crescendo as male frogs and toads continued the mating call for a partner. This
would grow in volume as the night wore on and my own bedroom was filled with
the frog chorus. Moorhens and dipchicks followed and paired up as the spring
unfolded into late spring and early summer, when the ponds teemed with wildlife
such as newts, dragon flies, water beetles, and surface skaters. If a Devon
wildlife pond dip could have been carried out then I expect we would be
horrified to measure what life we have lost today. Life continued to increase,
as mother would always hatch a dozen chicks (with normally nine surviving).
Bluetits nest in old rotten trunks of cider apple trees. Curlew and plover
would nest in our fields, and it was such a relaxing sight to see buzzards
soaring on the warm thermals. Grandfather Butt would have breakfast and always
asked mother to give him an update on the progress of the war. Then he would
walk from the house down the road to his galvanised workshop and sit on his
well-used chair and with faithful thatcher's hook made thatching spars.
Sometimes a visitor would find him making a beehive or skip, all from natural
sources of reed and sown together with a hazel thread made totally by
Grandfather. He had tubs of water with small bundles of these hazel windings to
keep them moist. With the war in progress, Dad was working on the farm of a Mr
Joe Coren at Shortacombe Farm, Yeoford. At some time Dad purchased a BSA 250cc
girder forked sprung motorcycle to save having to cycle. Grandfather was a
beekeeper and kept a total of 15 hives, all aforementioned natural home-made
skips. I can say nothing was manufactured in his workshop. He even used natural
fibre: bark was stripped off the willow and used as a binder for the bundle of
spars and twisted so expertly as not to undo.
