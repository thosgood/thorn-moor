%!TEX root = ../thorn-moor.tex

The whole set up was so extreme you could not even imagine such a thing could
exist.  Whilst I worked in the workshop I kept the door that opened on to the
street wide open whilst well loved children played happily with sheer delight.
Well loved but wearing thread bare clothes that were washed clean and well
darned and with patches where the pattern failed to match anything indicating
that it was cloth cut form worn out material ready for the trash bin.  Not once
did I receive a visit from any one ownership or other.

The boat was reassembled and performed satisfactorily under test and returned to
service, probably having to contend with the old habit.  Work was so varied
that having flown back from Malta I was called to attend a CAT D8 using a Kelly
ripper working on the face of the Meldon Dam wall.  The right hand track had
parted under a severe load and had whiplashed and landed upside down on the
track frame.  So it looked an impossible repair location so with a heavy heart
and with help from people on site we lowered the arc welder together with a
track pin repair kit and hydraulic piece kit to rest on the bulldozer blade in
the Dam wall.

Meldon Dam, Okehampton:  From the start to completion of the very very difficult
situation just everything went right for me.  The track was lying upside down
on top of the track frame.  I pulled it with a crow bar.  It fell off and
landed the correct side up and down hill sideways ending up positioned
correctly under the track rollers.  We took time on every move because of the
dangerous position we were located in and at each stage everything fell in
place so by 2.30 pm four hours after commencement the machine was returning to
work with a promise by the operator driver to drive up the less steep pathway
and to use the Kelly ripper to rip out going down only.  I walked back to my
van to write a report shaking my head in disbelief.

If anyone does not believe this story, get in your car, drive to Meldon Dam near
Okehampton, park up and walk to the dam wall, keyed into the valley A30 road
side and look down and try to imagine a large machine one third of the way up
with one track off. I have not been to Meldon for some time but the last time I
walked away in wonderment of the achievement of those past times.  There were
no Health and Safety rules then or hard hats worn.

Change of Lifestyles:  Vera and I sat down and did a rain check on our
lifestyles to date.  I was living out of a suitcase so radical changes were
made.  I joined Saville Tractors at Marsh Barton as a Workshop Supervisor and
life continued normally until two things happened.  Unexpectedly I was quickly
promoted to Service Manager and the company became main dealers of Listers of
Dursley, Glos.   This was just a homecoming for me because when I was an
agricultural engineer in my early days the Company held a Lister Generator
Dealership so I was very familiar with the product.  So instead of
familiarisation from scratch I immediately ran with the ball and put a small
team together repairing Start O Matic generators and moved into this marginal
mains failure generation winning contracts to install and service units for the
shop retail industry.  We did have the service contract for Devon and Cornwall
Police radio hilltop sites.  Then one morning there was a major announcement –
we had been taken over by a large conglomerate called Lonhro and Exeter was to
change into a Sales and Service Centre for heavy German lorries M-A-N and
Volkeswagon commercial vehicles.

I stood in my office looking into space and thinking all my working life I had
worked in the world where we judge reliability of services required by hours
worked and now we entered the world of miles operated and as I stood there that
morning I had no idea of what was a good or a bad performance.  Nevertheless
the boys were selected in their respective product team to meet the challenge
and meet it they did as proof the M-A-N Volkeswagon Company ran an inter-dealer
competition involving technical questions and performance on call out section
and my boys won the Gold Team award I think 6 years running.  One award was a
visit to Castle Coombe in Wiltshire to try out race driving skills on Formula
Ford racing and clay pigeon shooting, enough awards to muster confidence.
Policies changed within the Lonhro Group and it was not the thing to operate a
service operation.  For the first time in my life I was made redundant.

I then moved into a different world of sales and parts promotion of Lister
components to plant hire companies and the sale of marine engines in boat yards
and yacht marinas throughout the south of England.  This was a very
satisfactory occupation for me as the company Sleeman Hawken had the ability to
provide and supply parts overnight so it was great to speak to a customer with
confidence.

I then retired and have spent some days on my yacht moored at Dittisham on the
River Dart.  I had spent my working days trouble shooting motor vessels and
decided to have peace and quiet so I purchased a little Super Seal 26' yacht
and have spent  time sailing the south coast --- Dartmouth to Salcombe or
Dartmouth to Exmouth --- nothing too taxing as I have accomplished that.

As I did not want to close down completely after an active life Vera and I
decided to volunteer as speakers for Devon Air Ambulance so we have travelled
across the county of Devon, me standing out in front as a speaker and Vera
working the laptop power point unit.  We started 16 years ago, acknowledged the
challenges needed to keep up to date with cost and expansion.  I have been an
auctioneer selling cakes, bread, jars of jam and marmalade until the pandemic
closed power point operations.

I have always been active in my garage and as life has taken a turn have been
given a 1937 metal lathe with multiple speed pulley drives and a belt.   I have
dismantled this system and have converted a 3 phase motor to interphase with a
multi speed computer system thus eliminating the multiple pulley system.

I turned away from engines and gears and made staves wooden and brass fittings
for banners all used for parades in Plymouth.  I have taught myself leather
work and made sword belts and frogs for parades.  I have often wondered when
sewing up a thatcher's hook pouch using the saddle stitch if I had learned this
trade and followed the occupation and become a leather worker would I be
content in my peaceful workshop or would I yearn for something more active as I
did at Thorn Moor in my early days?

Exeter Workshop --- Work from out of the blue!  The phone rang one morning,
``Charlie Mann here.  Wonder if you could journey down to me this morning.   I
have a job I think you could manage to overcome for me.''  Charlie's company
over the years had grasped every opportunity as presented and found themselves
in demand for supplying the film industry with actual mock ups of military
vehicles.  It transpires that Colonel Gaddafi of Libya was financing a film
with the title ‘Lion of the Desert' involving the factual history of when Italy
invaded Lybia in the 1920s to attempt to increase the expansion of the Roman
Empire and Gaddafi's own popularity in the world as at that stage it was at an
all time low due to the disastrous crime of the Lockerby airline bombing.  The
problem in front of us is that the film required some military tanks Fiat 2 man
type.  A mock up  had been made from a Land Rover but failed as the wheels had
sunk in the desert sand.  This time it is hoped I could recreate the Fiat tank
as near to the real thing as possible by using a bulldozer.  This I did by
using an International 125 track loader from our sale stock and removed all
existing items such as the Operator's cab and load frames.  I redesigned the
track layout by extending the length, raising the idler to match up with the
profile of the only copy plans available from Bovington Tank Museum.  The tests
were so satisfactory that we finally manufactured six.  These were shipped to
Libya for filming.  They are found if you go into Google, type in Lion of the
Desert and scroll down the pages until you find ‘military vehicle'.  Designing
the bulldozer to replicate a Fiat tank was a very hairy affair as Charlie
Mann's group of very serious faced men and Charlie sitting at a table in a
portacabin they passed to me a photocopy pages of the profile of the Fiat tank.
One question from them ``Can you make one like this?''  I was sitting stunned
looking at nothing, my mind running wild trying to create the minds eye view of
the modifications required --- 15 minutes thinking time.  My response was, ``We
will do this for you''.  All stood up and shook hands, in cars and home to
Exeter.  We selected an International 125 from the used sales fleet. There were
no complicated drawings made to scale.  I just made free hand drawings as they
came into my head.  We conducted a completely different discipline of this
operation.   All regular and day to day workshop repairs were carried out to 5
o'clock.  Then all stopped, had a break and something to eat thus ensuring we
returned to the project with a completely changed mindset.

All bulldozer equipment set was removed on the yard workshop floor, washed with
hot water and detergent with the machine positioned and all recreations marked
and drawn out by chalk on the floor.  My boys responded to the challenge in top
shelf manner and by the end of the week of evening work there was a military
power unit.  The film company representative came and gave it a full test
resulting in a fully satisfactory pass and immediately ordered a further five
modified units.

Bee Keeping:  Having changed my lifestyle of working to the present normal
weekly hours I immediately started to put into place my deep wish and mirror my
Grandfather's life to become a bee keeper.  So I began to purchase and put
together equipment for the required need from a great friend of mine, Bill
Smith of Exeter Bee Supplies.  So I was well into bee keeping with 17 hives, 2
W.B.Cs, the rest national hives, 4 located on the roof of the deep freeze room
in my home and some in Days Pottles Lane, Exminster near Exeter and some beside
the railway line on Marsh Barton.  We would spin the honey into honey plastic
buckets which would hold about 1 cwt.  With the vast mix of trees around
Middlemore the quality of the honey was exceptional.  (Tim please this
paragraph should go in where I started working for Savilles)

A shock response greeted me one evening when I was stung in the face quite by
chance like it or not and I pretended it did not happen.  A trip to the
hospital was very necessary resulting in the solemn advice that stings over 14
years had built up so I was strongly advised to give up and get rid of the bees
before they got rid of me.  This was a shock to my system but nevertheless I
was thankful that I had had the experience of keeping bees.

I apologise to anyone who, having read this book, and walked away
thinking, ‘this man is arrogant and talks about himself'.  This opinion of me
is very disappointing as it is not intended to read that way as in the past 6
years I have been asked to put my life on paper before it is lost.  So I put
pen to paper in the full knowledge that I have no idea how to write a book or
to put any events in to the correct grammar context!  I just set forth!  What
is correct grammar?  Define grammar --- grammar teaches us the proper arrangement
of words according to the ideals and dialects of any particular kingdom of
people.  It teaches us to speak and write with precision.  Agreeably to
question authority so I am a Devonian born and bred in Devon so the West
Country culture sits squarely on my shoulders.

I thank my wife, Vera, for standing and supporting me throughout my many years
of projects. Without that support there would be nothing to write about and big
thanks to my lifelong friend Patience who took away my scribbles to be formed
into a booklet.

The trouble with we Devonians is that people think we are daft because we have
no sense.  How true as when I was a young man I set out with one aim in view to
finish the project only to find that a journey down the other road would have
achieved a better result --- a fault that holds fast in future years.

I decided to build a double seated canoe using plans from a practical
woodworking magazine.  The length of the designed vessel was 17' 6'' to be
built in my garage in a very cold winter season so fault 1.  I cut the length
from 17' 6'' to 17' to accommodate the project in the garage enabling the door
to close and lock.

Mistake 2.  To upgrade the marine ply from 3mm to 4mm.  The hull spares were
increased in size.  On completion the vessel looked solid but when launched it
immediately gave the impression that things were heavier than required and the
removal of the 6'' waterline length meant the hull sat lower in the water.
Nevertheless my friend and I entered into the Exeter Canoe Club race.   Despite
being left at the post to start with after half an hour of paddling we began to
catch the competitors and overtook at regular intervals being very concerned
that it was like pulling a log through the water rather than a responsive
vessel.  We reached Double Locks, turned and kept up our relentless paddling
finally reaching the finish at Exeter Basin.  We were congratulated on
achieving 2nd position in the touring canoe section.

Congratulations were quickly followed by reality as we wondered if there were
only 2 touring canoes!  A further follow up from there is to never look on one
side only.  Turn the coin over and look at the tails as well as the heads -
that applies to every avenue of problems presented.

Light and Water:  I have no idea what my father thought when he married and
followed his bride to live at Thorn Moor or Tilery House.  He had left a small
but solid house, all windows and doors fitted with no wind leaks.  A black iron
stove was in the kitchen, a bathroom with hot and cold running water and
outside an ash fill toilet.

Thorn Moor/Tilery House: no hot or cold running water.  A standing cold water
pipe stood outside the back door.  The hot water was heated by mother filling a
4 gallon cast iron crock with cold water in the morning and adjusting it over
one-quarter of the hearth fire with a big a brass tap looking out to draw off
water as required.   As a little person I was used to lighting the candles.
Then mother installed a table lamp which worked on paraffin.

The next big step was my father purchased a Tilly lamp from Reg Stanbury.  It
was great and gave heat to the room.  Cooking by a hearth fire:  kettle like a
giant egg poacher.  Years passed then calor gas came.   A cooker with gas rings
standing beside a pipe to a gas cylinder with one in reserve for changing over
and with copper tubes running across the ceiling to feed the hanging lamps.
Then a great step for mankind, a South Western Electricity Board van plus 2
workmen came up and installed the wiring and light and power points.  The rest
is history.
