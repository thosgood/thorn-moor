%!TEX root = ../thorn-moor.tex

Devon life continued in the usual pattern with the normal fall of snow over
January and February. I have particularly separated my experience of the worst
winter of 1947 when I was 9 years old. I awoke one morning to snow falling
heavily and standing 4 inches deep. My Grandfather John Preston was still with
us, and when he came down for his breakfast at 9.30am said ``I've seen worse
than this when the River Exe froze over and they roasted an ox on it'' and he
promptly moved and sat on the old settle wrapped around the side of the open
fire and continued with breakfast already prepared by my mother. The snow
continued to fall every day with the same response from John Preston until
Friday morning by which time the level of the snow outside had reached the
bottom quarter of the kitchen window and was level with the far banks across
the orchard, making all tree trunks appear short. The snow was still falling
with extreme cold making the flakes smaller. This continued until nightfall,
and with darkness the temperature changed and we realised that the small
snowflakes were indeed misty rain. We had no idea what was going to be the
result of this condition, but we did next morning in the full light of day: the
countryside was beautiful and looked like a tinsel town with all the trees and
bushes encased in ice. The falling misty rain continued all day. Consequently
the encased ice bushes became heavy and nature gave way to the weight, and at
intervals big crashes could be heard in the woods and hedges as branches broke
off under the overbearing weight of the ice encasement. The crash was closely
followed by a continuing sound of tinkling bells as the ice particles broke
into pieces as they cascaded through the remaining ice encased trees. When you
picked up the beautiful branch to examine it, the heat of the sun soon melted
the ice encasement and you would be holding a very wet and just normal looking
branch.

The temperature raised and spring entered our lives. Some evenings would find me
stalking the hedgerows with father, and his trusty double barrel 12-bore shot
gun under my arm looking for a rabbit to change the menu from beef and pork,
and later in the season helping father to receive the open hogsheads of newly
pounded cider juice, always discussing the quantity of the apple crop this
year. Every few weeks would find us racking off the cider. This entailed
draining off the clear clean liquid into a clean barrel and leaving the cloudy
to ferment out further. We would finish the racking period with about four
hogsheads of clear cider and one and a half with cloudy cider to use at a later
date. Time drew near for Dad and I to have a discussion about work and the
direction I would take. Dad made a phone call to Jack Saunders of Whiddon Down
and a date was set for me to attend one evening. We caught the Okehampton bus
and duly arrived at Whiddon Down. My first impression was The Post Inn on the
left with a long green shed running quite a distance on the right, which was a
blacksmiths shop coupled to a wheelwright shop combined. Little did I realise
how much time I would spend working the finer arts of ironwork and canvas
drive-belt manufacturing in the shed. We walked down through the yard with the
green corrugated workshop standing on each side and met Mr (and Mrs) John
Saunders, who I found to be a very quiet unassuming gentleman. It was agreed
that I should commence work one week after leaving school (a week to adapt to
the life change). It is said that help arrives from the most unexpected places.
My uncle, Les Stevens, who lived at Red Ridges, Cheriton Bishop, had offered to
give me a lift on the back of his motorcycle from Hooperton Cross to Whiddon
Down and return in the evenings, so I cycled to Hooperton Cross, jumped on the
pillion seat of my Uncle Les' motorcycle and was delivered to outside the long
green blacksmith's shop. I was met by Mr John Saunders who teamed me up with
another fine gentleman named George Endacott, the main waterworks man for the
depot, using a short wheel base Land Rover as a service vehicle. My first job
of the day was to attend and replace a set of knotter bills in a corn binder
standing in a field of the parish of Spreyton. I was so delighted to contribute
to repair as my small schoolboy hands could find entry into the system and line
up the drive gear roll pip with the knotter bill shaft. These hands changed
over the early years of engineering, as finger joints started to look like ball
bearings and the right hand thumb becoming quite flat due to constantly coming
second to the hammer on a miss-hit to the chisel head.

Much and various work was carried out throughout the region with some standing
out in my mind as very hairy in some instances. We were called to Wood House in
South Tawton, a fine house with quite a past history. The fault was to correct
a malfunction in a very deep fresh water well. It was reputed to have a depth
of 90' from surface cover to the bottom of the well. It was so deep it used a
secondary foot valve system to lift the water to the surface. To achieve access
to the problem George Endacott bound two full-length ladders with rope to serve
as one, lowered the result down the well, and slid sturdy wooden batons through
the top rung to hold, whilst a third full-length ladder was held vertically by
two men, which George again bound tight to the two ladders already down the
well. On completion of binding, the three ladders were then lowered further and
the sturdy wood bottom was slid in position to secure them all. A rope was tied
around George's waist and the loose end tied to a nearby post. Then,
unbelievably, George entered the open well. I watched through the well opening
until my hero looked minute in the gloom. Great care was taken by me not to
allow any stones to fall, as George was only wearing a cloth cap and a stone
falling from that height would have been catastrophic. All tools were lowered
by rope in a metal bucket. The faulty bucket system was removed and the item
repaired on the back of the Land Rover. Some time after I thought the physical
endurance of George Endacott was remarkable as he descended and ascended the
three full lengths of ladders and held on to its location at the bottom whilst
repairs were executed.

Another time, had I kept a diary its day should have read ``A funny thing
happened at work today''. The job in question was to attend a newly constructed
or sunken well that had been lined with two cement rings and install a new
suction pipe, the source of water being very good. So the well depth would be
approximately 28 to 30 feet. The surface of the water was approximately 6 feet
from the surface of the well, and, the well being newly dug, was still cloudy
but clearing. George unloaded the trusty double ladders and set them down into
the murky water. Tools and components were loaded into the roped galvanised
bucket ready for me to haul in when all was ready. George continued to step
down the ladder until he was in a position 6 inches above the surface of the
water. He then gave the signal for the bucket of tools. I turned to pick up
some and then heard an almighty splashing and coughing. Looking into the water
I was greeted with George's cloth cap floating on the surface together with the
top end of the ladder. Unknown to George the ladder had stopped 6 inches off
the bottom and rested on the small fragile ledge 6 feet from the bottom of the
well, and had given away under the weight of George and the ladder, and
continued to the well bottom. When I looked in the well George's cloth cap was
floating around, shortly accompanied by George's head resurfacing with eyeballs
that were so fixed that they looked like they had just emerged from a Soho
strip club. George recovered and we immediately drove off the site to George's
home at Throwleigh for a complete change of clothes and to dry off. It was a
good job the Land Rover was constructed by sheet aluminium as the volume of
water draining through the bottom and the side door was continuous on the
journey to Throwleigh. The incident passed and I was aware the subject was
never discussed again and was left to die with the time.

Later, we were instructed to attend the age old pub of Uncle Tom Cobley at
Spreyton and vent the well when a few days of repair work could be executed. In
short a visit was made to this well which was located indoors. The mats were
removed and the large slate stone cover was removed from the well top and the
safety guard put in place. This was necessary as the well had a reputation of
building up dangerous gases, so we left the well open for two days and returned
mid-morning on the third day and lit a candle which was tied to a long pole and
lowered into the well to see if the poisonous gas with no oxygen still existed
to keep the candle burning continuously. If the flame extinguished it is
considered unsafe as no life giving oxygen was present. This last test should
always be carried out when repair to a water well without history is to be
made. In the late 1940s and early 1950s there was a programme which entailed
the Compere, Wilfred Pickles, asking contestants general knowledge questions
and always the one, ``Can you tell us the most embarrassing moment of your
life?'' The lady questioned said, ``Yes. My father ran into the farmhouse
shouting, \emph{Quick, ring the man that puts the ram right}''. She did as
instructed and the veterinary came to be confronted by an equally embarrassed
farmer shouting ``\emph{No, I need a plumber to repair the water ram!}''.
Whilst I was travelling around with George I had never encountered this amazing
piece of equipment until I was out of my apprenticeship and carrying out every
repair needed in agriculture. The boss called me one afternoon and said that he
would like me to attend an address between Throwleigh and Chagford and install
a complete new pumping system to replace an existing system using a water ram.
Evidently it had been overhauled by a specialist using all new rubber washers
and diaphragms, but since the rebuild had never worked continuously patience
had run out thus we had been called in to carry out the major conversion. I set
forth armed with a Lister domestic water pump with an electric drive motor, a
coil of electric power cable, and a polythene water pipe. Whilst preparing the
pump house for the conversion base I could not help myself but investigate this
wonderful piece of equipment. So whilst laying the new foundations I stopped
and started the ram using various adjustments to no avail. It was while I was
home that evening I remembered a remark from Albert White : ``They water rams are
hard''. They will run for 20 years without stopping but can be stopped to carry
out minor repairs (they are little beggars to deal with) --- just one little wire
to restart and keep going. Sometimes a piece of wire twisted under the
diaphragm grid to make the surface uneven works magic. So the next day armed
with a piece of galvanised wire I twisted the wire in position and restarted
the unit immediately. I could hear a heavy ring to the pulse and it continued
all the morning. I was bolting in the new electric pump. I called the owner and
explained the situation. We stood and listened to the working unit. It was
agreed that I stop work on the replacement. He agreed to pay for all the
material and labour provided to date. I packed up and went back to the Depot
and having heard nothing more I assumed that things were working in a long
monotonous manner. So somewhere on Dartmoor stands a pump house with 2 types of
pumps. I often wonder if questions are ever asked. This proves that we should
listen to everyone and pick out what you think best. 
