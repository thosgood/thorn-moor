%!TEX root = ../thorn-moor.tex

How lucky was I to have been born and reared through to adulthood at Thorn Moor,
superbly located on the perimeter of Dartmoor, high enough to see Exeter
blackout glow and look in to Dartmoor's Causdon Beacon and snow line. Followed
by the charm of a deep valley and surrounding, my home, where there were reeded
marshlands and ponds. This wonderful expanse of terrain enabled me to
experience first-hand a part of the war action experienced in Normandy, and the
east coast seemed to live among the wildlife we can only dream of today. I will
briefly explain the Exeter bombing, which seemed to me to be a set pattern as
follows. An air raid siren would sound around the countryside. Mother would
take me from my bed and she and Dad and I went into the top bedroom and Mother
would have extinguished anything that could be lit --- not even a birthday
candle survived. We set our eyes to the eastern horizon and waited until the
dark bombers arrived, showering down fully lit colourful flares on Exeter,
giving a marker to the bombers that followed. Even from 12 or 14 miles away the
bombs and bombing aircraft engines were intense to my ears and in a short time
a group would be flying directly over our house. With Spitfires chasing them
over their flight path course over Dartmoor. Later in life I discovered the
reason the flight path was following the same course --- Germany having
occupied Normandy, the bombers were stationed near Caen and following a flight
path from Cherbourg across the Channel to Start Point, and then turning and
following the coast line to Exmouth, and then turning inland and following the
River Exe to Exeter to drop bombs --- mission completed. They would set a
compass course taking them over the area to Dartmoor and link up with the River
Dart to Dartmouth, then reset for the channel crossing to Cherbourg and home.
We did have some trauma after such a raid. We never gave any thought to the
machine gun bullets, but one morning Dad went into the vegetable garden and
found that a bullet from a plane had deeply grooved the handle of the garden
fork. The next time he was outside the Post Office at Cheriton Bishop, around
the corner from the school, he met Mrs Rice, a lovely serene lady who normally
wore a black dress with a spotless white collar and apron, and, finishing the
normal greeting, referred to the aerial battle two nights before, relaying
the bullet groove story to her, Mrs Rice responded by throwing her hands in the
air and exclaimed ``Oh my! I have never known such bravery like it. Come in and
have a cup of tea'' --- a hero for the day.

With regard to the wildlife that existed in my area, I will do an imaginary
walk, recalling life as it existed in those days. Coming out of the back door
and walking slowly around to the front garden, I walk over to a privet hedge of
medium height that was shielding a steep slope to the vegetable garden
(remember the bullet in the fork handle) and an orchard of cider apples with a
family tree George Ponsford cider, Tom Pud, and a Queenie red eating apple. I
would part the bushes of the privet and without fail would find a bullfinch
sitting on her nest. What a joyous sight! Continuing my walk down the path to
the road gate, I am aware of the steady hum from my Grandfather's top apiary of
bees consisting of one WBC and eight or nine home-made straw skips. As it is
nearly the end of July and the cut-off of the honey flow, my mind lingered on
the activity that will take place here in early August when honey will be
extracted in the old-fashioned way. I turned and walk down the hill passing
Little Thorn Farm over the small bridge and to the next field on the left. I
open the gate --- I did not climb over it as it was someone else's property. I
step in a few paces and look at the mixture of colours caused by the rough clay
ground, barren in areas with multitudes of stones the size of large potatoes
covering the surface, a haven for ground birds to lay their clutch of eggs
keeping them well concealed. So we had an abundance of curlew, larks, and
lapwing throughout the field giving a healthy replacement to nature
(not possible today). Today we have too many self-appointed naturalist
officials in high authority laying laws of preservation. The result is that
songbirds and ground birds are diminishing at an alarming rate. I will tell you
as it was in the early days: a sparrow hawk was looked on as no better than a
rat. It would catch small hen birds and kill them so the chicks would starve
and die, causing the end of that species for that year. So the sparrow hawk was
shot on sight thus reducing the numbers of these wasteful predators. The
badger, one of the most destructive predators to wild life, will eat all ground
birds' eggs and chicks and kill with a cruelty and savage any hedgehogs. No
wonder the nation is requested to report on the hedgehog population in the
garden, if any. I am afraid the sightings will diminish in time. In 2018 we
were recording by a night camera the movements of a family of four hedgehogs.
Then one day my neighbour told me with great joy that a badger had moved into a
fox home nearby. I told him that would be the end of the hedgehogs and, sure
enough, no more hedgehogs were seen by the overnight camera. In past times
badgers were considered a predator and encouraged to move on, as were the
magpies which do so much destruction to nesting birds. But in this instance I
walk out of this field of life feeling fully confident that the chicks reared
will add to the remaining wild life in this area. I return and retrace my
footsteps, continuing past my front road gate and continue up the road. I am
delighted that many yellow hammers are flying out of the hedge about 2 feet
high (nest height) in their continuing circle of chick feeding. I would repeat,
an abundance in those days, but regrettably not one yellow hammer is seen
flying today. I climb over the gate into my field on my return journey home
knowing full well that in August some pheasants will be nesting in the
well-established grass. Returning to the field at the rear of the house, nature
has fashioned it to slope down to an acre of wetland of approximately one acre
consisting of a healthy stream, resulting in the growth of water rushes. The
area was a mass of reeds and rich green grass that looked deserted but, should
you walk slowly across, you would suddenly spring to life as a snipe would take
flight with a squeal and combine a characteristic bowel movement from under
your feet, giving you a severe start before the heart returned to the normal
rhythm. A pair of woodcock and a jay would hurriedly leave the safety of a good
hideaway and fly into the next field of brushwood. This was proof that the area
was full and in abundance of small mammals that made up a plentiful food chain.
Two owls had taken up residence in the very old hollow tree and raised three
broods. Eight magpies and sparrow hawks were no threat to small songbirds as
they were well controlled. Even wood pigeons thrived to be shot in early winter
for pigeon breast pie. Remember: food was rationed.

I have no idea what my father thought when he married and followed his bride to
live at Thorn Moor or Tilery House. He had left a small but solid house, all
windows and doors fitted with no wind leaks; a black iron stove was in the
kitchen, a bathroom with hot and cold running water and outside an ash fill
toilet. In Thorn Moor/Tilery House, there was no hot or cold running water: a
standing cold water pipe stood outside the back door; the hot water was heated
by mother filling a four-gallon cast iron crock with cold water in the morning
and adjusting it over one-quarter of the hearth fire with a big brass tap
looking out to draw off water as required. As a little person I was used to
lighting the candles. Then mother installed a table lamp which worked on
paraffin. The next big step was my father purchasing a Tilly lamp from Reg
Stanbury. It was great and gave heat to the room. Cooking by a hearth fire:
kettle like a giant egg poacher. Years passed then Calor gas came. A cooker
with gas rings standing beside a pipe to a gas cylinder with one in reserve for
changing over and with copper tubes running across the ceiling to feed the
hanging lamps. Then a great step for mankind: a South Western Electricity Board
van plus two workmen came up and installed the wiring and light and power
points. The rest is history.
