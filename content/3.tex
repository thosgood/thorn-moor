%!TEX root = ../thorn-moor.tex

I was about 10 years old when one Sunday I was walking around Thorn Moor with my
gun looking for a lone pigeon. I jumped off a bank while paying no particular
attention to the landing area. This resulted in me landing on a broken bottle,
cutting my ankle into the bone. Luckily I was near the house so was able to
hobble indoors. One look from my father who asked me how did it happen and who
shall we find on a Sunday afternoon to stitch it, as our Dr Jackson lived in
Crediton. Suddenly inspiration came to my father who decided to ring Dr
Bonnelly, a new man living at Hittisleigh Mill who was an ex-army doctor
starting up as a goat farmer. ``I will ring him''. He arrived in a Land Rover
and said to my mother ``Clear the big dinner table and put a pillow at one end
and let me have two pudding basins''. He poured methylated spirit in and
sterilised his equipment by lighting a flame. He then said that he had some
horse hair but no anaesthetic. He asked my mother to put me on the table so
that he could stitch me up. He told my Dad to hold me down. I can tell you I
squealed throughout the procedure. Dr Bonnelly was thanked while I fought the
nerve-end pains that night. The doctor came back and removed three stitches.
From that day I had a special man and boy relationship with him as I saw him as
a robust Army doctor. It was not a long time from that we heard he had
purchased Venbridge House at Cheriton Bishop and started up as the Cheriton
Bishop doctor. I bet it paid better than goat farming. I would think I must
have been one of the first patients before the formation of his surgery to
stimulate the forming of a very excellent practice.

As time and seasons drifted by I would travel to the area of Forder river with
my home-made fishing rod copied from a fishing journal. I had made copies of
the fish hooks found in a picture book. Sometimes with my trusty Webley air
rifle and sometimes with lifetime birthday or Christmas presents. How strange:
no mobile phone or watch but always home in time for lunch, tea, or bed times.
My trips to the river fishing taught me the dark side of human nature. Whilst
fishing over the summer school holiday I met another local lad, maybe two years
older than I. We became friends and spent morning and afternoon fishing
together, and with great delight for me I discovered that I should be catching
the school bus and attending the same school, Chagford Secondary Modern.
Eventually the school start day arrived and I cycled to Rydons Cross to catch
the bus complete with new found friend. At 8 o'clock the bus ground to a halt.
What happened next was not anything I had ever experienced before or expected.
The school bus door flew open and my new found friend started to shout to the
rest of the kids in the bus ``look, bat's ears, I told you he would get on
here. They look bigger today than last. Welcome bat's ears.'' This onslaught
continued until we reached the school and for the rest of the week. I could not
come to terms with the actions of the lad in those days, particularly when at
the end of the week as I alighted from the bus my tormentor's voice quietly and
expectantly said ``See you down the river tomorrow''? I think I said, ``I do
not think so, as I am a bit busy''. This unhappy event faded with time.

Lessons and events at Chagford School were ideal in my opinion, for the nature
of the region had an agricultural background. They ran a scheme where, after a
pupil had completed a period of time, he or she could elect to join an
agricultural based class which was named Ag1 and Ag2. This represented the last
two years of school. Regular visits were made to our classes by a Mr Bill
Hingston from the Ministry of Agriculture giving us first-class instruction on
thatching and, in particular, stone walling, which was put to good use as we
commenced to construct an open air theatre for the drama groups to use. With
the school located in the harsh environment of Dartmoor you can imagine the
spartan conditions rendered on the parade ground with scanty kit of white
shorts and short-sleeved white shirts, while a thick white frost covered all
the grass and hedges, followed by a stern warning to us to run on only dry
patches of tarmac. Summertime swimming lessons were equally shocking. This
event started with the march from Chagford School to the swimming pool at
Rushford Mill to be greeted with a plain cement pool that was directly fed by
the very fresh water from the River Teign. We rushed from the changing rooms to
the pool and jumping in always resulted in the body feeling like it had been
crushed by ice. We did enjoy it but it took time to respond to the movement.
The school also kept six WBC beehives, and a particular event always brings a
smile to my memory. The opening construction consisted of the six beehives
positioned behind the school block on a corner that looked through the
headmaster's study window. It was deemed one sunny afternoon that we, Class
Ag1, would line up outside the headmaster's window and watch our class teacher,
Bill Warren (a.k.a. Willie) and an assistant carry out what is known in the
beekeeping world as a spring clean. Something my Grandfather, John Preston, had
carried out at home the weekend before, which entails smoking, lifting the
cover and scraping excess beeswax, and replacing items as needed, using a
steady hand so as not to disturbed the bee colony unduly. Unfortunately the
teacher and helper would probably not have done this job before and would not
have had the skill John Preston possessed. This had resulted in agitating the
colony so more bees than normal were flying out of the hive. The children
lining the headmaster's wall began to show concern and movement, at which stage
the Headmaster's voice boomed out with anger ``Stop moving about and watch the
demonstration'' and, to back up his frustration, walked through the door into
the apiary. He had two things against him: he was a heavy smoker and very tall.
Bees do not like stale tobacco breath and a great target, standing at the
height of 6' 2'', our Headmaster was walking around the apiary fast and
agitated with much arm movement. At this stage of play I thought things were
going to turn nasty, and sure enough a high pitched buzzing immediately
commenced, for a number of guard bees, indicating confrontation, started to
escape and attack the headmaster who was leaving the area and running bent
doubled up with his head about 2' off the ground, both hands vigorously beating
his bald head in a vain attempt to repel the four bees which were engaged on a
very successful stinging attack. I could never believe how a body could run so
fast bent double through crops of potatoes, onions, carrots, and leeks without
falling over. He disappeared from view. To keep face and dignity, a rumour
circulated the next day that he spent the rest of the afternoon recovering from
that event in bed! After that incident the bees received a new status from all
those who had been caned throughout the term and there was applause and high
praise, but from the girls a whisper of deep sympathy for the Headmaster.

I must bring attention to Bill (Willie) Warren, our class teacher who had been
tasked with cleaning the bees. In my opinion Willie brought a breath of fresh
air into everyday school activities. Having been demobbed from the war, an
ex-RAF bomber pilot flying sterling bomber planes on night time sorties over
Germany, he brought into the class a more down-to-earth and open attitude which
we all need in everyday life. We had many occasions when he would stop a
science or English class to give us a short talk about his experience as a
bomber pilot. Flying for hours in a freezing plane then flying over a city
wondering who was looking up, perhaps the enemy, normal people, family units
and thinking ``will I survive tonight'' and ``will I be walking about at home
tomorrow'', and so on. In my opinion these short stops never hindered a
sluggish class but rejuvenated the minds. It comes to my mind an event that
took place just before our breakup up for the summer school holidays. Class 1Ag
was involved in an activities afternoon cleaning the open air theatre, with
borders to be dug over and the grass on the pathways cut with a push drum
mower. Having commenced my duty to dig over the boarders it became quite warm
so I removed my coat and worked in shirt sleeves. All things were going well
until Willie Warren arrived on the scene, complete with a hand push roller drum
lawn mower, requesting that we stop work and move away from the lawn edges as
he was going to cut the lawn path. Willie commenced one end with purposefully
long strides whilst I remained standing admiring the even green grass cuttings
cascading into the grass box. Suddenly my admiration turned to horror as the
green turned to brown and woolly. I ran and retrieved my coat and, horror of
horrors, it looked like a hedgehog that had been shorn with sheep shears with
lumps of thread sticking out all over. This coat had been purchased by Mother
two weeks before from Thomas Moore, High Street, Exeter, so homecoming on that
day was not going to be very easy. Willie Warren denied responsibility with the
words ``typical Mike Saffin throws his clothes down all over the place and
never hangs anything up''. I responded, ``If this is the best you can see
pushing a lawnmower, how did your bombs hit anything when you were flying over
Germany?'' The homecoming to Thorn Moor was a more serious event --- me wearing
the coat as if there was nothing wrong with it. Kids cheering and shouting from
the school bus windows, me walking into the kitchen keeping my back to the wall
until Mother discovered the damage and expressed horror at the sight of the
damaged coat. I, for my contribution, expressed immense surprise at the extent
at the time as it appeared minor to me. Mother being resourceful said, ''What's
done is done. The coat can be used winter time to keep out the cold with a
light raincoat over to cover it up''.

The season slid away to the best part of the school year --- the August
six-weeks holiday period and, for me, the beginning of my last year at Chagford
school, and a chance event that changed my outlook and direction in life. It
started one afternoon when I heard a combination of mechanical noises coming
from an adjoining field to ours, a mixture of a corn binder and a type of
tractor which I had never heard before which was certainly not the sluggish
noise of a standard Fordson. Climbing through the hedge when carrying out
further investigation I was greeted with a combine corn binder under tow with a
happy round-faced elderly man wearing a grey pullover sitting and working the
machine controls. Towing the outfit was the most interesting tractor I had ever
set eyes on. In fact a totally home-made unit constructed by the
30-something-years handsome unshaven man wearing a black pork pie trilby type
hat. As the unit got nearer to me he stopped the tractor and offered me a place
to sit up beside him, not just any old iron seat as was the norm but a big
double American leather covered seat. I remained sitting in heaven until the
field was cut. I was so delighted that he offered me a lift back to his
farmstead, Hole Farm, which was on my list of farms easy for me to walk home
from using a direct cross-country route. On arrival at Hole Farm we stopped
outside a very large black corrugated workshop with a cob wall. On the Dartmoor
weather side of the workshop a small door was open and we walked in. The sight
that met my eyes was unbelievable: a wood saw, grinding and welding equipment,
spanners, a band saw, post drills, oil drums, grease guns, part-built farm
trailers in progress, mechanical gear boxes, engines, and various items that I
could only identify after a number of visits. Many items were American and had
been purchased from the Ministry of Defence and converted to run off a Lister
engine type, or from a large engine using some pulleys and belts that enabled
the engine to power more than one item. I must explain that all described was
magic to me, as the only things mechanical to me were wood saws, thatcher's
hooks, and a metal tin with a hinged lid --- the remaining items were reed
willow sticks and wood cut from the copse. A relationship was set up: Mum and
Dad would see me off on my bike, and I would visit Hole Farm and spend time in
the workshop watching or helping with small tasks such as cleaning out ex-army
boxes and tidying items unknown. One very pleasant job was to work or assist
Worthy (the man) to reclaim overgrown fields of birch saplings. This operation
entailed using a Caterpillar R2 TVO burning crawler with a long wire rope
attached to the draw bar. I would reverse the Crawler into the birch and Worthy
would hook up to the small tree and pull it out of the ground clear. We
reclaimed many acres in this manner.

One day an army tank was delivered and Worthy put it to good use: climbing up
the apple trees until they toppled over with the weight, enabling them to be
removed with the roots attached. The area of land would be cleaned, leaving
deep pits that had held the removed root system. A big single furrow plough
called a Paraine buster was used to plough the whole field thus levelling out
the earthy surface for crop plantation. The passing seasons saw many trailers
made, with wonderfully strong frames, and jacking systems attached, and sold to
farmers far and wide. The month of March 1950 took a change when a motorcyclist
rode into the yard, a well travelled man, with a line of small flags
advertising the countries visited and decked with very well made pannier boxes
and motorcycle spare tyres around his shoulders. This was John (Jan), Worthy's
older brother who had left Celon on Boxing Day and ridden his Triumph 500cc
motorcycle to England to be demobbed at Plymouth naval base from his duties as
a Chief Petty Officer --- another story told in a motorcycle magazine in the
1980s. The ambitious man then commenced to build a lorry, square-framed on a
trailer frame. When asked the plan I was told ``I am building a caravan home
for my family and going to position it in the little orchard behind this
workshop'', and in due time it was completed and positioned in the location
desired. His wife Elsie and very young son David took up residence. Life
continued with me keeping a watchful eye on the wonders of John Preston Butts'
thatching and bee-skip-making skills, watching with wonder at the engineering
activities. Life took a different pace when one day a giant low loader arrived
and proceeded to unload a used yellow Caterpillar D7 bulldozer. For Worthy,
this meant the nine months that followed saw new activity in Hole Farm
workshop, as the regular business of trailer making was replaced with
bulldozing hedges for farmers. I remained at home and, as a growing lad of 12
or 13, was tough enough to help with duties at home, such as walking up to Big
Pennypark and, with much pride, using the large heavy hay knife to endeavour to
cut out the hay square for the cows to feed, hoe turnips, cut hedges, and clean
out the cow sheds when needed. Sadly John Preston Butt passed away in his sleep
in April at the age of 90. From then on I never used John Preston's tin shed,
as I felt the shed and its contents were for a past world of natural skills
when everything could be made from nothing, and to introduce any other items
would not be correct. So it remained a shrine and I used my workshop set up in
the still room next to the main house. I shall never forget one Saturday
afternoon when I was helping farmer Burridge at Rydon Farm lead his trusty
carthorse pulling a weed hoe between rows of new growing potatoes, when news
arrived that Worthy Anstee had had a catastrophic accident with the Caterpiller
D7 bulldozer. This fatality changed the lifestyle of many people for some years
to come, just as it had changed my attitude on the day I had rode in to Hole
Farm on the home built tractor. Engineering was going to be my technical
passion, with Caterpillar the leading manufacture company for outstanding
engineering and reliability over the countryside.
