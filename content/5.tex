%!TEX root = ../thorn-moor.tex

Earlier I did mention that I was impressed with the long green
blacksmith/wheelwright shed. The length of this shed was used on a seasonal
basis. The end of April to June would find me as a young apprentice spending
many hours grinding combs and cutters for the farmers and shearers that sheared
sheep on the farms by contract. August would find me at the far end mending
canvas conveyors, a brittle strap of corn binders. The forge area was used all
season as some problems were needed to be solved immediately. Some work could
be listed and kept over for the cold winter days, such as the sharpening of
drag tines. It was on such occasions that I received the lesson of my life. One
day the boss, Mr Saunders, said that as we had saved up several units of drags
to have pointed and sharpened he felt it was an ideal time to complete the job,
particularly as the water work team was not busy and he could let one of the
men free to attend Whiddon Down and help me by turning the forge handle whilst
I carried out the blacksmithing. The man in question was called Bill Allen.
Normally with the water team of plumbers he was a `Man Friday' who carried out
general labouring work such as digging holes and trenches as required. He
tended to be overlooked by his team, probably due to his position with them. We
started the day together as planned --- Bill turning the wind vane handle and I
heating up and sharpening the multitudes of drag tines and rebuilding the
units. With pleasant conversation and a steady work pace we found the day's
labour was nearing an end by 3.30 pm. Suddenly Bill said ``Mike would you like
me to show you how to forge some metal and stone cutting chisels?''. The answer
was of course, ``Yes''. From that moment of time it was magical to watch as
Bill selected some metal from our spare metal stock, selecting particularly old
horse rake tines for the high-quality steel content. I turned the forge handle
while Bill worked his magic in constructing these very fine tools on the anvil.
It transpired in the conversation that followed that afternoon that Bill Allen
had been a blacksmith in the Plymouth dockyard over the war years, carrying out
very skilful work as a blacksmith laying the submarine keels and framework. Due
to the heavy bombing experience at night his nerve had finally broken and he
came out of the dockyard and moved house from Plymouth to Chagford to be in the
quiet of the countryside and chose to work for C J  Saunders in a
non-responsible job. I shall never forget Bill Allen and his quiet kindness to
me in demonstrating his very fine blacksmith skills. So never underestimate the
man that stands before you. He could have a wonderful story to tell. Nature
over power and man-made equipment.

Charlie Endacott and I were charged to investigate and prepare a water turbine
unit located in its turbine house near the River Teign in a valley under Castle
Drogo in the parish of Drewsteignton. Charlie Endacott was a lad 12 months
older than I, and had joined C J Saunders the same time as myself and had
served the term of apprenticeship to the end of his time. So we both must have
been active and experienced lads. I use the term `must have been' as if not it
would have been `up the road'. The water turbine unit was bolted to a long face
plate, bolted with a series of moveable turbine speed controls with a 4-inch
second shaft coming through a gland packing to a sturdy pedestal carrying the
overheating bearing, then attached next to the bearing was a 3-ton flywheel
keyed to the shaft which coupled to a 15kw generator and continued through a
glass-covered governing system manufactured by David Brown, the well-known
tractor manufacturer, to monitor the speed vein and finished with an identical
bearing pedestal. In total the length from the turbine wall face to the outer
pedestal was approximately 9 feet. It was acknowledged that, as this unit had
been running without stopping for year after year, the complete turbine
component should be dismantled and cleaned free of rotting leaves that had
passed through the many water filters and formed a very hard unshapely surface
which would cause a reduction of performance. In respect of the overheating
pedestal bearing, it was noticed that the tiled floor was breaking up and
distorting caused by an external tree root that had been growing for decades
and had expanded and lifted the drive shaft electric alternator and a 3-ton
flywheel out of alignment. One must wonder at the power of nature, as this
needed immense power to lift the total weight of equipment together while
ripping out the securing bolts from the floor. The problems were corrected by
digging a trench from the offending tree under the house foundations to the end
of the expanding root that finished some 2 feet past the lifted bearing block,
refilling with ballast, and re-tiling the floor. The operation was completed by
dismantling the water turbines and laboriously removing the build up of solid
leaf mould. We left the equipment working and from that day I have heard no
adverse report.

In the season following the agricultural world took on an adventurous roll. In
my earlier years I had been involved with the old standard Fordson tractor.
Everything was basic but practical --- multi-plate metal clutch plates that
acted as a clutch and brake combined, white metal Ford main bearings, a conrod
bearing that required the skill of scraping with white metal scraping knife,
and final fitment with skims, gearbox, and transmission filled with SAE 140
gear oil requiring the warming of the 5-gallon can on the workshop stove to
enable the content to commence to flow and normally if Wellworthy cord piston
rings were installed, as was the popular request in those days. With too snug a
fitting of the white metal main and connecting rod bearings and Wellworthy
piston, it was the normal practice to use two men with both hands on a length
of off-cut galvanised water pipe to keep the engine rotating whilst it coughed
and spluttered life into what appeared to be a lifeless lump of cast iron. More
extreme measures were sometimes put into action when a spare tractor in the
yard and was driven in line and a spare threshing belt was connected to each
tractor threshing pulley, pulling the yard tractor or power tractor engine at a
steady slow speed with both threshing pulleys engaged and the main clutch of
the power tractor gingerly engaged. This condition had its advantages, for
should the rebuild start up and then falter the drive tractor would keep up
engine speed and momentum. It must be remembered that very few old Fordson
tractors would have a temperature gauge so working temperature and overheating
was total guesswork, and if you had just rebuilt the engine yourself, perhaps
it was not unusual practice to fill the radiator, run the engine at a third
speed, then install the filler water hose in the radiator filler, undo the
water drain and adjust the filler hose to match the drain flow. This practice
is not compatible with today's engineering and technology, but there twas! It
got the job done.

Some workshop practices were trial and error and would leave one half like a 007
drink: shaken \emph{and} stirred. One particular incident came to mind which
left me in this condition. Geoff the main workshop mechanic was attempting to
carry out a repeat soldering repair to a Fordson Major fuel tank of twin fuels,
one main TVO tank, and one small petrol tank combined. I was working on another
repair in the same little workshop when Mr Saunders came through and, in his
endeavours to help, took the blow torch from Geoff's hands and said with
authority, at the same time turning up the heat of the torch, ``We need to get
this solder to run on the very hot parent metal''. Suddenly there was a massive
boom as our working area became filled with a swirling yellow flame. I stood in
shock but was able to see two figures running through the open door with their
heads about 2 feet off the ground. Their body actions mirrored the escaping
method of a headmaster running from angry bees mentioned earlier in this story.
I grabbed and extinguished the burning wreckage to just steaming metal. Mr
Saunders and Geoff gingerly looked through the door expecting me to look like
an old burnt out church candle. Thankfully all was well. I did feel shaky and
my hair had been singed. Something good that had revealed itself after this
dangerous accident was the knowledge that the internal construction of the fuel
tank was not like our imagination had envisaged. To correct this type of fault
would result in a radical change of approach: extra labour needed and extra
steam cleaning for safety. Not to say that the repair was impossible, but to
accomplish a guaranteed satisfactory repair, a very high cost of preparation
would result in a heavy repair cost to an old fuel tank.
