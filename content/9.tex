%!TEX root = ../thorn-moor.tex

On another occasion whilst with Caterpillar, I had worked continually without a
day's break, and experienced a desire to be freed up for the weekend. So you
can imagine the anticipated build up of the break was joyous and by 5 o'clock
on that Friday afternoon I jumped into the van and was about to insert the
ignition key when the field Service office door burst open and the supervisor
Joe burst through with a very serious red face and said, `` We have a
problem --- a new power boat has been delivered to St Hellier, Jersey this
morning and whilst demonstrating its manoeuvrability to the good and the great
in the harbour had suffered a major marine gear failure (and yes you've got it
in one) it is so late that we are unable to book you a flight out of Exeter so
you are booked to fly out of Southampton first flight in the morning. So it's
an early start next morning not a lie in but up with the blackbirds to
Southampton. My brief received between Jersey Airport and St Hellier was
equally intense. Plans were in position for a birthday party celebrating the
arrival of a new Fairy boat. All Fairy boat owners had been invited to assemble
with their boats to the inner harbour of St Hellier on Saturday morning with
mooring drinks at the Yacht Club and when refreshed to set off in a group to St
Malo in Brittany for the weekend celebration. In fact at this moment in time
the party was assembling and was advised that there had been a slight hitch and
that the Cat man would be arriving from England in the morning and should
correct the problem. Then all will be go! Using todays term no pressure then.
When I got to the waterside and observed the collection of power boats moored
alongside in anticipation of the forthcoming jolly together with the owners and
ladies one realised how far the original event had travelled before reality had
set in. Particularly when the South Pier manager, Gerald, walked over looking
ashen and said, ``I hope to God you can pull this out of the bag''. We shook
hands and I went below to examine he bowels of the problem. It was quickly
apparent that this was no quick fix as a clutch plate had shattered and
centrafugaly spread and jammed the drive astern gear solution, remove marine
gear and needed to be returned to Exeter Depot for rebuild. The news was
received with much disappointment and fury all directed at the manager.
Fortunately it was decided to continue with the party as all were gathered and
full arrangements had been made. I continued to remove the marine gear for
transportation greatly assisted by the South Pier shipyard which had a well
organised team. The weather over that period was exceptionally hot and sunny. I
unwisely decided to work shirtless so by the afternoon a bottle of calamine
lotion was purchased and used to cool my very hot back. By Monday the gear was
despatched via a boat motoring to harbour. Due to local fog the airport was
shut down so I was taken to the harbour and located a coaster that was on a
passage to Southampton that night. So I signed on as a supernumery with a
Captain Edwards as a non-working crew member of MV The Loon Fisher. I spent
part of the night sleeping in the sick bay and part admiring the fortitude of
the Bellman standing out in the bows dressed in storm clothing and covered in
damp fog particularly constantly peering into the mist of impending danger. We
docked at Southampton and I must have looked odd walking alone between the
shipping containers carrying a very heavy holdall looking for an exit. I was
quickly arrested by a Customs officer and taken for questioning and a strip
search. A degree of satisfaction spread over me when I was standing in pants
only whilst the Customs officer passed around the bottle of calamine lotion for
tasting and judgement proved satisfactory and I returned with the marine gear
that was rebuilt and taken back to St Hellier and refitted without further
dramas. Life continued with repairs and warranty work as required with normal
six weekly marine work continuing in Jersey. I was then called to another fast
patrol boat, a 45' Keith Nelson hull which was used for river police and a
pilot boat. I was engaged on a top overhaul to the engine, removing cleaning,
replacing and sealing air intake inter coolers. One turbocharger needed
rebuilding so an exchange unit was flown over. This tale is not about
engineering or technology. It is about things that unexpectedly happen that
cause disruption, I think you must stop and take time to think how do I get
this back to the norm with minimum action. Taking up the story the replacement
turbocharger was flown out and delivered to South Pier shipyard. Douglas Park,
a skipper to the Duchess of Normandy, an area patrol vessel, agreed to deliver
the unit to me by using his harbour runabout. To recap the scene, all boats in
the harbour were normally afloat and went aground at low tide of 2-3 hours. To
set the scene it was a beautiful afternoon with a spring tide washing up to the
top of the tide at approximately 7 o'clock with sun shining across the water
surface. Douglas threw a rope across to me with the words, ``Lift on that one
when ready as I have tied the package in a tight rope parcel until the package
was midway between the boats and open water. Suddenly the rope went slack
followed by a glug with a multitude of bubbles rising to the surface. A stunned
silence between us followed by me asking the question, well knowing the answer,
but it gave me time to recover. The response was 2 am tomorrow morning would
see the tide recede down enough for me to walk out in the harbour and retrieve
the package. So that evening I had a very uneasy dinner at Mrs Cornish's and
did not change out of working kit but walked across to the Folly and had a
couple of drinks with the boys and then went back to my room and watched the
clock. At 1 am I put on a sturdy evening coat and walked out into the night and
made my uncertain way down into the dark harbour. You could imagine every
shadow that bounced off the dark pillars in the moonlight was a demon ready to
cause me harm. So I quietly made my way to a pillar and stood very quietly in
the shadows keeping my eye on the surface of the harbour floor and quickly
walked out and retrieved the package putting it in my sack and walked back to
the Guest House. I have wondered how I would have responded to the questions
which might have been asked by a patrolling policeman. A lone man walking
through the streets of St Hellier carrying in both hands a valuable
turbocharger from power boats in the middle of the night I reason I could have
spent the night in a cell until the truth was confirmed. However next day the
unit was dismantled, washed of seawater and relubricated and installed to
continue to work as required. Between my marine work I was attending
the motorway section of the M5 and M4 carrying out maintenance as required
working in that location a week at a time from Banwell to Edith Mead. If parts
were required we collected it in the early morning from our Melksham Depot. It
was noted that in the warmer summer week the flies and mosquitoes multiplied
from the wet reed beds along the flat marshy land caused much stomach problems
among the machine operators working in that area. At that time I was called to
fly out to Malta to correct lack of boost pressure in a turbocharger with
intercooler in use in the boat which had limped down the Adriatic to arrive in
Silema Harbour. It was an 80' yacht with a beautiful name, The Phyllis Serena.
I went aboard this wonderful pristine boat into the bowels and opened a round
hatch into the black hole of the engine to be greeted by a noise like jingle
bells. I found myself standing knee deep in a mixture of empty one pint size
oil tins and pull off slips. It became apparent that no maintenance or kindness
was ever given to this engine. However the unit required an oil top up. Someone
would find a car garage and purchase an armful of one pint oil (any oil) and
return to the engine and fill as required and throw the empty can together with
tin foil slip on to the engine room floor and walk away. Sadly this state of
affairs had caused the internal components to work in mud so the turbocharger
together with the intercooler was removed and taken to a workshop and I
dismantled and rebuilt after a thorough degumming. The engine sump was removed
and the mud like black oil scraped out and the engine rebuilt. To think it cost
a lot of money to send me from Exeter to Malta to correct this problem and
someone had been employed as an engineer to this boat yet maintenance never
existed over many years. Let me tell you about the workshop that had kindly
been offered for me to use together with all the tools needed. I was driven
then by the boson, Charlie Greengrass in a Mini Moke. My first impression was
of quite a large galvanised corrugated shed with the roof completely covered in
live and growing course green grass. When I opened the door and walked in I saw
the green grass had caused the roof to rust through. When it rained the rain
would run through the funnel shaped hole and hit the centre of the workshop
floor. Modification had been made to the system. A hammer drill had been used
to cut a gutter from the leaking water across the floor under the door and
across the outside pavement and in to the street. The workshop was empty and
clear but around the wall hung new cabinets containing thousands and thousands
of pounds worth of USA snap-on tools not used and brand new. The whole set up
was so extreme you could not even imagine such a thing could exist. Whilst I
worked in the workshop I kept the door that opened on to the street wide open
whilst well loved children played happily with sheer delight. Well loved but
wearing thread bare clothes that were washed clean and well darned and with
patches where the pattern failed to match anything indicating that it was cloth
cut form worn out material ready for the trash bin. Not once did I receive a
visit from any one ownership or other. The boat was reassembled and performed
satisfactorily under test and returned to service, probably having to contend
with the old habit. Work was so varied that having flown back from Malta I was
called to attend a CAT D8 using a Kelly ripper working on the face of the
Meldon Dam wall. The right hand track had parted under a severe load and had
whiplashed and landed upside down on the track frame. So it looked an
impossible repair location so with a heavy heart and with help from people on
site we lowered the arc welder together with a track pin repair kit and
hydraulic piece kit to rest on the bulldozer blade in the Dam wall. Meldon Dam,
Okehampton: from the start to completion of the very very difficult situation
just everything went right for me. The track was lying upside down on top of
the track frame. I pulled it with a crow bar. It fell off and landed the
correct side up and down hill sideways ending up positioned correctly under the
track rollers. We took time on every move because of the dangerous position we
were located in and at each stage everything fell in place so by 2.30 pm four
hours after commencement the machine was returning to work with a promise by
the operator driver to drive up the less steep pathway and to use the Kelly
ripper to rip out going down only. I walked back to my van to write a report
shaking my head in disbelief. If anyone does not believe this story, get in
your car, drive to Meldon Dam near Okehampton, park up and walk to the dam
wall, keyed into the valley A30 road side and look down and try to imagine a
large machine one third of the way up with one track off. I have not been to
Meldon for some time but the last time I walked away in wonderment of the
achievement of those past times. There were no Health and Safety rules then or
hard hats worn.
