%!TEX root = ../thorn-moor.tex

The blizzard of 1963 started just after 12 o'clock on the morning of Boxing Day.
I had milked the cows and all was in order, as I thought it was just another
flurry of snow. Mother and father were stocked up with food and normal warmth.
Dad was recovering from an operation at this time. I then drove to Brampford
Speke to visit Vera for lunch. As I was visiting the love of my life, lunch
went into tea time, and at 9.30pm with no regard to the weather I looked out of
the window and dressed for the journey home. You can imagine the horror and
deep feeling of irresponsibility when I saw the extreme blizzard conditions
knowing that my mother was at home with my unwell father in a very isolated
house and me 15 miles away. At that time I was driving a Ford 1 Courtina with a
three speed column gear change and small wheels which were most unsuitable for
traction! However, I started to drive with a passion knowing that I must get
home to my parents. So I drove zig-zagging through snow drifts while keeping
the speed up to power through the drifts. People who have driven through snow
storms have experienced the feeling of uncertainty when driving uphill, along
the flat, or downhill. I experienced this ungodly sensation. When I reached the
top of Burridge Hill the engine gained speed and I identified that I had gained
the ascent and was descending far too fast to negotiate the T junction 100
yards ahead, so I did my best to reduce speed, well knowing that I was going to
contact the hedge, which I did. Fortunately only the front wheel ran up the
bank leaving the car sitting with the front wheel sitting in the hedge. So,
knowing the rear wheels would not grip, I pulled my raincoat out of the boot of
the car and placed it under the rear wheels, jumped back into the car, shut the
engine down, put the gear into reverse and pressed the start button and the car
wound its way back off --- a very handy trick that modern cars are unable to
do. I drove off leaving my raincoat in the road as the weather was too harsh to
linger. I drove continually zig-zagging through small snowdrifts knowing that
after Tedburn St Mary conditions would become much worse. Fortunately my
judgement could not have been further from the truth since, in reality, as the
A30 became more exposed to the strong winds tearing out of Dartmoor, the snow
was being blown over the hedges, leaving me a reasonably clear run until I
turned off onto the Yeoford road. The reality of the blizzard turned my
progress belly up. The wind had blown the snow from the open area of Red Ridges
over the hedge to cause a massive 11' high snow drift across the road into the
entrance to Glebelands Estate. Just down the nap from the Village Hall
everything was a white out and this obstacle could not be seen, so I went into
it at a fairly steady speed and I think the car climbed into the drift at about
8' high, first a rapid deceleration, then to a complete stop, and also in
complete darkness as the headlights were buried in the mass of snow. The door
could not be opened as we were wedged in the drift so I wound down the window
and climbed out on to the snow. The next big shock was when I sunk deeper down
in the snow and, with some degree of panic, I detached myself from the
snowdrift leaving my car to rest with windows open and with only the roof and
windows visible (but to be covered later with the falling snow). I switched on
my trusty Pifco lamp and set out to walk home. After I had walked, sometimes
through knee deep snow, I became aware of a clanking noise around ears, so,
feeling my head, I was then aware that the snowflakes landing on my warm head
had melted and, whilst running down my hair, had frozen in the cold harsh wind.
How unwise not to be wearing a hat. On arriving home I was relieved to hear
both my parents' calls who responded with reassurances that I was home. Next
morning found the snow level with the windowsills, and it was still falling
heavily. It was obvious to me that Thorn Moor, Little Moor Farm, and other
isolated farms were cut off. Whilst feeding the cattle I reflected on my very
lucky and dodgy drive back the previous night. I thought what a stupid selfish
bull-headed individual I was, well used to knowing the weather in store with a
responsibility for a sick father and a very anxious mother and with animal
feeding and milking to be done. However I settled down to the running of Thorn
Moor under blizzard conditions. Fortunately it was a holiday break so work was
out of the question. I cannot remember having a day when it didn't snow but it
eventually stopped. I then walked to Checkers shop at Cheriton Bishop for
provisions, and rather than return on the strength-sapping powdery snow I
decided to walk to Crockernwell via the A30 and Hooperton Cross, which was much
longer and involved an extra walk of about two miles in blizzard conditions
through knee-high snow. The worst was trying to breathe in the corridor of very
fine particles of broken-down cold snow that had filtered through. I set up my
very unsuitable car for snow use, adjusted the rear tyres to half normal
pressure and front tyres a little higher than standard. I wrapped a rope around
the rear drive wheels to connect blocks in the boot. I gave the farmer living
at Little Thorn Farm £1. He then towed me through the heavy snow to the stone
landing. My Ford was left unlocked because of the extra low temperature --- a
lock would never open with the boot facing Dartmoor's extreme weather. To keep
the engine clear of snow blown by the wind each morning I would walk from Thorn
Moor then drive to work, so never was a days' work lost, as we had plenty of
water pump air vessels to cast weld together with a couple of Allis Chalmers
engine cast blocks to also weld. After some days I was aware that the road had
hard packs of snow so the car was driven home. After some more days it was
noticed that I could see over the hedges and walls, so the thickness of the
packed ice was approximately one inch deep and I was very aware that the thaw
would cause problems when the surface broke up. So for some days over the thaw
the car was left at Hooperton Cross and I walked to and from it each day until
the normal surface returned.

One spring evening in 1965 a very pleasant man walked into my garden and said
``Evening Mike. I would like you to help with two or three others to form a
pistol club. What do you think?''. I told him that I was not sure about
revolvers and fast draw seems to be on film sets, but when he produced a
collection of semi-automatic pistols for precision marksmanship my attitude
changed completely. So Wednesday evening found me joining a small group at
Jim's house to discuss the future of the group that formed the Exeter Pistol
Club: Ken Chard, the owner of the gun shop located at Exe Bridge, Exeter; Jim
Austin, whose home we used for Committee Meetings; Ernie Hart, who had
knowledge of competitive shooting, who was the controlling body of the NSRA;
and yours truly, who had nothing to offer but listen to all that was said. We
agreed to shoot at Wyvern Barracks on Topsham Road twice a week and join the
NSRA postal shooting competition. I was given a target so I manufactured wooden
target holders. We each put money in the kitty and purchased two Russian
Volslocks as Club pistols. I sold my father's 12 bore in exchange for a Smith
and Wesson 41. All administrative and other items were in place ready for the
first evening with the Pistol Shooting Club at Wyvern Barracks. So with targets
and new holders 25 yards down the range, tables across the range, and a Range
Officer wearing an arm band, the four of us lined up. Having listened to the
instructions of Ken Chard and Erny Hart and having not taken arm muscle
exercise, I commenced to fire off my clip of 10 bullets. It would have been a
praise were we called awful. Not one shot hit the paper target but the bits of
wood flying off the target holders gave a good indication of our lack of
ability. At the end of one hour of shooting the holders were just hanging
together and looked like they had been dipped in a tank of sharks. We all
slowly improved and, with arm exercises, together with the more we practised,
Jim decided that deepened experience was needed and so booked us into Bisley
Pistol Championships. On arrival, Jim attended the Office Administrator and, in
collecting our target numbers and shooting times, he discovered Devon had not
entered a team in the County Championships. When he returned to our camp hut,
our lodgings for the night, he said, ``My men, I have entered the four of us in
the Devon Pistol team''. I said, ``What! You must be off your head. We are only
learning to be marksmen. We will never level with the worst''. The return
comment was that we would be OK and that it is the experience that counts. With
a good night's sleep under our belts we set forth into the marked areas at
Bisley carrying out our own shooting in various ranges. Our individual results:
my Club average was not disappointing I was told, particularly under
competition pressure. Jim went to the official marking office for the results
of the inter County Championships. We, as expected, did not do well. As Jim
returned, not sure of our position, he told us that it was very difficult to
find the results. Years passed by and we all improved and were shooting in
Division 2. I entered a postal elimination shoot at the Eley Olympic 1980
competition and ended the postal round in Bisley 60 finalists. In August, Vera
and I went to Bisley and, to my surprise, I finished third. The cut-glass bowl
stands in pride of place today. As I was beginning to help charity
organisations I decided to move on from pistol shooting, remembering the great
fun and good time over the years. I purchased and gave the Club a cup to be
presented each year to the most improved shot in the last 12 months, but I
never heard anything back so they must run without a Chairman or Secretary. I
believe they are shooting somewhere in Pinhoe and I hope they have same
enjoyment and sense of achievement that we had all those years ago.