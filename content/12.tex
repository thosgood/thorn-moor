%!TEX root = ../thorn-moor.tex

When I retrace my memory back to my age of 10 years I become aware of how much
we as individuals manufactured for ourselves. It started by me sitting on a log
in the orchard watching Dad make me a whistle from a willow branch using his
penknife. A few days later found me sitting on the same log making a whistle
with my penknife --- every country boy would own a penknife which was probably
given as a Christmas present or found in a box of trinkets. As time went on I
wanted a bow and arrow so I made myself a good one in my opinion. Then a
catapult winter arrived with snow, and Mother's big dinner tray was ditched as
I made a good toboggan copied from a winter Christmas card.

Going back to my grandfather John Preston Butt, known to all as Jack Butt, he
had talents which were greatly admired by me as a small boy. Naturally my
awareness expanded as the years went by and I immediately acknowledged his
skill as a thatcher and his ability in making roof spars, straw ropes, and bee
skips. Then one morning I was aware that a letter addressed to Jack Butt had
arrived by postman Mr Hawkins (Smiler). Mother read the contents to Grandfather
whilst he sat on the Devon settle listening with great concentration --- we
must remember that he had limited writing skills and even less reading
abilities. I recall Grandfather saying ``I was taught how to read and write by
a man who would say if I stopped (because I did, not knowing the word)
\emph{read on boy} thinking thicky man's daid'' (dead). Irrespective of this
limited tuition he was able to write on a slate with a long carpenter's pencil
the customers' names and numbers of spars made and bundled for them, and total
up the pounds shillings and pence. Following on from the letter reading I
found Jack Butt dressed in his second best waistcoat and carrying his bowler
hat and three or four newly cut hazel sticks waiting for a taxi. Naturally
after his departure I made enquiries to be told that he had gone to do some
water divining for someone who wished to dig a new well. It was some years
later that he showed me the technique telling me that only I would know when I
was ready. I have never committed myself to divine water for anyone as in my
opinion it could cost money. For my own peace of mind I have divined and found
a healthy flow of water running under Cowick Street shopping centre car park
in Exeter --- maybe a big drainage. I was also asked if I would confirm the
finding of two professional diviners who had reported the finding of a large
spring flow of water across the road on a driveway between Newton Abbot and
Penn Inn. I attended and to my horror and disappointment could not get any
reaction from my hazel sticks but walking 50 yards away a great reaction
against a stone wall. I reported my finding and disappointment to have lost
the little gift I had! So I got further help. Approximately a month later I
received a phone call from the land owner saying that I should not be
disappointed as they had carried out drilling probes in the drive which
confirmed that I was correct that there was no water under the driveway but
they had found water 50 yards away, against the wall which was once used as a
communal water tap from a well, so I felt very pleased. In general, a letter
would arrive giving a communication for a date at Grandfather's workshop for a
taxi car to arrive, and the visitor going into the shed with the door tightly
closed for privacy time. After some years it was noted that a few small
parcels addressed to Jack Butt and tied up with string with the knots sealed
with sealing wax would be delivered and opened to reveal an ounce of Digger
shag tobacco or may be two. Grandfather would never give away anything about
charming warts. I was then told that it must be passed from male to female so
Mother received the information from Grandfather but did not feel happy to use
it so passed it to me. I charmed many warts over the years but did not sell
the gift, although I have received great pleasure when thanked after the warts
disappeared. Folks might laugh and doubt, but if they get a wart on their nose
they might try, if it won't go, to see me.

William James Saffin, my Great Grandfather, lived at Poleford between
Crockernwell and Cheriton Bishop. He was a stone cracker, as in his early days
the A30 road was not covered in tarmacadam (it was probably not invented then)
but was a hard earth surface with the covering of duck egg sized stones that
would gradually break down to small granulated rubber under the iron rim of
various stage coaches, farm wagons, and carts. In fact, Great Grandfather
remembered a special occasion when a man on horseback rode through with one of
the first motor cars, which drove through with a red flag bearing the words of
a warning of mortal danger and ``beware'' hanging on its side in the form of a
notice. This was the first motor car to drive through the A30 road. Dad said
there was one story Great Granddad told in solemn mood, which was the appalling
attitude of the aristocracy towards the working man experienced by him. The
story goes that Great Grandfather had finished work for the day and started to
walk home. After a time of walking he observed a gentleman smartly dressed and
riding a horse. As he drew nearer my Great Grandfather gave a happy smile. His
greeting was received with a very cold, stony expression and he was taken
completely by surprise when this individual gave him a full and hefty lash with
his horsewhip across his back with a force draining all his strength, causing
him to fall on his hands and knees. Fortunately he had been an active working
man and was in fairly good rugged bodily condition. His recovery was quite
quick and as he started to rise to his feet he selected a good size ragged
stone. The horseman had lingered to watch my Great Grandfather's plight while
he was lying among the dust and stones, and turned away to continue his
journey. My Great Grandfather threw his stone as hard as he could, hitting the
back of the horseman's head. He fell off his horse with one foot tangled in the
stirrup. He regained balance, remounted and kicked the horse to continue his
journey. There was never a word spoken about the savage incident between my
great grandfather and the gentleman horse rider. I have often thought of the
unsavoury incident and wondered what blackness must have been in the heart of
the rider --- the only two men in a lonely country road, one giving a smile to
then receive a whip lash with no other people present to witness the violence.

Another character: dear old Farmer Palmer, as seen through the eyes of a young
boy of my age, was short, portly, and so innocent that the reality of life must
have mercifully passed him by. He passed on his various encounters to second
and third parties which were received with goodwill and mirth. Having seen the
special tractor that Worthy Anstee had built for himself, Farmer Palmer
requested that Worthy build a similar unit for him, which was duly done and the
day came for a test and customer approval, which was received. The tractor was
delivered and with Worthy in the driving seat Farmer Palmer was pleased sitting
beside him. The tractor was driven at top speed to show off its capabilities.
The speed frightened Farmer Palmer who, not wishing to offend, exclaimed ``slow
down, tis fast enough for cutting corn.'' In the early days of the bombing of
Exeter, Farmer Palmer stopped his car beside our main gate. After the normal
greeting to my father he said, ``Quite a heavy night of bombs. I said to my
wife hail up. You yerd the Germans are coming''. So we lifted our main sheets
over heads and settled down underneath. The local garage had an urgent call
from Farmer Palmer saying ``Will you come quick as our old car is alive. Every
time me son-in-law, Jack, turns the starting handle to drive the car it keeps
walking towards him and has pinned him up against the wall so would you come
and rescue him''.

I did say the dog fights in our area were the first indication of war to us but
as time developed the civilian nation became drawn in, adding to my interest as
a small boy because my Dad joined the Cheriton Bishop Home Guard platoon. I
remember Dad returning home with an arm band with black letters L-D-V printed
on it. When I asked the meaning of the letters Dad told me with laughter in his
voice that it meant look, duck, and vanish. As time went on, equipment arrived,
such as a uniform and rifle, and this continued expanding by each visit from a
Sergeant Lock. He brought Sten guns and magazines with bullets that needed to
be installed in the empty magazine cases. When fully operational my father
would go off on Sunday morning exercises and, by stories relayed at our dining
table with mirth, we could follow the Dad's Army style television tales. An
example is that a Sunday morning would finish at 1pm when father returned to
have a late lunch and normally sleep in the chair in preparation for Monday. A
car would draw up about 6pm and a very flustered and red faced Sergeant Lock
would ask if we had seen the man who was put on guard duty at the waterworks.
Father would reply that he had not. The guard had been on duty all day because
he had misheard at the briefing. One farmer in particular would let the side
down. For example, under manoeuvres a small number of the team were silently
creeping up beside a hedge with the intention of a surprise attack. One very
enthusiastic farmer would break from the cover of the bushes and rush into the
field and exclaim loudly, ``Look at this wonderful field of growing early
potatoes'' thus wrecking any chance of a surprise attack.

I recall going to an evening dance in a marquee at the village of Crockernwell
when I was 12 or 14 years old. I was accompanied by my parents who sat and
chatted all evening with a Mr and Mrs Herbert Gillard, and as they had their
15-year old son, George, and their very young daughter, Sheila, with them, it
was natural that George and I should pal up for the evening, as we did, and
built up a friendship that has lasted a lifetime, even though in the past years
we rarely visited each other except to attend each other's 50th wedding
anniversary parties, playing our original roles of Best Men. Reverting to our
relationship, when we were young bucks cementing our friendship by swearing
allegiance to the following we agreed not to steal each other's girlfriends. A
pledge kept but sometimes regretted, it was even made depending on the quality
of the other's latest girlfriend. We also pledged to be each other's Best Men
if we ever became married. Both pledges were actioned 50 years later by both of
us. The first deal that transpired between us involved me purchasing an orange
all-spare-parts four-speed Sturmey Archer bicycle for £4. This machine was
ridden everywhere as it was far better than my discarded single-speed black
Halfords bicycle. We then started analysing our success with girls. In our
opinion, the lack of memory, bearing in mind we were only 16 or 17 and green as
grass, George decided it was the wearing of spectacles that held the secret and
promptly ordered a set of contact lenses. He duly came home with them having
waited for some weeks for delivery. My impression of them was nothing short of
horror. They were the size of the eyeball made of glass that had to be filled
with a small amount of fluid which was supplied to keep the minute gap between
the eyeball and glass dirt free. These lenses remained a secret between George
and I. George's story to all was that his eyes had improved! However on one
particular hot night at a dance at Chagford, events caused my friend and myself
panic and disruption. It was about 10.45pm when the dance floor was packed and
body heat of all was at a maximum that I became aware of a group of people
looking worried advancing towards me. This was the start of a minor panic as
voices said in unison ``George looks very ill''. His eyes had glazed over and,
dancing close to George, I immediately realised what was happening. The heat of
the dance hall had caused the film of fluid to break from between the eye and
the contact lenses and condensation had covered the inner glass surface. Both
eyes were completely grey and looking like the television production of the Old
Wise Man who commenced with grasshopper before giving advice to the Kung Fu
hero. I was aware that our mutual secret could be exposed so I said loudly
``Leave George to me. I will sort him out in the cloakroom''. This was greeted
with relief from all as dancing was the joy for all and they returned to the
dance floor. The contact lenses were cleaned and we returned to continue
dancing. Sunday afternoon was always the time to analyse the problems
experienced on the dance nights, particularly with regard to relationships with
girls, or lack of them --- a unanimous decision was then made to ditch the use
of the contact lenses, put them away in a drawer and revert to spectacle use in
George's case and act normally.

Working back in time, one morning Mr Saunders asked me to take a look at a large
granite roller with a broken main pin with the following remark ``this is a
repair that was executed in the dark ages. I have never carried out such a
repair myself but I know how so I will give you step by step moves.'' First
step, light the blow torch and melt the retaining lead until all clear, gently
pull out the broken shaft together with the many retaining wedges until the
stone hole is clean. Second step: cut a piece of 3/4''round bar and put one end
in the forge and smith a mushroom square, put shaft in stone hole and pack the
sides with 4 mushroom metal 5'' pieces. Third step: manufacture by forge many
wedges thin enough to curl around the larger metal to cause and complete lock
up. It is very important that you listen to the sound of the hammer to stop
high pressure causing severely strong damage. Fourth step: Dig a hole in the
ground to take half the length of the stone roller. Final step: Heat the metal
pin together with retaining wedges hot enough to be tinned with the lead that
is heated to liquid in a pouring ladle (if the lead spits when pouring, heat up
the pin more as it is not hot enough to pour the lead. Keep on until it looks
like liquid. Leave to cool for many hours and return to the customer.
One day after lunch Mr Saunders asked me to get in the car as we were visiting
the water gang working in a very isolated farmstead. We were to look at
designing a small cattle grid. We arrived and proceeded to walk across 2 fields
to locate the men and the mole ploughing exercise. Suddenly Mr Saunders barked
(quote) ``What is the matter with our lorry driver, Arthur Lock.'' I looked in
the direction of his pointed finger and observed Arthur standing as high as
possible lifting his cap up and down on the top of his head and looking in the
direction Arthur was facing, a chain reaction set in. Other members of the team
commenced to repeat the signal. I of course said I had no idea, well knowing it
was the team's warning sign of authority approaching. We continued the visit
taking measurements of the grid. Nothing was said about the incident but I am
sure the boss had worked out the sign because when we returned and opened the
doors of the Triumph Herald to sit in John Saunders removed his cap and
repeated the signal whilst starting the car, a slight smile on his face and
nothing more was said.
