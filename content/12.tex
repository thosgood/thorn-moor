%!TEX root = ../thorn-moor.tex

Dear old Farmer Palmer, as seen through the eyes of a young boy of my age, was
short, portly, and so innocent that the reality of life must have mercifully
passed him by. He passed on his various encounters to second and third parties
which were received with goodwill and mirth. Having seen the special tractor
that Worthy Anstee had built for himself, Farmer Palmer requested that Worthy
build a similar unit for him which was duly done and the day came for a test
and customer approval which was received. The tractor was duly delivered and
with Worthy in the driving seat Farmer Palmer was pleased sitting beside him.
The tractor was driven at top speed to show off its capabilities. The speed
frightened Farmer Palmer who not wishing to offend exclaimed ``slow down. Tis
fast enough for cutting corn.'' In the early days of the bombing of Exeter
Farmer Palmer stopped his car beside our main gate. After the normal greeting
to my father he said, ``Quite a heavy night of bombs''. ``I said to my wife
hail up. You yerd the Germans are coming''. So we lifted our main sheets over
heads and settled down underneath. The local garage had an urgent call from
Farmer Palmer saying ``Will you come quick as our old car is alive. Every time
me son-in-law, Jack, turns the starting handle to drive the car it keeps
walking towards him and has pinned him up against the wall so would you come
and rescue him''.

When I retrace my memory back to my age of 10 years I become aware of how much
we as individuals manufactured for ourselves. It started by me sitting on a log
in the orchard watching Dad make me a whistle from a willow branch using his
penknife. A few days later found me sitting on the same log making a whistle
with my penknife --- every country boy would own a penknife which was probably
given as a Christmas present or found in a box of trinkets. As time went on I
wanted a bow and arrow so I made myself a good one in my opinion. Then a
catapult winter arrived with snow. Mother's big dinner tray was ditched as I
made a good toboggan copied from a winter Christmas card.

I did say the dog fights in our area were the first indication of war to us but
as time developed the civilian nation became drawn in to add to my interest as
a small boy because my Dad joined the Cheriton Bishop Home Guard platoon. I
remember Dad returning home with an arm band with black letters L-D-V printed
on it. When I asked the meaning of the letters Dad told me with laughter in his
voice that it meant look, duck and vanish. As time went on equipment arrived
such as a uniform, a rifle and this continued expanding by each visit by a
Sergeant Lock. He brought Sten guns and magazines with bullets that needed to
be installed in the empty magazine cases. When fully operational my father
would go off on Sunday morning exercises and by stories relayed at our dining
table, with mirth, we could follow the Dad's Army television tales. An example
is that a Sunday morning would finish at 1pm when father returned to have a
late lunch and normally sleep in the chair in preparation for Monday. A car
would draw up about 6pm and a very flustered and red faced Sergeant Lock would
ask if we had seen the man who was put on guard duty at the waterworks. Father
would reply that he had not. The guard had been on duty all day because he had
misheard at the briefing. One farmer in particular would let the side down. For
example, under manoeuvers a small number of the team were silently creeping up
beside a hedge with the intention of a surprise attack. One very enthusiastic
farmer would break from the cover of the bushes and rush into the field and
exclaim loudly, ``Look at this wonderful field of growing early potatoes'' thus
wrecking any chance of a surprise attack. I recall going to an evening dance in
a marquee at the village of Crockernwell when I was 12 or 14 years old. I was
accompanied by my parents who sat and chatted all evening with a Mr and Mrs
Herbert Gillard and as they had their 15 year old son, George, and their very
young daughter, Sheila, with them it was natural that George, and I should pal
up for the evening as we did and built up a friendship that has lasted a
lifetime even though in the past years we rarely visited each other except to
attend each other's 50th wedding anniversary parties playing our original roles
of Best Men. Reverting to our relationship, when we were young bucks cementing
our friendship by swearing allegiance to the following we agreed not to steal
each other's girlfriends. A pledge kept but sometimes regretted it was even
made depending on the quality of the other's latest girlfriend. We also pledged
to be each other's Best Men if we ever became married. Both pledges were
actioned 50 years later by both of us. The first deal that transpired between
us involved me purchasing an orange all spare parts four speed Sturmey Archer
bicycle for £4. This machine was ridden everywhere as it was far better than my
discarded single speed black Halfords bicycle. We then started analysing our
success with girls. In our opinion the lack of memory, bearing in mind we were
only 16 or 17 and green as grass George decided it was the wearing of
spectacles that held the secret and promptly ordered a set of contact lenses.
He duly came home with them having waited for some weeks for delivery. My
impression of them was nothing short of horror. They were the size of the
eyeball made of glass that had to be filled with a small amount of fluid which
was supplied to keep the minute gap between the eyeball and glass dirt free.
These lenses remained a secret between George and I. George's story to all was
that his eyes had improved! However on one particular hot night at a dance at
Chagford events caused my friend and myself panic and disruption. It was about
10.45pm when the dance floor was packed and body heat of all was at a maximum
that I became aware of a group of people looking worried advancing towards me.
This was the start of a minor panic as voices said in unison ``George looks
very ill as his yes had glazed over and dancing close to George I immediately
realised what was happening. The heat of the dance hall had caused the film of
fluid to break from between the eye and the contact lenses and condensation had
covered the inner glass surface. Both eyes were completely grey and looking
like the televsion production of the Old Wise Man who commenced with grass
hopper before giving advice to the Kung Fu hero. I was aware that our mutual
secret could be exposed so I said loudly ``Leave George to me. I will sort him
out in the cloakroom''. This was greeted with relief from all as dancing was
the joy for all and they returned to the dance floor. The contact lenses were
cleaned and we returned to continue dancing. Sunday afternoon was always the
time to analyse the problems experienced on the dance nights, particularly with
regard to relationships with girls, or lack of them --- a unanimous decision
was then made to ditch the use of the contact lenses, put them away in a drawer
and revert to spectacle use in George's case and act normally.

One spring evening in 1965 a very pleasant man walked into my garden and said:
``Evening Mike. I would like you to help with 2 or 3 others to form a pistol
club. What do you think?'' I told him that I was not sure about revolvers and
fast draw seems to be on film sets but when he produced a collection of
semi-automatic pistols for precision marksmanship my attitude changed
completely. So Wednesday evening found me joining a small group at Jim's house
to discuss the future of the group that formed the Exeter Pistol Club. Ken
Chard, the owner of the gun shop located at Exe Bridge, Exeter, Jim Austin
whose home we used for Committee Meetings, Ernie Hart who had knowledge of
competitive shooting who was the controlling body of the NSRA and yours truly
who had nothing to offer but listen to all that was said agreed to shoot at
Wyvern Barracks on Topsham Road twice a week and join the NSRA postal shooting
competition. I was given a target so I manufactured wooden target holders. We
each put money in the kitty and purchased 2 Russian Volslocks as Club pistols.
I sold my father's 12 bore in exchange for a Smith and Wesson 41. All
administrative and other items were in place ready for the first evening with
the Pistol Shooting Club at Wyvern Barracks. So with targets and new holders 25
yards down the range, tables across the range and a Range Officer wearing an
arm band, 4 of us lined up. Having listened to the instructions of Ken Chard
and Erny Hart and having not taken arm muscle exercise I commenced to fire off
my clip of 10 bullets. It would have been a praise we called awful. Not one
shot hit the paper target but the bits of wood flying off the target holders
gave a good indication of our lack of ability. At the end of one hour of
shooting the holders were just hanging together and looked like they had been
dipped in a tank of sharks. We all slowly improved and with arm exercises,
together with the more we practised, Jim decided that deepened experience was
needed so booked us into Bisley Pistol Championships. On arrival Jim attended
the Office Administrator and in collecting our target numbers and shooting
times he discovered Devon had not entered a team in the County Championships.
When he returned to our camp hut, our lodgings for the night, he (Jim) said,
``My men. I have entered 4 of us in the Devon Pistol team.'' I said, ``What!
You must be off your head. We are only learning to be marksmen. We will never
level with the worse.'' The return comment was that we would be OK and that it
is the experience that counts. With a good nights sleep under our belts we set
forth into the marked areas at Bisley carrying out our own shooting in various
ranges. Our individual results: my Club average was not disappointing I was
told, particularly under competition pressure. Jim went to the official marking
office for the results of the inter County Championships. We, as expected, did
not do well. As Jim returned not sure of our position he told us that it was
very difficult to find the results. Years passed by and we all improved and
were shooting in Division 2. I entered a postal elimination shoot at the Eley
Olympic 1980 competition and ended the postal round in Bisley 60 finalists. In
August Vera and I went to Bisley and to my surprise I finished third. The cut
glass bowl stands in pride of place today. As I was beginning to help charity
organisations I decided to move on from pistol shooting remembering the great
fun and good time over the years, I purchased and gave the Club a cup to be
presented each year to the most improved shot in the last 12 months but I never
heard anything back so they must run without a Chairman or Secretary. I believe
they are shooting somewhere in Pinhoe and I hope they have same enjoyment and
sense of achievement that we had all those years ago.

Working back in time one morning Mr Saunders asked me to take a look at a large
granite roller with a broken main pin with the following remark ``this is a
repair that was executed in the dark ages. I have never carried out such a
repair myself but I know how so I will give you step by step moves.'' First
step, light the blow torch and melt the retaining lead until all clear, gently
pull out the broken shaft together with the many retaining wedges until the
stone hole is clean. Second step: cut a piece of 3/4''round bar and put one end
in the forge and smith a mushroom square, put shaft in stone hole and pack the
sides with 4 mushroom metal 5'' pieces. Third step: manufacture by forge many
wedges thin enough to curl around the larger metal to cause and complete lock
up. It is very important that you listen to the sound of the hammer to stop
high pressure causing severely strong damage. Fourth step: Dig a hole in the
ground to take half the length of the stone roller. Final step: Heat the metal
pin together with retaining wedges hot enough to be tinned with the lead that
is heated to liquid in a pouring ladle (if the lead spits when pouring, heat up
the pin more as it is not hot enough to pour the lead. Keep on until it looks
like liquid. Leave to cool for many hours and return to the customer.

One day after lunch Mr Saunders asked me to get in the car as we were visiting
the water gang working in a very isolated farmstead. We were to look at
designing a small cattle grid. We arrived and proceeded to walk across 2 fields
to locate the men and the mole ploughing exercise. Suddenly Mr Saunders barked
(quote) ``What is the matter with our lorry driver, Arthur Lock.'' I looked in
the direction of his pointed finger and observed Arthur standing as high as
possible lifting his cap up and down on the top of his head and looking in the
direction Arthur was facing, a chain reaction set in. Other members of the team
commenced to repeat the signal. I of course said I had no idea, well knowing it
was the team's warning sign of authority approaching. We continued the visit
taking measurements of the grid. Nothing was said about the incident but I am
sure the boss had worked out the sign because when we returned and opened the
doors of the Triumph Herald to sit in John Saunders removed his cap and
repeated the signal whilst starting the car, a slight smile on his face and
nothing more was said.

The next morning having made the extra early start found me being collected for
Jersey airport by an engineer of south Pier Shipyard who explained the set up
by the harbour. All owners of Fairy power boats were invited to what is
described as a birthday party for the new Fairy Swordsman named Point One. The
plan was that all owners and their ladies congregate with the new owner at St
Hellier harbour to have drinks and as a group make passage to St Marlow in
France for the weekend celebration and return. On arrival I was greeted with
horror of horrors. The quay was lined with groups of power boat owners and
their ladies. The new boat owner who had given his full feelings to the
shipyard manager and looked somewhat flushed greeted me with, ``they can't
expect you to mend it in 10 minutes. I unfortunately found the problem was a
seized clutch and explained that the major failure would result in the gearbox
removal and returning to England for repair. The men and I were greeted with
the words what can be described as extreme disappointment. The boats departed
from St Hellier on passage to St Marlo and me left with the boat to remove the
gearbox. This bit is outside the box --- whilst removing the gearbox and
working with no shirt I became sunburnt and had a bottle of Calomine lotion to
help the soreness. The gear box was loaded on a fairly hefty sailing schooner
on delivery to Hamble. Fog had shut down Jersey Airport so needing to be at
Southampton I went to the docks and signed on as supernumery with a coaster
named Loone Fisher controlled by a Captain Edwards. On arrival at Southampton I
started walking with my heavy holdall of tools making my way to the main gates
when I was picked up by the Customs Officer and taken off the for a strip
search. I remember watching with pleasure as each officer tasted the bottle
contents, shaking their heads and passing it on to the next. Eventually I was
cleared and the gear box rebuilt and I returned to reinstallation.

Another very interesting thing was to attend Fairy Marina at Hamble to carry out
sea and speed trials to a new 45' Fairy Swordsman using the then new 6 cylinder
overhead cam force valve for piston engines. This Company built the Fairy Delta
which went through to 1000 miles per hour with the test pilot named Peter
Twist. Peter had been retained when the aircraft works closed and remained as
test skipper for the power boat division and he skippered the boat on the out
on the Solent so that I could proceed with the engine power test. The very same
boat kicked back at me. The following Friday at 5 o'clock I had just put my
keys in the ignition of the van to go home for a long overdue weekend with Vera
when the field service controller came out of the office waving and making
windmill impressions and said, ``Big problems! The Fairy boat was delivered
that morning and was demonstrating its ability to an invited crowd in St
Helllier harbour when one turning clutch had jammed in gear so I need you on
site tomorrow at 9 o'clock and by the way, Exeter airport is not running so its
back to Southampton for 7 o'clock --- a bit of an early start boy'' poking the
back of my head through the van window.
