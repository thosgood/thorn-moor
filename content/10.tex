%!TEX root = ../thorn-moor.tex

Vera and I sat down and did a rain check on our lifestyles to date. I was living
out of a suitcase so radical changes were made. I joined Saville Tractors at
Marsh Barton as a Workshop Supervisor and life continued normally until two
things happened. Unexpectedly I was quickly promoted to Service Manager, and
the company became main dealers of Listers of Dursley, Glos. This was just a
homecoming for me because, when I was an agricultural engineer in my early
days, the Company held a Lister Generator Dealership, so I was very familiar
with the product. So instead of familiarisation from scratch I immediately ran
with the ball and put a small team together repairing Start-O-Matic generators
and moved into this marginal mains failure generation, winning contracts to
install and service units for the shop retail industry. We had the service
contract for Devon and Cornwall Police radio hilltop sites. Then one morning
there was a major announcement: we had been taken over by a large conglomerate
called Lonhro, and Exeter was to change into a Sales and Service Centre for
heavy German lorries M-A-N and Volkeswagon commercial vehicles. I stood in my
office looking into space and thinking that all my working life I had worked in
the world where we judge reliability of services required by hours worked, and
now we entered the world of miles operated, and as I stood there that morning I
had no idea of what was a good or a bad performance. Nevertheless the boys were
selected in their respective product team to meet the challenge, and meet it
they did. As proof, the M-A-N Volkeswagon Company ran an inter-dealer
competition involving technical questions and performance on call-out section,
and my boys won the Gold Team award for, I think, 5 years running. One award
was a visit to Castle Coombe in Wiltshire to try out race-driving skills on
Formula Ford racing, and another was clay pigeon shooting. Enough awards to
muster confidence. Policies changed within the Lonhro Group and it was not the
thing to operate a service operation. For the first time in my life I was made
redundant.

I then moved into a different world of sales and parts promotion of Lister
components to plant-hire companies, and the sale of marine engines in boat
yards and yacht marinas throughout the south of England. This was a very
satisfactory occupation for me as the company, Sleeman Hawken, had the ability
to provide and supply parts overnight, so it was great to speak to a customer
with confidence. This opened up new avenues. I got the contract to carry all
Hilltop Radio Station general enquiries for Devon \& Cornwall Police. I could
carry out my customer PR visits with great confidence knowing that any Lister
Petter parts ordered would be with the customer within a day or so. The area I
travelled was Cornwall to Poole RNLI Headquarters at Malmsbury, and Wales, from
the M4 to the Gower Peninsula. Vera and I would load the van with three Lister
Marine engines and show-kit and depart on Sundays or Bank Holidays in the
morning, arrive at Newlyn fish market and build up a complete marine engine
show stand. We were ready for the Newlyn fish festival that commenced on Bank
Holiday Monday --- a very popular event that drew in approximately 22,000
visitors. That day engines sold would be followed up with a visit to the
installation site. How useful my past training in Geneva came into play. We
would follow by having a stand in the NEC at Birmingham and one in Lyme Regis.
I was well pleased to be able to load the three Lister marine engines and
show-kit into my Astra van. This was achieved because I owned a three-piece
breakdown engine lift, and on weekends was able to check out Health and Safety
with fingers on the buzzers, i.e. I manufactured an extension on the lifting
arm so angled that it lay parallel with the engine, thus manoeuvring the
clearance between the engine rocker cover to the height of the van roof. With
my minimising engine lift I was self-contained and able to load.

Having changed my lifestyle of working to the present normal weekly hours I
immediately started to put into place my deep wish and mirror my Grandfather's
life to become a bee keeper. So I began to purchase and put together equipment
for the required need from a great friend of mine, Bill Smith of Exeter Bee
Supplies. So I was well into bee keeping with 17 hives: 2 W.B.Cs, and the rest
national hives, with 4 located on the roof of the deep-freeze room in my home,
and some in Days Pottles Lane, Exminster near Exeter, and some beside the
railway line on Marsh Barton. We would spin the honey into honey plastic
buckets which would hold about 1 cwt. With the vast mix of trees around
Middlemore the quality of the honey was exceptional. A shock response greeted
me one evening when I was stung in the face quite by chance, like it or not,
and I pretended it did not happen. A trip to the hospital was very necessary,
resulting in the solemn advice that stings over 14 years had built up, so I was
strongly advised to give up and get rid of the bees before they got rid of me.
This was a shock to my system but nevertheless I was thankful that I had had
the experience of keeping bees.


Then, work from out of the blue! The phone rang one morning, ``Charlie Mann
here. Wonder if you could journey down to me this morning. I have a job I think
you could manage to overcome for me''. Charlie's company over the years had
grasped every opportunity as presented and found themselves in demand for
supplying the film industry with actual mock ups of military vehicles. It
transpires that Colonel Gaddafi of Libya was financing a film with the title
``Lion of the Desert'' involving the factual history of when Italy invaded
Lybia in the 1920s to attempt to increase the expansion of the Roman Empire,
and Gaddafi's own popularity in the world was at that stage at an all-time low
due to the disastrous crime of the Lockerby airline bombing. The problem in
front of us is that the film required some military tanks, Fiat 2-man type. A
mock up had been made from a Land Rover but failed as the wheels had sunk in
the desert sand. This time it was hoped I could recreate the Fiat tank as near
to the real thing as possible by using a bulldozer. Designing the bulldozer to
replicate a Fiat tank was a very hairy affair as Charlie Mann's group of very
serious faced men and Charlie sitting at a table in a portacabin they passed to
me a photocopy pages of the profile of the Fiat tank. One question from them:
``Can you make one like this?''. I was sitting stunned, looking at nothing, my
mind running wild trying to create the mind's eye view of the modifications
required --- 15 minutes thinking time. My response was, ``We will do this for
you''. All stood up and shook hands, in cars and home to Exeter. We selected an
International 125 from the used sales fleet. There were no complicated drawings
made to scale. I just made free-hand drawings as they came into my head. I
redesigned the track layout by extending the length, raising the idler to match
up with the profile of the only copy plans available from Bovington Tank
Museum.  We conducted a completely different discipline of this operation. All
regular and day to day workshop repairs were carried out to 5 o'clock. Then all
stopped, had a break and something to eat thus ensuring we returned to the
project with a completely changed mindset. All bulldozer equipment set was
removed on the yard workshop floor, washed with hot water and detergent with
the machine positioned and all recreations marked and drawn out by chalk on the
floor. My boys responded to the challenge in top shelf manner and, by the end
of the week of evening work, there was a military power unit. The film company
representative came and gave it a full test, resulting in a fully satisfactory
pass and immediately ordered a further five modified units. You can find
information Lion of the Desert in detail online, and if you search for
``military equipment used'' you will see a full photo of my Fiat 2-man tank
conversions with various comments from military experts still trying to
identify the type of tank used. It was nothing like they guessed it to be, but
came out of my head and was made by the boys of Exeter from an International
125.

I then retired and have spent some days on my yacht moored at Dittisham on the
River Dart. I had spent my working days troubleshooting motor vessels and
decided to have peace and quiet, so I purchased a little Super Seal 26' yacht
and have spent time sailing the south coast --- Dartmouth to Salcombe, or
Dartmouth to Exmouth --- nothing too taxing, as I had already accomplished
that. When Vera and I enjoyed our Ruby wedding anniversary we invited friends
and relatives to a Sunday lunch with a request for no presents but, if they
wished to put a donation for the Devon Air Ambulance in a bucket, that would be
appreciated. The following morning we delivered the bucket to the Devon Air
Ambulance HQ to be opened. It was an exceptional surprise to find generous
amounts of money had been donated. We were then asked if we could help as there
was a shortage of speakers. Vera said, ``Mike will be a speaker''. So we joined
the team of speakers, travelling all over Devon. This we continued to do until
the Covid 19 lockdown. In the previous 16 years we have given talks at the
length and breadth of Devon, manning stalls at shows. After this, Vera worked
in the garden whilst I spent time in the garage/workshop making leather swords,
fobs, and belts, as I had taught myself leather work at the beginning of my
story when I mentioned a very young John Anstee who would spend much time with
me as an engineer, who controls a welding and engineering business of his own.
We spent many hours discussing engineering projects from the latest to the old
types of problems, and carried out modifications to a great friend's son's new
kit car. To cast my mind back in time, I was introduced to John Anstee by his
Mum, Elsie, when I was 15 years of age. He was a babe in arms and, as you can
imagine, we lost touch over the years, until the 1990s when John carried out a
specialist stainless steel welding upgrade on my Super Seal 26 boat rudder. Now
standing before me is this very experienced welder with all metal type
experience. From that time on we sailed on the south coast and carried out
mooring repairs at the Ham, Dittisham thus extending our happy boating life
until Covid 19 hit the world.

Vera and I have had great fortune in having such lovely people around us to keep
our early retirement alive, such as Julie, first cousin to Vera (who is
Godmother to Julie's daughter), and her husband Richard, who both chaperoned us
to five unbelievable holidays on Lake Garda, and then to Venice and to Florence
in Tuscany, and kept a close watch on us at all times. Also John and Ann
Anstee, now identifying that the old joints need an extra grease nipple fitted
together with an extra squirt of WD40, collect us and take us for Sunday lunch
and the everlasting friendship of Judy and Gordon Long, always ready to present
a meal, and our very many friends and acquaintances. 

We were very lucky because we Caterpillar men, known as the CAT men, had not met
for 49 years, so I set forth with pen and phone and arranged a reunion on the
15th of February 2020, which developed into a great event at the Beefeater,
Middlemoor, with Devon Life sitting with us and interviewing us all for an
article online. As a band of brothers we had the following boys: Joe Morrish,
aged 94, who was Depot Service Controller; Jack Carter, Depot Parts Manager;
and Colin Wroth, Engine Rebuild Supervisor. Also there was Bob Richards, aged
70ish, Depot Overseas Service Engineer; John Cole, aged 80, Depot and overseas
Service Engineer; Johnny Blackmore, aged 72; and myself, as Field Service
Engineer. We all worked the length of the M5, part of the M4, Telegraph Hill,
and I did a section of the Plymouth Highway whilst Johnny Blackmore, accompanied
by the late Tony Butt, continued to manage the Plympton site (and sometimes I
would break from duty to engage on marine work, as discussed earlier in my
story).

I also remember being in daily contact with Gordon Long. One day I was contacted
by Poltimore Church warden in desperation for a replacement old-fashioned
church key. I told him that I would make a replacement. Having never made a key
before I thought ``this should be interesting'', so I examined the old 8-inch
long item and selected the metal required. I set forth with quite an amount of
hand hacksawing followed by an equal amount of small filing, and it came to
pass that a key presented itself. I ended up making two or three more for
Poltimore, and one for Huxham Church. I returned them with the completion of
the regular 2.5'' key rings, as I remembered at Sunday School in Cheriton
Bishop Mrs Hill explaining that all church keys should have a 2.5'' key ring,
for should the bridegroom or best man forget the wedding ring the vicar would
call the Church Warden to hand over the church key and the special ring was
placed on the bride's finger and duly blessed, so three good things came out of
the manufacturing of the keys. The Church Committees were not overladen with
the cost of a replacement key though these keys were approved of by the old
customers. Two Church Wardens had been re-educated to cope with modern day
weddings should the unthinkable happen.
