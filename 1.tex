%!TEX root = ./thorn-moor.tex

Thorn Moor/Tilery House nestled very comfortably in the moorland countryside
with all its beauty in the summertime, softness in spring and raw brutal
harshness of the long dark winter. However it was well blessed with farmsteads
and families that were great friends, with a community spirit that collectively
pulled us through the season, and the prospective tools required for survival.
In no particular order lies a house named Little Thorn with a Mrs Vodden and
her daughter Beattie, who both visited my mother spending time sharing area
gossip, all very light but good company. About one and a half miles away to the
south east lay Half Acre Farm, occupied in the early days by a Mrs Anstee, a
grandmother to a great friend of mine today who I shall cover later in my
story. This farm, Half Acre was then occupied by a Mr and Mrs Down who joined
in the area community spirit by Mrs Down visiting afternoons to gossip with
mother about local issues and the progress, good or bad, of the war. Mr Down
would visit evenings at about 8 o'clock in rain, snow or wind. Dad would laugh
at the order of his entry — knocking on the back door, door would open and Mr
Down would walk straight in though the living room behind the old Devon settle
shouting ``anybody in''. Dad wondered about a response with ``No''. Much
conversation ensued covering crop growing, threshing, reed combing and market
prices. Refreshments were offered in the nature of sandwiches, tea, and a glass
of cider drawn by jug from the outer cellar. The evening stopped at
approximately 10.30 with a ``Is it that time already?  I better make me way
back.'' Thorn Moor is located next to Thorn Farm and Thorn Cross on the
Cheriton Bishop Road. There is a disused tile yard and kiln within the house
purchased by John Preston Butt, Master Thatcher, which had been converted from
an active tile kiln into three double bedrooms, one single room which I used
until 10 years of age, a front parlour on right side of front door, a front
room used at Christmas on the left of front door (kiln area), kitchen with cold
water tap, walk-in larder with hooks in the ceiling to hang quarters of cured
meats, covered rear door area for clothes washing and killing pigs at times.
One divided shed: one part used for Grandfather to keep his pony in his
thatching days and the second part with a built-in still to distil cider into
calvados as well as for storing my motorbike and engine parts. The south-west
side was built with a long flat roof to accommodate four milking cows, thus
making natural use of warm animal body mass to heat up the building of weather
walls.

Travelling around in clockwise direction, approximately 3 miles as the crow
fliesm situated in the village of Crockernwell were various services — Tom
Ching the saddler and cobbler would provide and repair new shoes, boots and
leather together with nails, scoots and studs to enable Dad to carry out
repairs at home. Tom Ching and Dad were good friends and both were members of
the Cheriton Bishop Council. Tom was the Clerk, and had copperplate
handwriting, and Dad said whilst a Parish Council meeting was in progress and
controversial subjects were discussed Tom would show distress by running his
fingers around his shirt collar to relieve excess body heat.

Standing between Ching's Saddlers and the pub, then the Royal Hotel, was a
wheelwright and hardware store owned by Reg Stanbury, the son of a world-famous
clay pigeon shooter. People including ourselves would visit Reg at all times of
the day to purchase a vast variety of working items: a dry battery for the
radio, rabbit wires, axes, bicycle tyres, nails and screws. The visit incurred
the same format — knocking and entering the workshop. Reg was normally working
at his work bench and the initial conversation was always the family health
followed by the present day weather. Nothing further happened until a well-worn
shiny tobacco tin was removed from the top of his bib and brace overalls and a
very carefully rolled cigarette had been prepared and lit. Then action
commenced with the following words ``let's go and see what we've got for
you''.

The village pub, The Royal Hotel, was tended by a Mr Tom Edwards over the war
years. It later passed to the hands of David Sibbles, his wife, Irene, and his
mother. They transformed the old look to a very new and at the time modern look
to include a cafeteria in the large main hall. They subsequently obtained a
contract with the Royal Blue coaches, usually passing through to Cornwall, for
the drivers to have their required breaks and passengers to receive refreshment
including sometimes overnight stays, sometimes shorter mid-day or comfort
breaks. They were one of the very few establishments locally to install a juke
box, which attracted hoards of teenagers from miles around, mostly on Saturday
and Sunday evenings. At the top of the village was a garage owned by a Mr
Watts, who would attend my home driving an Armstrong Siddley car wearing a
shiny peaked chauffeur's cap and delivering John Preston Butt to his daughter,
Hilda, at Kenton for a holiday. After some years the garage name was changed to
Crockernwell Motors and a new proprietor came in named Jim Sharp. The nature of
the workshop changed with a great deal of fun attached to the overhaul
operation. Jim would take a chance on anything with little regard for Health
and Safety and as I used to help him of an evening for extra cash my attitude
was as a young man equally maverick. One evening I arrived at the garage and
Jim greeted me full of smiles and said ``I've got just the job for you!  We
have a car to rescue from suicide corner near the Bay Tree Motel.''  On arrival
we found an Austin A35 had failed to negotiate the bend and had finished its
journey 15' up the bank wedged at an angle between two hazel bushes. I then sat
in the driver's seat whilst Jim in a Land Rover towed us off. At this stage
quite a crowd had gathered to view the action. I have no idea how the car
remained on its four wheels but it did. I was told that the gathered crowd was
a little disappointed that the Austin did not turn over. However, much back
slapping was rendered to all. Moving down to Hooperton Cross road and turning
into Thorn Road, we find Hooperton Farm. When I was a small boy I remember it
was farmed by Mr and Mrs Wreford until age forced them to retire and move to a
bungalow in Cheriton Bishop. A Mr and Mrs Wotton and their son, Francis, became
well-known walking in the area training their greyhounds. He would take me out
on a dark blustery evening and taught me the skill of catching rabbits with a
spot lamp and whippet. He was well skilled in the art of rabbit netting with a
ferret. Unfortunately Francis' happy lifestyle ended in such a tragic manner
that I am not prepared to expand any further.

The next farm is Bowden which was always very popular with small boys. When
walking by with Mum and Dad we saw a well-rotted tree stump which had twisted
into the shape of a dragon's head, and with the able assistance of outside help
had an interior red painted mouth and a red rubber tongue hanging out and two
carefully placed cycle reflectors to illuminate the eyes. This savage beast
came alive at night when passing car lights flashed by. The owners of Bowden
were Donald Seagus and his son. I lived in heaven as evenings I was lent a
BSA.22 five clip rifle. This might shock some people as I was only 14 years of
age.
